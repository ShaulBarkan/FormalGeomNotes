
\chapter{The $\sigma$-orientation}

\section{The splitting principle}

Here's some old stuff from when I thought it was good to prove the splitting principle for real bundles in the first case study.  This \emph{may} all be needed earlier than this, when we go through a proof of Quillen's theorem on $MU$.  Maybe I'll defer it then too.

---

Our first goal for today is to show the following freeness property of the ring spectrum $MO$:
\begin{theorem}
Let $E$ be a ring spectrum.  Homotopy classes of ring maps $MO \to E$ are in natural bijection with factorizations \[\S \to MO(1) \to E\] of the unit map for $E$. \qed
\end{theorem}

\noindent This proof falls into two halves, and one half is much easier than the other.  The data of a ring map $MO \to E$ appears to be considerly more data than a factorization, and showing that one begets the other turns out to be the easier direction of the proof.  Suppose that we're given such a ring map $MO \to E$, so that we can apply the Thom isomorphism machinery from the beginning of this story.  Then, recall the definition of $MO(1)$ as a Thom spectrum: \[MO(1) = T(\L - 1 \downarrow \RP^\infty).\]  Restricting the base space all the way to a point gives \[T(\L - 1 \downarrow *) = \S,\] and this fits into the following commutative diagram with the ring map we were given:
\begin{center}
\begin{tikzcd}
\S \arrow{r} \arrow{d} & MO \arrow{d} \\
MO(1) \arrow{ru} \arrow[densely dotted]{r} & E.
\end{tikzcd}
\end{center}
The horizontal arrow across the top is the unit map for $MO$, so the long composite is the unit map for $E$, and the dotted composite is the desired factorization of the unit.  In terms of cohomology classes, the Thom isomorphism gives \[E^* \RP^\infty \cong \widetilde E^* MO(1).\]  The left--hand group has the canonical element ``$1$'', and the data of ``$\cong$'' is the Thom isomorphism, sending $1$ to a canonical map $MO(1) \to E$.  This, too, is the dotted arrow.\todo{Can this all be phrased more clearly?}

Remembering that $MO(1) \simeq \Susp^{-1} \Susp^\infty \RP^\infty$, we see that this was what powered our computation of $H\F_2^* \RP^\infty$ from earlier, and in fact this map $MO(1) \to E$ is enough to deduce an Thom isomorphism in $E$--cohomology for $MO(1)$ alone.  The other direction of the proof then sounds more serious: we have to show that if we have a Thom isomorphism for the bundle involved in forming $MO(1)$, then we can extract from that compatible Thom isomorphisms for all bundles.  This kind of reduction is famous enough to have a name: it is called ``the splitting principle''.

\todo{Keep talking.}

------

The main point is that the fiber map $\CP^{n-1} \to P(V) \to X$ postcomposes by $P(V) \to \CP^\infty$ to the skeletal inclusion, so the Serre spectral sequence degenerates since its fiber is unmolested (or: the Leray--Hirsch theorem). The Chern classes are \emph{defined} by the one remaining multiplicative extension.



\section{Divisors and divisor schemes}

Divisors and schemes of divisors on $1$-dimensional formal groups

\section{Schemes for finite $K$--theory spaces}

$BU(n)_E$, $BU_E$, and $(BU \times \Z)_E$

$BSU_E$ and special divisors
