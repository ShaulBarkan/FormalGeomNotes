% -*- root: main.tex -*-

\chapter{The $\sigma$-orientation}\label{ChapterSigmaOrientation}


\todo[inline]{Write an introduction for me.  Use unstable cooperations from Morava's theories to classical complex and real $K$--theory.}

\todo[inline]{Part of the theme of this chapter should be to use the homomorphism from topological vector bundles to algebraic line bundles --- Neil's $\mathbb L$ construction --- as inspiration for what to do, given suitable algebraic background.}




\section{Coalgebraic formal schemes}

\begin{definition}
The scheme associated to a coalgebra $C$ over a field $k$ is defined by the formula \[(\Sch C)(T) = \left\{f \in C \otimes T \middle| \begin{array}{c} \Delta f = f \otimes f \in (C \otimes T) \otimes_T (C \otimes T), \\ \eps u = 1 \end{array} \right\}.\]
\end{definition}

\begin{lemma}
If $A$ is a $k$--algebra that is finite--dimensional as a $k$--module, then $\Spec A \cong \Sch A^*$. \qed
\end{lemma}

\begin{lemma}\label{kCoalgebrasAreIndFinite}
For $C$ a coalgebra over a field $k$, any finite--dimensional $k$--linear subspace of $C$ can be finitely enlarged to a subcoalgebra of $C$.  Accordingly, taking the colimit gives a canonical equivalence \[\Ind(\CatOf{Coalgebras}_k^{\mathrm{fin}}) \xrightarrow{\simeq} \CatOf{Coalgebras}_k. \qed\]
\end{lemma}

This is supposed to indicate that coalgebras \emph{want} to model formal schemes.

\begin{corollary}\label{CoalgsAndFSchsAgreeOverk}
For $C$ a coalgebra over a field $k$ expressed as a colimit $C = \colim_k C_k$ of finite subcoalgebras, there is an equivalence \[\Sch C \cong \{\Spec C_k^*\}_k.\]  This induces an equivalence of categories \[\CatOf{Coalgebras}_k \cong \CatOf{FormalSchemes}_{\Spec k}. \qed\]
\end{corollary}

This covariant algebraic model for formal schemes is very useful.  \todo{Do you also need to compare the (Cartesian) monoidal structures?}  For instance, this equivalence makes the following calculation trivial:
\begin{lemma}
Select a coalgebra $C$ over a field $k$ together with a pointing $k \to C$.  Write $M$ for the coideal $M = C / k$.  The free formal group on the pointed formal scheme $\Sch k \to \Sch C$ is given by \todo{Describe the diagonal map on this guy.} \[F(\Sch k \to \Sch C) = \Sch \Sym_k M. \qed\]
\end{lemma}

It is unfortunate, then, that when working over an object more general than a field \Cref{kCoalgebrasAreIndFinite} fails.  Nonetheless, it is possible to bake into the definitions the machinery needed to get an analogue of \Cref{CoalgsAndFSchsAgreeOverk}.

\begin{definition}
Let $C$ be an $R$--coalgebra which is free as an $R$--module.  A basis $\{x_j\}$ of $C$ is said to be a \textit{good basis} when any finite subcollection of $\{x_j\}$ can be finitely enlarged to a subcollection that spans a subcoalgebra.  The coalgebra $C$ is itself said to be \textit{good} when it admits a good basis.  A formal scheme $X$ is said to be \textit{coalgebraic} when it is isomorphic to $\Sch C$ for a good coalgebra $C$.
\end{definition}

\begin{theorem}[{\cite[Proposition 4.64]{StricklandFSFG}}]
Suppose that $F\co \CatOf I \to \CatOf{Coalgebras}_R$ is a colimit diagram of coalgebras such that each object in the diagram is a good algebra.  Then \[\Sch \circ F\co \CatOf I \to \CatOf{FormalSchemes}\] is also a colimit diagram. \qed
\end{theorem}

\begin{corollary}
For $X$ a coalgebraic formal scheme, $X^{\times n}_{\Sigma_n}$ exists. \qed
\end{corollary}


Prop 6.4 (of Section 6.1 of FSFG): The coproduct $\coprod_{n \ge 0} X^{\times n}_{\Sigma_n}$ models the free formal monoid

Section 6.2 of FSFG: Free formal groups and how the free pointed monoid on a formal curve is automatically group-complete






Use this to construct the various stable divisor schemes and verify the freeness assertions about them (i.e., compare with the homology of $BU$)



Cartier duality

% Cartier duality: There is a duality on commutative and cocommutative finite--dimensional Hopf algebras, given by taking the linear dual.  There is also a duality on Dieudonn\'e modules, also given by taking the linear dual.  These also relate the two kind of Dieudonn\'e functors described above.



Maybe build the ring $C_*(\G)$?  Or at least indicate that it's possible?





\section{Special divisors and $MSU$--orientations}

Starting today, after our extended interludes on chromatic homotopy theory and cooperations, we are going to return to thinking about bordism orientations directly.  To start, we will recall the various perspectives adopted in \Cref{ComplexBordismChapter} when we were studying complex-orientations.
\begin{itemize}
\item A complex-orientation of $E$ is, definitionally, a map $MUP \to E$ of ring spectra in the homotopy category.
\item Such a map is equivalent to specifying a multiplicative system of Thom isomorphisms for complex vector bundles.  Using the splitting principle, it suffices to consider just complex line bundles, and just the universal line bundle $\L$ over $\CP^\infty$ at that.  We can phrase what we want algebro-geometrically: a Thom isomorphism is the data of a trivialization of the Thom sheaf $\ThomSheaf{\L}$ over $\CP^\infty_E$.
\item Because $E_0 MU$ is a free $E_0$--module, ring spectrum maps $MUP \to E$ biject with $E_0$--algebra maps $E_0 MUP \to E_0$.  This, too, can be phrased algebro-geometrically: these are elements of $(\Spec E_0 MUP)(E_0)$.\todo{AHS blame this on an Atiyah--Hirzebruch spectral sequence, but I'm still not sure I believe this.}
\end{itemize}
We can summarize our main result about these as follows:
\begin{theorem}[{\cite[Example 2.53]{AHSTheoremOfTheCube}}]\label{BUZTriumvirate}\citeme{Put in cross references}
The functor \[(\Spec T \xrightarrow{u} \Spec E_0) \mapsto \left\{ \text{trivializations of $u^* \ThomSheaf{\L}$ over $u^* \CP^\infty_E$} \right\}\] is isomorphic to the affine scheme $\Spec E_0 MUP$.  Moreover, the $E_0$--points of this scheme biject with ring spectrum maps $MUP \to E$. \qed
\end{theorem}

An analogous result holds for ring spectrum maps $MU \to E$ and the line bundle $\L - 1$, and it again is easiest expressed by an extension of the splitting principle.  Specifically, given finite complex $X$ and a rank $0$ virtual vector bundle over it classified by a map \[\tilde V \co X \to BU,\] there exists an integer $n$ so that $\tilde V = V - n \cdot 1$ for an honest rank $n$ vector bundle $V$ over $X$.  Using \Cref{OriginalSplittingPrinciple}, the splitting $f^* V \cong \bigoplus_{i=1}^n \L_i$ over $Y$ also gives an \emph{internal splitting} of $\tilde V$ as \[\tilde V = V - n \cdot 1 = \bigoplus_{i=1}^n (\L_i - 1),\] where each bundle $\L_i - 1$ naturally has the structure of a rank $0$ virtual vector bundle, and the sum is taken internally to $BU$.  This begets the following extension of the previous result:
\begin{theorem}[{\cite[Example 2.54]{AHSTheoremOfTheCube}}]\label{BUTriumvirate}\citeme{Put in cross references}
The functor \[\{\Spec T \xrightarrow{u} \Spec E_0\} \to \left\{ \text{trivializations of $u^* \ThomSheaf{\L - 1}$ over $u^* \CP^\infty_E$} \right\}\] is isomorphic to the affine scheme $\Spec E_0 MU$.  Moreover, the $E_0$--points of this scheme biject with ring spectrum maps $MU \to E$. \qed
\todo{Identify this bundle $\ThomSheaf{\L - 1}$ as $\omega \otimes \sheaf I(0)^{-1}$, we can think of sections as elements of $E^0 T(\L - 1 \to \CP^\infty)$ which restrict to the identity under the inclusion \[S^0 \to P^{\L-1}.\]}
\end{theorem}

\todo{Identify the restriction orientation $MU \to MUP \to E$.  Make absolutely clear that a section $f$ is sent to the normalized object $f'(0) / f$.}

The special unitary group $SU$ is explicit enough that orientations with source $MSU$ can be fully understood along similar lines.  Our jumping off point for that story will be, again, an extension of the splitting principle.
\begin{lemma}\label{SplittingPrincipleForBSU}
Let $X$ be a finite complex, and let $\tilde V\co X \to BU$ classify a virtual vector bundle of rank $0$ over $X$.  Select a factorization $\tilde{\tilde V}\co X \to BSU$ of $\tilde V$ through $BSU$.  Then, there is a space $f\co Y \to X$ over $X$ and a collection of line bundles $\sheaf H_j$ expressing a $BSU$--internal decomposition \[\tilde{\tilde V} = -\bigoplus_{j=1}^n (\sheaf H_j - 1)(\sheaf H'_j - 1).\]
\end{lemma}
\begin{proof}
Begin by using \Cref{OriginalSplittingPrinciple} on $V$ to get an equality of $BU$--bundles \[\tilde{\tilde V} = V' + \L_1 + \L_2 - n \cdot 1.\]  Adding $(\L_1 - 1)(\L_2 - 1)$ to both sides, this gives
\begin{align*}
\tilde{\tilde V} + (\L_1 - 1)(\L_2 - 1) & = V' + \L_1 + \L_2 + (\L_1 - 1)(\L_2 - 1) - n \cdot 1 \\
& = V' + \L_1\L_2 - (n-1) \cdot 1.
\end{align*}
By thinking of $(\L_j - 1)$ as an element of $kU^2(Y) = \CatOf{Spaces}(Y, BU)$, we see that $(\L_1 - 1)(\L_2 - 1)$ has the natural structure of a $BSU$--bundle and hence so does the sum on the left-hand side\footnote{Really, we are using the Hopf ring $\circ$--product.}.  The right-hand side is the rank $0$ virtualization of a rank $(n-1)$ vector bundle, hence succumbs to induction.  Finally, because $SU(1)$ is the trivial group, there are no nontrivial complex line bundles with structure group $SU(1)$, grounding the induction.
\end{proof}
\begin{corollary}
Ring spectrum maps $MSU \to E$ biject with trivializations of \[\ThomSheaf{(\L_1 - 1)(\L_2 - 1)} \downarrow (\CP^\infty)^{\times 2}_E. \qed \]
\end{corollary}

If we can show that $E_0 BSU$ is free as an $E_0$--module, then this will complete the $BSU$--analogue of \Cref{BUZTriumvirate,BUTriumvirate}.  This is quite easy, following directly from the Serre spectral sequence:
\begin{lemma}[{\cite[Lemma 6.1]{AndoStrickland}}]\label{BSUtoBUtoCPinftyIsSexseq}
The Postnikov fibration \[BSU \to BU \xrightarrow{B\det} BU(1)\] induces a short exact sequence of Hopf algebras \[E^0 BSU \from E^0 BU \xleftarrow{c_1 \mapsfrom c_1} E^0 BU(1). \qed\]
\end{lemma}
\begin{corollary}\label{BSUTriumvirate}
The functor \[\{\Spec T \xrightarrow{u} \Spec E_0\} \to \left\{ \text{trivializations of $u^* \ThomSheaf{(\L_1 - 1)(\L_2 - 1)}$ over $u^* \CP^\infty_E$} \right\}\] is isomorphic to the affine scheme $\Spec E_0 MSU$.  Moreover, the $E_0$--points of this scheme biject with ring spectrum maps $MSU \to E$. \qed
\end{corollary}

However, the use of \Cref{BSUtoBUtoCPinftyIsSexseq} inspires us to spend a moment longer with the associated formal schemes.  An equivalent statement is that there is a short exact sequence of formal group schemes
\begin{center}
\begin{tikzcd}
BSU_E \arrow{r} \arrow[-,double]{d} & BU_E \arrow{r}{B\det} \arrow[-,double]{d} & BU(1) \arrow[-,double]{d} \\
\SDiv_0 \CP^\infty_E \arrow{r} & \Div_0 \CP^\infty_E \arrow{r}{\mathrm{sum}} & \CP^\infty_E,
\end{tikzcd}
\end{center}
where the scheme ``$\SDiv_0 \CP^\infty$'' of ``special divisors'' consists of those divisors which vanish under the summation map.  However, where the comparison map $BU(1)_E \to BU_E$ has a universal property --- it presents $BU_E$ as the universal formal group on the pointed curve $BU(1)_E$ --- the description of $BSU_E$ as a scheme of special divisors does not bear much immediate resemblance to a free object on the special divisor $([a] - [0])([b] - [0])$ classified by \[(\CP^\infty)^{\times 2}_E \xrightarrow{(\L_1 - 1)(\L_2 - 1)_E} BSU_E \to BU_E = \Div_0 \CP^\infty_E.\]  We will spend the rest of the day straightening this point out.








Construct $C_2 \G$.



\begin{align*}
\phi_n \co \G^{\times n} / \Sigma_n & \to C_2(\G), \\
\sum_{i=1}^n [a_i] & \mapsto \sum_{i=1}^n \left[\sum_{j=1}^i a_j, a_i \right].
\end{align*}

\citeme{Neil's FSKS note, Lemma 2.9} The map $\phi_n$ is invariant under $\Sigma_n$ and extends as $n \to \infty$.  \citeme{Lemma 2.10} It's almost a homomorphism: \[\phi_{n+m}(\underline a, \underline b) = \phi_n(\underline a) + \phi_m(\underline b) + [\sigma(a), \sigma(b)].\]  \citeme{Lemma 2.11} It's also cancellative, if you set \[\phi_{n, m}(\underline a, \underline b) = \phi_n(\underline a) - \phi_m(\underline b) - [\sigma(a), \sigma(b)] + [\sigma(b), \sigma(b)]\] so that \[\phi_{n+k,m+k}((\underline a, \underline c), (\underline b, \underline c)) = \phi_{n,m}(\underline a, \underline b).\]  This means that it extends to a \emph{map} off of $C_1(\G)$, and it becomes a homomorphism to $C_2(\G)$ when restricted to $\ker(\sigma)$.  The main claim\citeme{Proposition 2.13} is that this gives an inverse to $\delta\co C_2(\G) \to \ker(\sigma) \subseteq C_1(\G)$.











Note that the Cartier dual object $C^2(\G, \Gm)$ is much easier to build: given a coordinate on $\G$, this has an obvious affine presentation.

Identify the Thom sheaf of the universal bundle on $BSU$ as $C^2(\G; \sheaf I(0))$, together with its $C^2(\G, \Gm)$--torsor structure, and note that this automatically promotes the classifying map $MSU^E \to C^2(\G; \sheaf I(0))$ to an isomorphism. 

Identify the restriction orientation $MSU \to MU \xrightarrow{s} E$ as $\delta(s_E)$

\todo[inline]{Is it possible to give an example of a complex-oriented theory which receives an $MSU$ orientation which \emph{does not} factor the complex orientation but \emph{does} (\emph{must}, really) factor the unit?  Even if one can find an example of this, I think it will be somewhat artificial: the sequence of group schemes \[0 \to BSU_E \to BU_E \to BU(1)_E \to 0\] is short exact, \emph{and} it has a splitting on the level of formal schemes.  The splitting is what gives you the isomorphism on points $BSU_E(T) \times BU(1)_E(T) \cong BU_E(T)$.  On the other hand, because the splitting \emph{isn't} a map of formal groups, it doesn't survive to the Cartier dual short exact sequence \[0 \from BSU^E \from BU^E \from BU(1)^E \from 0,\] so this will come down to exhibiting a test ring $T = E_*$ for which $BU^E(T) \to BSU^E(T)$, despite being induced by an fppf-surjective map of sheaves, is not surjective on sections over $T$.  Of course, this comes down concretely to solving for a preimage under the map \[1 - g(x) \mapsto \frac{1 - g(x +_{\G} y)}{(1 - g(x))(1 - g(y))},\] which is more plodding but might offer insight into what sort of ring (and thus orientation) you're looking for.}


------


\begin{definition}
Let $X$ be a group object in $\CatOf{Spaces}$ or $\CatOf{Spectra}$ so that $E_* X$ is concentrated in even degrees.\todo{I'm not sure this is sufficient.  Possibly $H_* X$ needs to be even too.}  We define \[X^E := \Spec E_* X.\]  This construction is contravariantly functorial in the choice $X$.\todo{As recently pointed out on ALGTOP-L, there's not actually a category of $H$--spaces, so maybe ``functorial'' is not a good word here.}
\end{definition}

\todo[inline]{Relationship between $X^E$ and $X_E$ for a nice $H$--space $X$: linear duality and Cartier duality.}

$E_* BU$ is the symmetric algebra on $E_* \CP^\infty$ and $E_* (BU \times \Z)$ is that tensored with the group ring on $\Z$, i.e., with an extra invertible generator.  A coordinate is a map $\G_E \to \A^1$ which is an isomorphism of formal varieties, but not anything special w/r/t the group structure of $\G_E$, but $\A^1$ is also isomorphic to $\G_a$, and so it induces a map $\Div \G_E \to \A^1$ off of the free object.

A line bundle is a $\Gm$--torsor, btw.  If you use $\sheaf O_{\G}$ as the line bundle, all these things should degenerate to the usual ones.  And, so, if you have a trivializable bundle, a trivialization of $\L$ should induce a comparison of the $\L$--twisted thing with the untwisted thing.

\citeme{This is Section 2.4.5 of the published AHS} The map $\rho_k: P^k \to BU\<2k\>$ pulls back the tautological bundle over $BU\<2k\>$ to $V$, the external tensor product of the reduced tautological bundles.  Passing to Thom spectra gives a map $s_k: (P^k)^V \to MU\<2k\>$, and since $\ThomSheaf(\L) = \sheaf I(0)$ we get $\ThomSheaf{V} = \Theta^k(\sheaf I(0))$ on $P^k_E$.  The map $s_k$ is supposed to give a section of the pullback of $\Theta^k(\sheaf I(0))$ along the projection $MU\<2k\>^E \to S_E$.  More than that, $s_k$ is a $\Theta^k$--structure, since you can check those cocycle identities on $V$ itself, and so there's a classifying map \[MU\<2k\>^E \to C^k(\CP^\infty_E; \sheaf I(0)).\]  Since we calculated $BU\<2k\>^E \cong C^k(\CP^\infty_E, \Gm)$, we know that $C^k(\CP^\infty_E; \sheaf I(0))$ is a torsor for the group scheme $BU\<2k\>^E$.  So, we just need to check that the displaymode map is a map of torsors.  (Along the way, we should be able to check that the natural map $MU\<2k+2\> \to MU\<2k\>$ induces the map $C^k(\CP^\infty_E; \sheaf I(0)) \to C^{k+1}(\CP^\infty_E; \sheaf I(0))$.)  Matt's proof of this looks kind of vacuous\citeme{Theorem 2.50 of published AHS}, but I guess a key step is looking at the $V$ situation:
\begin{center}
\begin{tikzcd}
T(V \to (\CP^\infty)^k) \arrow{r}{\Delta} \arrow{d} & T(V \to (\CP^\infty)^k) \sm (\CP^\infty)^k_+ \arrow{d} \\
MU\<2k\> \arrow{r}{\Delta} & MU\<2k\> \sm BU\<2k\>_+.
\end{tikzcd}
\end{center}









\section{Elliptic curves and $\theta$--functions}

\todo{The goal of this lecture should be to set up all the algebraic geometry we'll need, in a coherent-enough way that the students will be able to think back and at least mumble ``yeah, ok, reasonable''.}

Today will constitute something of a r\'esum\'e on elliptic curves.  We'll hardly prove anything, and we also won't cover many topics that a sane introduction to elliptic curves would make a point to cover.  Instead, we'll try to restrict attention to those concepts which will be of immediate use to us in the coming couple of lectures.

To begin with, recall that an elliptic curve in the complex setting is a torus, and it admits a presentation by selecting a lattice $\Lambda$ of full rank in $\C$ and forming the quotient \[\C \xrightarrow{\pi_\Lambda} E_\Lambda = \C / \Lambda.\]  The meromorphic functions $f$ on $E_\Lambda$ pull back to give meromorphic functions $\pi_\Lambda^* f$ on $\C$ satisfying a periodicity constraint in the form of the functional equation \[\pi_\Lambda^* f(z + \Lambda) = \pi_\Lambda^* f(z).\]  From this, it follows that there are no holomorphic such functions, save the constants --- such a function would be bounded, and Liouville's theorem would apply.  It is, however, possible to build the following meromorphic special function, which has poles of order $2$ at the lattice points: \[\wp_\Lambda(z) = \frac{1}{z^2} + \sum_{\omega \in \Lambda \setminus \{0\}} \frac{1}{(z - \omega)^2} - \frac{1}{\omega^2}.\]  Its derivative is also a meromorphic function satisfying the periodicity constraint: \[\wp_\Lambda'(z) = -2 \sum_{\omega \in \Lambda} \frac{1}{(z - \omega)^3}.\]  In fact, these two functions generate all other meromorphic functions on $E_\Lambda$, in the sense that the subsheaf spanned by the algebra generators $\wp_\Lambda$ and $\wp_\Lambda'$ is exactly $\pi_\Lambda^* \sheaf M_{E_\Lambda}$.  This algebra is subject to the following relation, in the form of a differential equation: \[\wp_\Lambda'(z)^2 = 4 \wp_\Lambda(z)^3 - g_2(\Lambda) \wp_\Lambda(z) - g_3(\Lambda),\] for some special values $g_2(\Lambda)$ and $g_3(\Lambda)$.  Accordingly, writing $C \subseteq \CP^2$ for the projective curve $wy^2 = 4x^3 - g_2(\Lambda) w^2 x - g_3(\Lambda) w^3$, there is an analytic group isomorphism
\begin{align*}
E_\Lambda & \to C, \\
z \pmod \Lambda & \mapsto (1, \wp_\Lambda(z), \wp_\Lambda'(z)).
\end{align*}
This is sometimes referred to as the Weierstrass presentation of $E_\Lambda$.

There is a second standard embedding of a complex elliptic curve into projective space, using \textit{$\theta$--functions}, which are most naturally expressed \emph{multiplicatively}.  To begin, select a lattice $\Lambda$ and a basis for it, and rescale the lattice so that the basis takes the form $\{1, \tau\}$ with $\tau$ in the upper half-plane.  Then, the normalized exponential function $z \mapsto \exp(2 \pi i z)$ has $1 \cdot \Z$ as its kernel, and setting $q = \exp(2 \pi i \tau)$ we get a second presentation of $E_\Lambda$ as $\C^\times / q^{\Z}$.\todo{I think it's helpful to draw a picture here of an annulus with some identification made.}

The associated $\theta$--function is defined by \[\theta_q(u) = \prod_{m \ge 1} (1 - q^m) (1 + q^{m-\frac{1}{2}}u) (1 + q^{m-\frac{1}{2}}u^{-1}) = \sum_{n \in \Z} u^n q^{\frac{1}{2} n^2}.\]  It vanishes on the set $\exp(2 \pi i (\frac{1}{2}m + \frac{\tau}{2}n))$, the center of the fundamental annulus.  However, since it has no poles it cannot descend to give a function on $\C^\times / q^{\Z}$.  A different obstruction to this descent is its imperfect periodicity relation: \[\theta_q(qu) = u^{-1} q^{\frac{-1}{2}} \theta_q(u).\]  We can also shift the zero-set of $\theta_q$ by rational rescalings $a$ of $q$ and $b$ of $1$: \[\theta_q^{a,b}(z) = q^{\frac{a^2}{2}} \cdot u^a \cdot \exp(2 \pi i a b) \theta_q(u q^a \exp(2 \pi i b)).\]

\begin{remark}[{\cite[Proposition 10.2.6]{Husemoller}}]
For any $N > 0$, define $V_\tau[N]$ to be the space of entire functions $f$ with $f(z + N) = f(z)$ and $f(z + \tau) = e^{-2 \pi i N z - \pi i N^2 \tau} f(z)$.  Then, $V_\tau[N]$ has $\C$--dimension $N^2$, and the functions $\theta_\tau^{a, b}$ give a basis by picking representatives $(a_i, b_i)$ of the classes in $((1/N)\Z / \Z)^2$.
\end{remark}

Even though these functions do not themselves descend to $\C^\times / q^{\Z}$, we can collectively use them to construct a map to complex projective space, where the quasi-periodicity relations will mutually cancel in homogeneous coordinates.
\begin{theorem}[{\cite[Proposition 10.3.2]{Husemoller}}]
Consider the map
\begin{align*}
\C / N(\Lambda) & \xrightarrow{f_{(N)}} \P^{N^2-1}(\C), \\
z & \mapsto [\cdots : \theta_\tau^{i/N, j/N}(z) : \cdots].
\end{align*}
For $N > 1$, this map is an embedding. \qed
\end{theorem}

\begin{example}
One can work out how it goes for $N = 2$, which will cause some of our pesky $\frac{1}{2}$'s to cancel.  The four functions there are $\theta_q^{0,0}$ with zeroes on $\Lambda + \frac{\tau + 1}{2}$, $\theta_q^{0,1/2}$ with zeroes on $\Lambda + \frac{\tau}{2}$, $\theta_q^{1/2,0}$ with zeroes on $\Lambda + \frac{1}{2}$, and $\theta_q^{1/2,1/2}$ with zeroes on $\Lambda$ exactly.  The image of $f_{(2)}$ in $\P^{2^2 - 1}(\C)$ is cut out by the equations
\begin{align*}
A^2 x_0^2 & = B^2 x_1^2 + C^2 x_2^2, &
A^2 x_3^2 & = C^2 x_1^2 - B^2 x_2^2,
\end{align*}
where
\begin{align*}
x_0 & = \theta_\tau^{0, 0}(2z), &
x_1 & = \theta_\tau^{0, 1/2}(2z), &
x_2 & = \theta_\tau^{1/2, 0}(2z), &
x_3 & = \theta_\tau^{1/2, 1/2}(2z)
\end{align*}
and
\begin{align*}
A & = \theta_\tau^{0, 0}(0) = \sum_n q^{n^2}, &
B & = \theta_\tau^{0, 1/2}(0) = \sum_n (-1)^n q^{n^2}, &
C & = \theta_\tau^{1/2, 0}(0) = \sum_n q^{(n + 1/2)^2}
\end{align*}
upon which there is the additional ``Jacobi'' relation \[A^4 = B^4 + C^4.\]
\end{example}

\begin{remark}
This embedding of $E_\Lambda$ as an intersection of quadratic hypersurfaces in $\CP^3$ is quite different from the Weierstrass embedding.  Nonetheless, the embeddings are analytically related.  Namely, there is an equality \[\frac{d^2}{dz^2} \log \theta_{\exp{2 \pi i \tau}}(\exp{2 \pi i z}) = \wp_\Lambda(z).\]  Separately, Weierstrass considered a function $\sigma_\Lambda$, defined by \[\sigma_\Lambda(z) = z \prod_{\omega \in \Lambda \setminus 0} \left( 1 - \frac{z}{\omega} \right) \cdot \exp \left[ \frac{z}{\omega} + \frac{1}{2} \left( \frac{z}{\omega} \right)^2 \right],\] which also has the property that its second logarithmic derivative is $\wp$ and so is ``basically $\theta_q^{1/2,1/2}$''.  In fact, any elliptic function can be written in the form \[c \cdot \prod_{i=1}^n \frac{\sigma_\Lambda(z - a_i)}{\sigma_\Lambda(z - b_i)}.\]
\end{remark}

The $\theta$--functions version of the story has two main successes.  One is that there is a version of this story for an arbitrary abelian variety.  It turns out that all abelian varieties are projective\citeme{Milne's abelian varieties, Theorem 7.1}, and the theorem sitting at the heart of this claim is
\begin{corollary}[Theorem of the cube]\label{Theta3IsTrivial}\citeme{Milne's Abelian Varieties chapter, Corollary 6.4 and Theorem 7.1}
Let $A$ be an abelian variety, let $p_i: A \times A \times A \to A$ be the projection onto the $i${\th} factor, and let $p_{ij} = p_i +_A p_j$, $p_{ijk} = p_i +_A p_j +_A p_k$.  Then for any invertible sheaf $\L$ on $A$, the sheaf \[\Theta^3(\L) := \frac{p_{123}^* \L \otimes p_1^* \L \otimes p_2^* \L \otimes p_3^* \L}{p_{12}^* \L \otimes p_{23}^* \L \otimes p_{31}^* \L} = \bigotimes_{I \subseteq \{1, 2, 3\}} (p_I^* \L)^{(-1)^{|I| - 1}}\] on $A \times A \times A$ is trivial.  If $\L$ is rigid, then $\Theta^3(\L)$ is canonically trivialized by a section $s(A; \L)$. \qed
\end{corollary}
\noindent However, the proof of projectivity arising from this method rests on choosing an ample line bundle on $A$ and using some generating global sections to embed into $\P(\L^{\oplus n})$.  Mumford showed that a choice of ``$\theta$--structure'' on $A$, which is only slightly more data given in terms of Heisenberg representations\todo{I think.}, gives a canonical identification of $\P(\L^{\oplus n})$ with a \emph{fixed} projective space.  Separately, Breen\citeme{Breen} showed that if $\L$ is a line bundle on $A$ with a chosen trivialization of $\Theta^3 \L$ and $\pi\co A' \to A$ is an epimorphism that trivializes $\L$, then one can also associate to this a theory of $\theta$--functions.




\todo{I wish I knew how to motivate this.}\citeme{AS Prop 3.3, Lemma 3.7}An important piece of both of these stories is the $e_n$--pairing associated to such a bundle.  Given a section $u$ of $\Theta^3(\sheaf O_C)$, we define \[e_n(g, h) = \prod_{j=1}^{n-1} \frac{u(g, jg, h)}{u(g, jh, h)}.\]  It is immediate that $e_n(g, g) = 1$ and $e_n(g, h) = e_n(h, g)^{-1}$, and properties of $u$ further show that
\begin{align*}
e_n(g_1 +_{\G} g_2, h) & = e_n(g_1, h) e_n(g_2, h) \frac{u(ng_1, ng_2, h)}{u(g_1, g_2, nh)}, \\
e_n(g, h_1 +_{\G} h_2) & = e_n(g, h_1) e_n(g, h_2) \frac{u(ng, h_1, h_2)}{u(g, nh_1, nh_2)}.
\end{align*}
In particular, when $e_n$ is restricted to $C[n]$, this map becomes biexponential.  In the case of an elliptic curve, this lines up with more traditional definitions --- for instance, using the isogeny $n: C \to C$.  In the case of a \emph{complex} elliptic curve $\C / (1, \tau)$, this degenerates further to the assignment \[\left( \frac{a}{n}, \frac{b}{n} \tau \right) \mapsto \exp\left(-2\pi i \frac{ab}{n}\right).\]



\citeme{This part of the story is cribbed from AHS Section 2.6}
The other success of the $\theta$--functions story is that it gives access to a small piece of the compactified moduli of elliptic curves, $\overline{\moduli{ell}}$, called the Tate curve.  In this parametrized setting, define $D'$ to be the punctured complex disk, and set \[C'_{\mathrm{an}} = \C^\times \times D' / (u, q) \sim (qu, q).\]  The fiber of $C'$ over a particular point $q \in D'$ is the curve $\C^\times / q^{\Z}$ considered above, and $\theta$ determines a holomorphic function on the total space $\C^\times \times D'$.  The Weierstrass embeddings discussed above give an embedding of $C'_{\mathrm{an}}$ into $D' \times \CP^2$ described by \[wy^2 + wxy = x^3 - 5 \alpha_3 w^2 x + -\frac{5 \alpha_3 + 7 \alpha_5}{12}w^3\] for certain functions $\alpha_3$ and $\alpha_5$ of $q$.  At $q = 0$, these curve collapses to the twisted cubic \[wy^2 + wxy = x^3,\] and over the whole open unit disc $D$ we call this extended family $C_{\mathrm{an}}$.

Now let $A \subseteq \Z\ps{q}$ by the subring of power series which converge absolutely on the open unit disk.  It turns out that the coefficients of the Weierstrass cubic (i.e., $5\alpha_3$ and $\frac{1}{12}(5\alpha_3+7\alpha_5)$) lie in $A$, so it determines a generalized elliptic curve $C$ over $\Spec A$, and $C_{\mathrm{an}}$ is the curve given by base-change from $A$ to the ring of holomorphic functions on $D$.  The Tate curve $C_{\Tate}$ is defined to be the ``generalized elliptic curve'' over the intermediate object $D_{\Tate} = \Spec \Z\ps{q}$ as base-changed from $A$.  ``Generalized'' here means that the fiber over $q = 0$ is \emph{not} an elliptic curve --- but on its smooth locus it still carries the structure of a group scheme, and so one can still make sense of the associated formal group.

There are three natural coordinates available to us on $\widehat C_{\mathrm{an}}'$:
\begin{align*}
t & = x/y, &
\theta^{1/2,1/2}_q(u) & = \theta(t), &
1 - u & = 1 - u(t).
\end{align*}
Of these, only $t$ gives an algebraic coordinate on $C_{\mathrm{an}}'$ (and in fact on $C_{\mathrm{an}}$).  The others expand as power series in $t$ to
\begin{align*}
\theta(t) & = t + O(t^2), &
1 - u(t) & = t + O(t^2).
\end{align*}
The coefficients of the powers of $t$ in these series are holomorphic functions on the punctured disk $D'$.  In fact, they extend to $D$ and have integer coefficients\todo{... which you see by working over the completion of $\Z[u^pm]\ps{q}$ at $(1 - u)$}.  Thus $\theta(t)$ and $u(t)$ actually lie in $A\ps{t}$, so give functions on $\widehat C$ and $\widehat C_{\Tate}$.

Now, although $\theta^{1/2,1/2}_q$ does not give a section of $\sheaf I(0)$ on $C^3_{\mathrm{an}}$, it does descend to trivialize $\Theta^3 \sheaf I(0)$.  Then, since meromorphic sections of $\Theta^3 \sheaf I(0)$ on $C^3$ inject into such over $C_{an}^3$, the (transcendental) equation $s(C_{an}^3) = \delta^3 \theta^{1/2,1/2}_q(u)$ nonetheless determines the cubical structure on $\sheaf I(0)$ over $C$ and hence over $C_{\Tate}$ and $\widehat C_{\Tate}$ as well --- it can be expressed by $\delta^3 \theta(t)$.  It follows, finally, that the cubical structure extended over the compactification $D$ is also $\delta^3 \theta(t)$, since this function extends to $D$.\todo{Also, the invariant differential is $dx/(2y+x) = du/u$?}


% \begin{theorem}\citeme{Strip out the easy parts of Milne's Theorem 7.1}
% Every elliptic curve is projective.
% \end{theorem}
% \begin{proof}
% First, assume that the ground field $k$ is algebraically closed.  We aim to construct a very ample linear system, meaning that it separates points (for every pair $a, b$ of distinct closed points on the variety, there is a $D$ in the system with $a \in D$ but $b \not\in D$) and it separates tangent vectors (for every closed point $a$ and tangent vector $t$ at $a$, there is a $D$ in the system with $a \in D$ but $t \not\in T_a D$).  Consider just the condition that a linear system separates the origin point from all other closed points in $A$; the case of a curve is too low-dimensional for this is to be interesting, as we can pick our linear system to consist of only the point divisor $D = \{0\}$.  If $a$ and $b$ are any two points in $A(k)$, then one can always find a third point $c$ so that $D_a + D_c + D_{-c-a}$ misses $b$ --- that is, the duplicated and translated system of divisors separates $a$ from $b$.  The theorem of the cube shows that $D_a + D_c + D_{-c-a} = 3D$, regardless of $a$ and $c$, so we take $3D$ as the basic unit of our system of divisors.  Finally, looking around II.7.8.2 of Hartshorne\todo{Expand this out.} shows that this gives rise to the appropriate projective embedding.

% Finally, one can show that the projectivity of $A_{\bar k}$ entails the projectivity of $A_k$.\todo{Expand me?}
% \end{proof}

% Corollary 7.2 of Milne's chapter is that every abelian variety has a \emph{symmetric} ample invertible sheaf.  (Remarks after that say that any ample sheaf becomes very ample after cubing it. That's really what the theorem of the cube's use in the above proof is saying.)

% The theorem of the cube: $\Theta^3 \L$ is canonically trivial for any rigidified line bundle $\L$ on an abelian scheme $A$.  You can find this in Olsson Thm 2.2.  Or, Akhil Mathew has notes from an algebraic geometry class ( https://math.berkeley.edu/{\textasciitilde}amathew/232b.pdf ) where lectures 3--5 address the theorem of the cube.

% ------

% Take $S = \Spec \bar k$, let $E/S$ be an elliptic curve, and let $\L = \sheaf O_E(ne)$ for an integer $n \ge 1$.  Then for a point $a \in E(\bar k)$ \[t_a^* \L \otimes \L^{-1} \simeq \sheaf O_E(n(-a) -n(e))\] which is trivial if and only if $na = e$, hence $K_{(E, \L)} \cong E[n]$.  The $\Theta$--group $\sheaf G_{(E, \L)}$ consists of the set of points $(a, f)$ where $a \in E[n]$ and $f \in \bar k(E)$ is a function with $\div(f) = n(-a) - n(e)$.  There is a natural antisymmetric pairing, called the Weil pairing, coming from the central extension \[1 \to \Gm \to \sheaf G_{(E, \L)} \to E[n] \to 1.\]

% (See Olsson 3.5, 3.11.  You can also see Mumford's \textit{Abelian varieties} p.\ 222.)









\section{Unstable chromatic cooperations for $kU$}

\todo[inline]{We're moving to $C_*$ schemes, so we should include a description of the comparison map between $\SDiv_0$ and $C_2$.}

\begin{lemma}\label{EasyCompatibilityWithdn}\citeme{Lemma 4.5 of AS}
There is a unique $\lambda \in \Z/n$ such that the following triangle commutes:
\begin{center}
\begin{tikzcd}
(B\Z/p^j)^{\times 2} \arrow{d}{\lambda \beta \mu} \arrow{rd}{d_n(\L_1, \L_2)} \\
K(\Z, 3) \arrow{r}{i} & BU[6, \infty).
\end{tikzcd}
\end{center}
\end{lemma}
\begin{proof}
Recall the definition of $d_n(\L_1, \L_2)$:
\begin{align*}
d_n(\L_1, \L_2) & := \sum_{k=1}^{p^j-1} \left( (\L_1 - 1)(\L_1^k - 1)(\L_2 - 1) - (\L_1 - 1)(\L_2^k - 1)(\L_2 - 1)\right). \\
\intertext{Forgetting down to $BU$ and working with just virtual bundles rather than lifts of virtual bundles to elements of $kU^6$, this gives:}
& = (\L_1 - 1)(\L_2^{p^j} - 1) - (\L_1^{p^k} - 1)(\L_2 - 1).
\end{align*}
Since we have restricted to $(B\Z/p^j)^{\times 2}$, $\L_1^{p^k} = 1$ and $\L_2^{p^k} = 1$, so this formula collapses and the composite map \[(B\Z/p^j)^{\times 2} \xrightarrow{d_n} BU[6, \infty) \xrightarrow{\pi_2} BSU \xrightarrow{\pi_2} BU\] is null.  We now study the lifting problem across the two maps:
\begin{enumerate}
\item[$\pi_1$:] Since the long composite is null, it follows that the shorter composite $\pi_2 \circ d_n$ factors through the fiber of $\pi_2$.  But, the fiber of $\pi_2$ is $S^1$, and there are no maps \[\{(B\Z/p^j)^{\times 2} \to S^1\} \cong H^1((B\Z/p^j)^{\times 2}; \Z) = 0.\]  It follows that $\pi_2 \circ d_n$ is itself null.
\item[$\pi_2$:] Since $\pi_2 \circ d_n$ is null, $d_n$ must factor through the fiber of $\pi_2$, which is $K(\Z, 3)$.  With identical motives, one considers $H^3((B\Z/p^j)^{\times 2}; \Z)$, which is cyclic of order $n$ and generated by $\beta \circ \mu$. \qedhere
\end{enumerate}
\end{proof}

\begin{theorem}\citeme{AS Theorem 4.2, proved in Section 5}
The value $\lambda$ can be taken to be $\pm 1$ in \Cref{EasyCompatibilityWithdn}. \qed
\end{theorem}

\begin{corollary}\citeme{Corollary 4.4 of AS}
The following square commutes (up to sign):
\begin{center}
\begin{tikzcd}
BU[6, \infty)^E \arrow{r}{\gamma^E} \arrow{d}{\Pi_3} & K(\Z, 3)^E \arrow{d}{b_*} \\
C^3(\CP^\infty_E; \mathbb G_m) \arrow{r}{e} & \Weil(\CP^\infty_E).
\end{tikzcd}
\end{center}
\end{corollary}
\begin{proof}
We will check compatibility on $\Weil_n(\CP^\infty_E)$ for arbitrary $n$.  (Note: the sign can't bounce with $n$ because $\CP^\infty_E$ is $p$--divisible.)  Since $\Weil_n(\CP^\infty_E)$ is a subscheme of $\InternalHom{FormalSchemes}(\CP^\infty_E[n]^{\times 2}, \Gm)$, we can push forward to checking equality here, i.e., of the two maps \[\CP^\infty_E[n]^{\times 2} \times BU[6, \infty)^E \to \Gm.\]

The construction of adjoint elements from maps of spectra converts sums to products and is natural in the source spectrum.  Writing $z = \prod_{i=1}^3 (1 - \L_i) \in kU^6 (\CP^\infty)^{\times 3}$, it follows that the adjoint map $\hat z$ is given by the composite \[\hat z \co (\CP^\infty_E)^{\times 3} \times BU[6, \infty)^E \to (\CP^\infty_E)^{\times 3} \times C^3(\CP^\infty_E; \Gm) \xrightarrow{\operatorname{eval}} \Gm.\]  It follows by naturality that if $z = (1 - \L_1)(1 - \L_1^k)(1 - \L_2) \in kU^6 (\CP^\infty)^{\times 2}$, then $\hat z$ corresponds to the map \[\hat z \co (g_1, g_2, f) \mapsto f(g_1, kg_1, g_2),\] and continuing along these lines we see \[d_n(\L_1, \L_2)\hat{} \co (g_1, g_2, f) \mapsto \prod_{k=1}^{n-1} \frac{f(g_1, kg_1, g_2)}{f(g_1, kg_2, g_2)} = e_n(f)(g_0, g_1).\]

That described the bottom-left arm of the square.  For the other arm, take $w = \beta \circ \mu \in H^3(B\Z/p^j)^{\times 2}$ with adjoint $\hat w \co \CP^\infty_E[p^j]^{\times 2} \times K(\Z, 3)^E \to \Gm$.  Naturality shows $\hat w \circ \gamma^E = \widehat{\gamma_* w}$, and the theorem shows $\gamma_* w = \pm d_n(\L_1, \L_2)$, hence $\widehat{\gamma_* w} = (\pm d_n(\L_1, \L_2))\hat{}$, which is adjoint to $(e_n \Pi_3)^\pm$.
\end{proof}

\begin{lemma}[{\Cref{BSUtoBUtoCPinftyIsSexseq} + Cartier duality}]
There is a short exact sequence \[BSU^E \from BU^E \from (\CP^\infty)^E.\]
\end{lemma}

\todo[inline]{There's a nice construction of the $\Pi_k$ maps in the AHS preprint where the ``$C_k$'' constructions are performed directly on the spectra, then applying $(-)_E$ carries those constructions to relevant constructions on group schemes, and finally Cartier duality gives the maps $\Pi_k$ of the sort described above.  This is superior to saying ``adjoint'', in my opinion, though it could be remarked that these are equivalent.}

\begin{lemma}\citeme{Lemma 6.2 of AS}
The adjoint of the map $E_0 \CP^\infty \to E_0 BU$ induces a map $\Pi_1: BU^E \to C^1(\CP^\infty_E; \mathbb G_m)$ which is an isomorphism.  In fact, the Cartier duality isomorphism $(\CP^\infty)^E \cong \InternalHom{FormalGroups}(\CP^\infty_E, \mathbb G_m)$ fits into a commuting square
\begin{center}
\begin{tikzcd}
(\CP^\infty)^E \arrow{r} \arrow{d} & \InternalHom{FormalGroups}(\CP^\infty_E, \Gm) \arrow{d}{\begin{array}{c} \text{natural} \\ \text{inclusion} \end{array}} \\
BU^E \arrow{r}{\Pi_1} & C^1(\CP^\infty_E; \Gm).
\end{tikzcd}
\end{center}
\end{lemma}
\begin{proof}
\todo{Include this.}
\end{proof}

\begin{theorem}\label{BUBSUandC1C2Commute}
The following square commutes:
\begin{center}
\begin{tikzcd}
BU^E \arrow{d}{\Pi_1} \arrow{r} & BSU^E \arrow{d}{\Pi_2} \\
C^1(\CP^\infty_E; \Gm) \arrow{r}{\delta} & C^2(\CP^\infty_E; \Gm).
\end{tikzcd}
\end{center}
\end{theorem}
\begin{proof}
\citeme{Lemma 6.4 of AS}
This is a matter of expanding definitions and using \[(\L_1 - 1)(\L_2 - 1) = \mu^*(\L - 1) - \pi_1^*(\L - 1) - \pi_2(\L - 1).\]
\end{proof}

\begin{theorem}
The following is a map of short exact sequences:
\begin{center}
\begin{tikzcd}
0 \arrow{r} & (\CP^\infty)^E \arrow{d} \arrow{r} & BU^E \arrow{r} \arrow{d} & BSU^E \arrow{r} \arrow{d} & 0 \\
0 \arrow{r} & \InternalHom{FormalGroups}(\CP^\infty_E; \Gm) \arrow{r} & C^1(\CP^\infty; \Gm) \arrow{r} & C^2(\CP^\infty; \Gm) \arrow{r} & 0.
\end{tikzcd}
\end{center}
\end{theorem}
\begin{proof}
\citeme{Prop 6.5 of AS. They reference a superior alternate argument in the AHS preprint though\ldots}
\end{proof}

\begin{lemma}
The same holds as in \Cref{BUBSUandC1C2Commute} with $BSU$, $BU[6, \infty)$, $C^2$, and $C^3$.
\end{lemma}
\begin{proof}
\citeme{AS Lemma 7.1}
\end{proof}

\begin{lemma}
The map $C^2 \to C^3$ is injective for $\CP^\infty_E$ a $p$--divisible group.
\end{lemma}
\begin{proof}
The kernel of this map consists of maps alternating, biexponential maps $(\CP^\infty_E)^{\times 2} \to \Gm$.  We can restrict such a map to get a map \[f \co \CP^\infty_E[p^j] \times \CP^\infty_E \to \Gm,\] where we can calculate \[f(x, p^j y) = f(p^j x, y) = f(0, y) = 1.\]  But since $p^j$ is surjective on $\CP^\infty_E$, every point on the right-hand side can be so written, so at every left-hand stage the map is trivial.  Finally, $\CP^\infty_E = \colim_j \CP^\infty_E[p^j]$, so this filtration is exhaustive and we conclude that the kernel is trivial.\citeme{Lemma 7.2 of AS}
\end{proof}

\begin{lemma}
In fact, the following sequence is exact\todo{This is not a typo, we don't get right-exactness yet.} \[0 \to C^2(\CP^\infty_E; \Gm) \xrightarrow\delta C^3(\CP^\infty_E; \Gm) \to \Weil(\CP^\infty_E).\]
\end{lemma}
\begin{proof}\citeme{Lemma 7.3 of AS}
This is hard work.  Breen's idea is to show that picking a preimage under $\delta$ is the same as picking a trivialization of the underlying symmetric biextension of the cubical structure.  Then (following Mumford), one shows that the underlying symmetric biextension is trivial exactly if the Weil pairing is trivial.
\end{proof}

These together culminate in a map of exact sequences with marked isomorphisms:
\begin{center}
\begin{tikzcd}
0 \arrow{r} & BSU^E \arrow{r} \arrow{d}{\simeq} & BU[6, \infty)^E \arrow{r} \arrow{d} & K(\Z, 3)^E \arrow{d}{\simeq} \arrow{r} & 0 \\
0 \arrow{r} & C^2(\CP^\infty_E; \Gm) \arrow{r} & C^3(\CP^\infty_E; \Gm) \arrow{r} & \Weil(\CP^\infty_E).
\end{tikzcd}
\end{center}

\begin{corollary}
The map \[BU[6, \infty)^E \to C^3(\CP^\infty_E; \Gm)\] is an isomorphism.  Also, the map \[C^3(\CP^\infty_E; \Gm) \to \Weil(\CP^\infty_E)\] is a surjection. \qed
\end{corollary}

-------------

Moving from $BU$ to $MU$: $MU\<6\>^E$ is a $\G_m$--torsor over $BU\<6\>^E$, so if you can produce another torsor and any map between them, that automatically gives you an isomorphism and hence a description.  This is pretty easy to read about in section 2.4 of the AHS preprint.  The big theorem is Theorem 2.42 in Section 2.4.4.







\section{Unstable additive cooperations for $kU$}

Example 10.4 in Goerss's Hopf Rings paper does the case of periodic $K$--theory.

You can do the case $H\F_2^* BU[6, \infty)$ by hand, using the Serre spectral sequence (and ``Wu formulas'' for the action of the Steenrod algebra on the Chern classes --- which you can probably read off instead from the divisorial description).

The analogues of Wu formulas in mod--$p$ cohomology are due to Shay, in \textit{mod--$p$ Wu formulas for the Steenrod algebra and the Dyer--Lashof algebra}.

There are also versions of the calculation due to Stong and to Singer, which deserve mention.

At the end of the day, you want to be able to write down the Poincar\'e series for each of the prime fields and $BU[6, \infty)$.






\section{Elliptic spectra}

\begin{definition}
An \textit{elliptic spectrum} consists of:
\begin{enumerate}
\item An even-periodic ring spectrum $E$.
\item A (generalized) elliptic curve $C$ over $S_E$.\todo{I'm not sure if this is worth explaining. I guess we just mean elliptic curves with certain singularities allowed far away from the origin. Maybe it is worth explaining: you don't get examples like $H\Z P$ or $K^{\Tate}$ without allowing degeneracies.}
\item An isomorphism $\phi: C^\wedge_0 \cong \CP^\infty_E$.
\end{enumerate}
A \textit{map of elliptic spectra} consists of
\begin{enumerate}
\item A map of ring spectra $f: E \to E'$.
\item An \emph{isomorphism} of elliptic curves $f^* C \to C'$.
\end{enumerate}
\end{definition}

\todo{In particular, isogenies of elliptic curves are \emph{not} allowed. This is the realm of power operations.}

\begin{example}
Cohomology with complex coefficients and a selected lattice in the plane: $HP_\Lambda$.  The required isomorphism of formal groups comes from the logarithm map inverse to formally expanding $\C \to \C/\Lambda$ at the origin.
\end{example}

\begin{example}
Integral cohomology with the curve $zy^2 = x^3$.
\end{example}

\begin{example}
Ordinary $K$--theory with the curve $zy^2 + zxy = x^3$.
\end{example}

\begin{example}
Tate $K$--theory
\end{example}

\todo[inline]{I want to sketch the reduction for even-periodic elliptic cohomology theories to the case of $MUP$, then from there to $HkP$ for the prime fields $K$, then from there to questions about additive cocycles.  We certainly don't need to recall any of these calculations, but I think it's a nice example of the philosophy that the additive formal group is such a knotted point of $\moduli{fg}$ that it suffices to check something there to learn it for the rest of the stack.  This survives in the published form of AHS, but it's stated pretty clearly as Prop 3.4 in the unpublished verison.  See also 5.12 of the unpublished version.}

\todo[inline]{From the intro to the AHS preprint: For any lattice $\Lambda \subseteq \C$, we get a map $\Phi: MU[6, \infty) \to HP_\Lambda$ which sends $(2n)$--dimensional bordism classes $M$ to numbers $\Phi(M; \Lambda) \cdot u_{\Lambda}^n$. Suppose $\Lambda$ and $\Lambda'$ are two lattices with $\lambda \cdot \Lambda = \Lambda'$. This induces a map $HP_{\Lambda'} \to HP_\Lambda$ which intertwines the maps $\Phi$ by $\Phi(M; \lambda \cdot \Lambda) = \lambda^{-n} \Phi(M; \Lambda)$.  The usual appearance of a modular form (via $SL_2$ invariance) can be extracted from the top of page 5, if you want.  You can also show that this ``functional equation for a modular form'' is actually realized by a function by considering the elliptic cohomology theory built out of the bundle of elliptic curves $\mathfrak h \times \C/(1, \tau) \to \mathfrak h$ and the ordinary coefficient ring $\sheaf O[u^\pm]$, $\sheaf O$ the ring of holomorphic functions on $\mathfrak h$.}

-----------

Define the classical $\theta$--function on the Tate curve by \[\tilde \theta_q(u) = (1 - u) \prod_{n > 0} (1 - q^n u)(1 - q^nu^{-1}) \in \Z[u^{\pm}]\llbracket q \rrbracket.\]  Write $t = 1-u$ for the usual coordinate on the formal multiplicative group; then we can think of $\tilde \theta_q(u)$ as an element of $\Z\llbracket q\rrbracket\llbracket t\rrbracket$ and thus as a function on $\G_m \times D_{\Tate}$, $D_{\Tate} = \Spec \Z\llbracket t \rrbracket$ the Tate domain.  In fact, $\tilde\theta_q(u)$ is even a coordinate on this formal group over $D_{\Tate}$, which one can identify with $\widehat C_{\Tate}$.

By formal rearrangements one can produce the familiar functional equations
\begin{align*}
\tilde \theta_q(qu) & = -u^{-1} \tilde\theta_q(u), \\
\tilde \theta_q(q^k u) & = q^{-k(k-1)/2} (-u)^{-k} \tilde\theta_q(u).
\end{align*}
\todo{This is actually kind of hard to do algebraically. It's discussed in Appendix A of the AHS preprint.}
Regarding $\tilde\theta$ as an element of $C^0(\widehat C_{\Tate}; \L)$, this gives a cubical structure \[\delta^3(\tilde\theta) \in C^3(\widehat C_{\Tate}; \L),\] and one computes $\delta(\tilde\theta) = dt/\theta$ for \[\theta_q(u) = (1 - u) \prod_{n > 0} \frac{(1 - q^nu)(1 - q^n u^{-1})}{(1 - q^n)^2} \in \Z[u^\pm]\llbracket q \rrbracket,\] so you can also apply $\delta^2$ to this expression.\todo{Why would someone find this more familiar?  Also, in what sense is $\delta$ anything like differentiation?}  The functional equation has something to say about this cubical structure: \[(\delta^3 \tilde\theta_q)(u, v, w) = \begin{cases} (\delta^3 \tilde\theta_q)(qu, v, w), \\ (\delta^3 \tilde\theta_q)(u, qv, w), \\ (\delta^3 \tilde\theta_q)(u, v, qw). \end{cases}\]\todo{Mike has a nice remark about this: the exponents in the iterated functional equation for $\tilde\theta_q$ are quadratic in $k$ and so killed by $\delta^3$, which is another differentiation-type claim.}

\begin{theorem}\citeme{AHS preprint Prop 2.49}
The cubical structure $\delta^3(\tilde \theta)$ is the restriction of $s(C_{\Tate} / D_{\Tate})$ to $\widehat C_{\Tate}$.
\end{theorem}
\begin{proof}
The ratio $s(\widehat C_{\Tate} / D_{\Tate}) / \delta^3(\tilde\theta)$ is a power series $g \in \Z\llbracket q, t_0, t_1, t_2\rrbracket$ and we need to show that $g = 1$.  This will hold if we can show that there is a neighborhood of $0$ in $\C^4$ on which $g$ converges to $1$, so we can employ complex analytic techniques.  Fix $q \in \C$ with $0 < |q| < 1$ and let $C_q$ be the $\C$--analytic elliptic curve fibering over this point in $D_{\Tate} \times \Spec \C$.  The product expansion of $\tilde\theta_q(u)$ converges locally uniformly to an analytic function on $\C^\times$ vanishing only on $q^{\Z}$ and there only to first order.  It should suffice to show that $s(C_q/\C) = \delta^3(\tilde\theta_q)$ as analytic functions on $(\C^\times)^{\times 3}$.  The consequence for $\delta^3 \tilde\theta$ of the functional equation for $\tilde\theta$ recalled above shows that $\delta^3\tilde\theta_q$ descends to give a meromorphic $1$--form $\phi$ on $C_q^{\times 3}$.  Then, because $\tilde\theta_q$ has only simple poles on $q^{\Z}$ and none elsewhere, we deduce that $\phi$ is actually a cubical structure, and unicity then finally forces $\phi = s(C_q / \C)$.
\end{proof}

\todo[inline]{See also the bottom of page 22 in the AHS preprint for the relevant theory of integration (esp. Prop 2.54), and see Proposition 2.56 for a comparison theorem between the integration theory and $\sigma_{\Tate}$.}

\begin{definition}
Let $\gamma$ denote the element $K_{\Tate}(\Z \times BU)$ determined by the vector bundle operation \[\gamma: -V \mapsto \prod_{n > 0} \sum_{k \ge 0} q^{nk} \Sym^k(V),\] and let $\bar\gamma$ denote its complex conjuguate.  Since $K_{\Tate}^0(MUP)$ is a module over $K_{\Tate}(\Z \times BU)$, we can define an element \[\sigma_{\Tate} := \gamma \cdot \bar\gamma \cdot \alpha,\] where $\alpha$ is the usual orientation $MP \to KU$ corresponding to the coordinate $1 - t$ on the formal group $\G_m$.
\end{definition}

\todo{``Modularity'' of the $K_{\Tate}$ orientation?}


------------

Criteria for the existence of symmetric cocycle schemes.

AHS: they exist
\todo[inline]{The technical condition guaranteeing the existence of symmetric power schemes is that the symmetric cocycle schemes are coalgebraic formal schemes, since then we have an involutive Cartier duality functor.  This comparison essentially comes out of saying that $C_k$ can be defined by a strong colimit, so if we can check that this strong colimit exists\ldots (cf. Prop 3.3 of the AHS preprint).}

------------

Here's what the published version of AHS has to say about the $\sigma$--orientation of $K_{\Tate}$.  (See Section 2.7.)

$K_{\Tate}$ has multiplicative cohomology theory $K\ps{q}$, formal group multiplicative (as induced by $K$--theory), and isomorphism to $\widehat C_{\Tate}$ given by $1 - u(t)$ as in Lecture 5.2.  Since the cubical structure on $\widehat C_{\Tate}$ is given as $\delta^3$ of something, it follows that the $\sigma$--orientation factors as
\begin{center}
\begin{tikzcd}
MU[6, \infty) \arrow{d} \arrow{drr} \\
MU \arrow{r} & MUP \arrow{r}{\tilde \theta} & K\ps{q}.
\end{tikzcd}
\end{center}
Our goal is to express in terms of characteristic classes the induced map on homotopy by the horizontal composite.

To start with, the topological restriction $MU \to MUP \to E$ sends the coordinate $f$ on $\G_E$ to the rigid section $\delta f$ of $\Theta^1(\sheaf I(0)) = \sheaf I(0)_{0} \otimes \sheaf I(0)^{-1}$.  The most straightforward formula for $\delta f$ is $f(0) / f$, which is confusing, since $f(0) = 0$ usually but not as a section of $\sheaf I(0)_0$.  It's probably clearer to express $\delta f$ in terms of the isomorphism \[\sheaf I(0)_0 \otimes \sheaf I(0)^{-1} \cong \omega \otimes \sheaf I(0)^{-1},\] where $\delta f$ is given by the formula \[\delta f = \frac{f'(0) Dx}{f(x)},\] $Dx$ the invariant differential with value $dx$ at $0$.

In the example of $K$--theory, the usual complex Atiyah--Bott--Shapiro map $MP \to K$ corresponds to the coordinate $1 - u$ on the formal completion of $\Gm = \Spec \Z[u^\pm]$.  The invariant differential is $D(1 - u) = -du/u$, and the restriction to $MU$ classifies the $\Theta^1$--struucture \[\delta(1 - u) = \frac{1}{1 - u}\left(-\frac{du}{u}\right).\]  In the more complex example of Tate $K$--theory, the map \[MU \to MUP \xrightarrow{\tilde\theta} K_{\Tate}\] factors by the coordinate change map \[MU \to MU \sm MU \simeq MU \sm BU_+ \xrightarrow{\delta(1 - u) \sm \theta'} K_{\Tate},\] where $\theta'$ is the element of $BU^{K_{\Tate}} \cong C^1(\widehat C_{\Tate}; \Gm)$ given by the formula \[\theta' = \prod_{n \ge 1} \frac{(1 - q^n)^2}{(1 - q^n u)(1 - q^n u^{-1})}.\]

\needproof{The Todd genus}
In geometric terms, the homotopy groups $\pi_* MU \sm BU_+$ are the bordism groups of pairs $(M, V)$ consisting of a stably almost complex manifold $M$ and a virtual complex vector bundle $V$ over $M$ of virtual dimension $0$.  The map \[\pi_* MU \to \pi_* (MU \sm BU_+)\] sends a manifold $M$ to the pair $(M, \nu)$ where $\nu$ is the reduced stable normal bundle.  Next, the map $\pi_* \delta(1 - u)$ sends a manifold $M$ of dimension $2n$ to $p_!(1) \in K^{-2n}(*)$ where $p: M \to *$ is the unique map.  One has \[p_!(1) = \operatorname{Td}(M) \left( -\frac{du}{u} \right)^n,\] where $\operatorname{Td}(M)$ is the Todd genus of $M$ (and it is customary to suppress the grading and write $p_!(1) = \operatorname{Td}(M)$).  The map $\theta'$ is where the real work is: it is the stable exponential characteristic class taking the value \[\prod_{n \ge 1} \frac{(1 - q^n)^2}{(1 - q^n \L)(1 - q^n \L^{-1})}\] on the reduced class $(1 - \L)$ of a line bundle $\L$.  This stable exponential characteristic class can easily be identified with \[V \mapsto \bigotimes_{n \ge 1} \Sym_{q^n}(- \bar V_{\C}),\] where $V_{\C} = V \otimes_{\R} \C$, $\bar V_{\C} = V_{\C} - \eps^{\oplus \dim V}$, and $\Sym_t(W)$ is defined for (complex) vector bundles $W$ by \[\Sym_t(W) = \bigoplus_{m \ge 0} \Sym^m(V) t^m \in K(M)\ps{t}\] and extended to virtual bundles using the exponential rule $\Sym_t(W_1 - W_2) = \Sym_t(W_1) / \Sym_t(W_2)$.  Altogether, the effect of the $\sigma$--orientation therefore sends an almost complex manifold $M$ of dimension $2n$ to
\begin{align*}
\pi_* \sigma_{K_{\Tate}}(M) & = f_! \left( \bigotimes_{n \ge 1} \Sym_{q^n}(\bar T_{\C}) \right) \\
& = \operatorname{Td}\left(M; \bigotimes_{n \ge 1} \Sym_{q^n}(\bar T_{\C}) \right) \left( -\frac{du}{u} \right)^n \\
& \in \widetilde K\ps{q}^0(S^{2n}).
\end{align*}

This is basically Witten's formula for his genus.  There is a small caveat: Witten's genus is defined for Spin manifolds.  With some care, perhaps we could construct a homotopical square\todo{Is this possible?  Or do we really need the $\String$--orientation to make this happen?  Is this exactly the topic for the next day?!!}
\begin{center}
\begin{tikzcd}
MSU \arrow{r} \arrow{d} & MU \arrow{d} \\
M\Spin \arrow[densely dotted]{r} & K_{\Tate},
\end{tikzcd}
\end{center}
but for the moment we content ourselves with a square of homotopy groups
\begin{center}
\begin{tikzcd}
\pi_* MSU \arrow{r} \arrow{d} & \pi_* MU \arrow{d} \\
\pi_* M\Spin \arrow[densely dotted]{r} & \pi_* K_{\Tate}.
\end{tikzcd}
\end{center}
Let $M$ be a $\Spin$--manifold of dimension $2n$, and use the splitting principle to write \[TM = \L_1 \oplus \cdots \oplus \L_n\] for complex line bundles $\L_i$.  The $\Spin$--structure gives a square root of $\prod \L_i$, which is equivalent to picking a square root for each $\L_i$.\todo{Is it?}  Since, for each $i$, the $O(2)$--bundles underlying $\L_i^{1/2}$ and $\L_i^{-1/2}$ are isomorphic, we can write \[TM = \sum \L_i + \L_i^{-1/2} - \L_i^{1/2},\] which is now a sum of $SU$--bundles.  Using this, one easily checks that the $\sigma$--orientation of $M$ gives \[\widehat A \left(M; \bigotimes_{n \ge 1} \Sym_{q^n}(\bar T_{\C}) \right) \left(-\frac{du}{u}\right)^n,\] where the $\widehat A$--genus is the push-forward in $KO$--theory associated to the unique orientation $M\Spin \to KO$ fitting into the commutative diagram
\begin{center}
\begin{tikzcd}
MSU \arrow{r} \arrow{d} & MU \arrow{d} \\
M\Spin \arrow{rd} \arrow{r} & K \\
& KO \arrow{u}{\widehat A}.
\end{tikzcd}
\end{center}

Finally, we want to see that for $[M] \in \pi_{2n} MU[6, \infty)$, \[\Phi(M) := (\pi_{2n} \sigma_{K_{\Tate}})(M) \left( -\frac{du}{u} \right)^{-n} \in \pi_0 K_{\Tate} = Z\ps{q}\] is a modular form.  We've at least seen that $\Phi(M)$ is a holomorphic function on $D$, with integral $q$--expansion coefficients.  It suffices to show that if $\pi: \h \to D$ is the map $\pi(\tau) = e^{2 \pi i \tau}$, then $\pi^* \Phi(M)$ transforms correctly under the action of $SL_2(\Z)$.  This is supposed to follow from stuff in the introduction (pp.\ 600-1, also Example 2.3).


\todo[inline]{Now that you have an extra few days, you could actually go through the calculation of $H\F_{p*} \OS{ku}{2k}$ and $C^k(\G_a; \Gm)$.}






\section{$\Spin$ and $\String$ orientations}

Formal schemes for certain real $K$--theory spaces

The Atiyah--Bott--Shapiro orientation and the fibration $BSU \to BSpin$ \citeme{Theorem 2.3.5.iv in KLW, the last fibration in 2.3.2 at $k=-2$, and sections 5.3 and 5.13}

The $\String$ orientation and $\Sigma$--structures




The following definition is meant to curtail the behavior of Hopf algebras at the prime $p = 2$, where the erasure of signs can cause strange things to happen.  In particular, Hopf algebras satisfying the following condition split as the tensor product of an exterior odd--dimensional part and a commutative even--dimensional part.

\begin{definition}\citeme{KLW Remark 4.5}
A Hopf algebra is said to be \textit{Restriction A} when it is formed from the following components:
\begin{itemize}
\item A polynomial algebra on a single even--degree generator.
\item A truncated polynomial algebra on a single even--degree generator.
\item An algebra on a single even generator $x$ where a large power $x^{p^d}$ can be rewritten in terms of lower powers.
\item An exterior algebra on a single odd--degree generator.
\item A ``divisible algebra'' $P_\infty = k\<a_0, a_1, \ldots, a_n, \ldots\>$ where $a_n^p = a_{n-1}$ and $a_0^p = 0$. \todo{Pepper these with topological examples.}
\end{itemize}
\end{definition}

\begin{theorem}\citeme{Theorem 4.2 of KLW}
Let $F \xrightarrow i E \xrightarrow r B$ be a fibration of connected double loopspaces, let $K$ have K\"unneth isomorphisms, and let $K_* F$ be a bicommutative Hopf algebra.  Then there is a spectral sequence of Hopf algebras \[E^2_{*, *} = \Tor^{K_* F}_{*, *}(K_* E, K_*) = \Tor^{\ker i_*}_{*, *}(K_*, K_*) \otimes \coker i_* \Rightarrow K_* B,\] where $\coker i_* = \Tor^{K_* F}_{0, *}(K_* E, K_*)$.  If $i_*$ is injective, this gives a short exact sequence of Hopf algebras \[K_* \to K_* F \to K_* E \to K_* B \to K_*.\]  If $\ker i_*$ is Restriction A and $\coker i_*$ is even, then $\coker i_*$ injects into $K_* B$ and all differentials take place in $\Tor^{K_* F}_{*, *}(K_* K_*)$.  If \[\Loops F \to \Loops E \to \Loops B\] induces a short exact sequence of Hopf algebras on $K$--homology with $K_* \Loops B$ Restriction A and $K_* F$ even, then the original fiber sequence also induces a short exact sequence of Hopf algebras. \qed
\end{theorem}

\begin{corollary}\citeme{Theorem 4.4 of KLW}
Let $E \to B' \to B$ be connective coverings of a simply connected double loopspace $B$.  Consider the following diagram of fiber sequences:
\begin{center}
\begin{tikzcd}
F' \arrow{d} \arrow{r} & E \arrow[-,double]{d} \arrow{r} & B' \arrow{d} \\
F \arrow{r} & E \arrow{r} & B.
\end{tikzcd}
\end{center}
If the bottom row induces a short exact sequence of Hopf algebras on $K$--homology, then so does the top row. \qed \todo{Maybe you could prove this. It relies on the finite Postnikov systems result which you may have talked about back in the Dieudonne modules sections.}
\end{corollary}

\todo{We mostly need to see that Postnikov sections induce short exact sequences.  For that, note that the $K$--theory of EM spaces is always Restriction A.}

\todo{Include a remark about Section 5.2 on ``Bicommutativity''.}







\begin{theorem}
There is a bi-Cartesian square
\begin{center}
\begin{tikzcd}
& \Div_0 \overline{\G}[2] \arrow{rr} \arrow{ld} & & \Div_0 \G[2] \arrow{ld} \\
\Div_0 \overline{\G} \arrow{rr} & & BO_K.
\end{tikzcd}
\end{center}
\end{theorem}
\begin{proof}
\todo{Analyze the Atiyah--Hirzebruch spectral sequence}
\end{proof}

\begin{lemma}
Consider the cube constructed by taking pointwise fibers of the composite to $X$.
\begin{center}
\begin{tikzcd}
& A' \arrow{rr} \arrow{ld} \arrow{dd} & & B' \arrow{ld} \arrow{dd} \\
C' \arrow{rr} \arrow{dd} & & D' \arrow[crossing over]{dd} \\
& A \arrow{rr} \arrow{ld} & & B \arrow{ld} \\
C \arrow{rr} & & D \arrow{rr} & & X.
\end{tikzcd}
\end{center}
If the bottom face is bi-Cartesian, then so is the top.
\end{lemma}
\begin{proof}
\todo{Prove this? It's valid in an arbitrary abelian category.}
\end{proof}

\begin{corollary}
There is a bi-Cartesian square
\begin{center}
\begin{tikzcd}
& \Div_0 \overline{\G}[2] \arrow{rr} \arrow{ld} & & \SDiv_0 \G[2] \arrow{ld} \\
\Div_0 \overline{\G} \arrow{rr} & & BSO_K. & & \qed
\end{tikzcd}
\end{center}
\end{corollary}
\begin{proof}
\todo{Write this for real.}
The fibration $BSO \to BO \to BO(1)$ gives a short exact sequence of Hopf algebras.  The composite $\Div \overline{\G} \to \G[2]$ acts by zero and the composite $\Div \G[2] \to \G[2]$ acts by summation.  The summation one you can probably prove by comparing with the determinant (or Postnikov) section for $BU$.
\end{proof}

\begin{corollary}
\todo{Write this for real. Even the statement is bad: see ``$\ker \omega$''.}
There is a bi-Cartesian square
\begin{center}
\begin{tikzcd}
& \Div_0 \overline{\G}[2] \arrow{rr} \arrow{ld} & & \ker \omega \arrow{ld} \\
\Div_0 \overline{\G} \arrow{rr} & & B\Spin_K. & & \qed
\end{tikzcd}
\end{center}
\end{corollary}
\begin{proof}
This goes similarly to the one above.  You can compute that the composite $\Div \overline{\G} \to \G[2]^{\wedge 2}$ is zero using an identical technique.  To compute the action on the other factor, KLW show that there's a diagram of exact sequences
\begin{center}
\begin{tikzcd}
& & & K_* \arrow{d} \\
& & & K_* K(\Z, 3) \arrow{d} \\
K_* \arrow{r} & K_* B\Spin \arrow{r} \arrow{d} & K_* BSU \arrow{r}{\tau} \arrow[-,double]{d} & K_* BU[6, \infty) \arrow{d}{\delta} \\
K_* \arrow{r} & K_* BSO \arrow{r}{i} & K_* BSU \arrow{r}{1 - \xi} & K_* BSU \arrow{d} \\
& & & K_* .
\end{tikzcd}
\end{center}
Since $(1 - \xi) \circ i = 0$, we have that $\delta \circ \tau \circ i = 0$ and hence that $\tau \circ i$ lifts to $K_* K(\Z, 3)$.  Identifying $\SDiv_0 \G[2]$ with $C_2 \G[2]$, we check that the composites \[C_2 \G[2] \xrightarrow{\omega} \G[2]^{\wedge 2} \xrightarrow{\eps} C_3 \G\] and \[C_2 \G[2] \to C_2 \G \xrightarrow{\tau} C_3 \G\] agree.  For a point $[a, b] \in C_2 \G$, this is the claim
\begin{align*}
0 & = \eps(a \wedge b) - \tau[a, b] \\
& = [a, a, b] - [b, a, b] - [-a - b, a, b] \\
& = [a, a, b] - [b + a, a, b] + [b, a + a, b] - [b, a, b],
\end{align*}
and this is the expression called $R(b, a, a, b)$, which is forced null in $C_3 \G$.

\todo[inline]{I'm a little fuzzy on the coherence of this with the Bockstein: this computes the lift of $\tau \circ f$ into $K(\Z, 3)_K$, and it does happen to factor through the subscheme $K(\Z/2, 2)_K$ determined by the Bockstein. However, I don't immediately see why this agrees with the bottom Postnikov section of $BSO$: that's a map off of $BSO$ and this is a rotated map into $BU[6, \infty)$, so it's not an immediate consequence of naturality.}
\end{proof}



\todo{What follows is the analysis for $M\String$.  Is the one for $M\Spin$ analogous and do-able?  Does it involve $CK_2$ and maybe a clever choice of $MSU$--orientation?}


The sequence $\Spin/SU \to BU[6, \infty) \to B\String$ is exact and right-exact.  The kernel of the map $\Spin/SU \to BU[6, \infty)$ is ``$CK_3$'' \citeme{Theorem 2.3.5.vi of KLW}, where \[CK_j = \bigoplus_{k=j}^\infty K_* K(\Z/2, k).\]  More than that, KLW even say where the polynomial and nonpolynomial parts of $K_* \Spin/SU$ land inside of $K_* BU[6, \infty)$.  I think\todo{But I have not checked!} that this means that $K_* BU[6, \infty)$ is a flat $K_* \Spin/SU$--module at heights $d \le 2$.

Anyway, there's always a $\Tor$--spectral sequence owing to the pushout diagram
\begin{center}
\begin{tikzcd}
\Susp^\infty_+ \Spin/SU \arrow{r} \arrow{d} & MU[6, \infty) \arrow{d} \\
* \arrow{r} & M\String
\end{tikzcd}
\end{center}
of signature \[\Tor^{K_* \Spin/SU}_{*, *}(K_* MU[6, \infty), K_*) \Rightarrow K_* M\String.\]  So, under the flatness hypothesis above, there are no higher $\Tor$ terms so the spectral sequence collapses to give \[K_* M\String \cong K_* MU[6, \infty) \mmod K_* \Spin/SU.\]  So, what remains to be shown is that $K_* \Spin/SU$ picks out the correct extra relation for $\Sigma$--structures.  Then, we need a density argument to show that this handles all of the at-a-point cases of elliptic cohomology.







-------

Some other things that might belong in this chapter:

The cubical structure on a singular (generalized) elliptic curve is not unique, but (published) AHS has an argument showing that the unicity of the choice on the nonsingular ``bulk'' extends to a unique choice on the ``boundary'' of the compactified moduli too.


There's also the work of Ando--French--Ganter on factorized / iterated $\Theta$ structures and how they give rise to the ``two--variable Jacobi genus''.












