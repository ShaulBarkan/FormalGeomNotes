% -*- root: main.tex -*-

\chapter{Loose ends}

---

\todo[inline]{I'd like to spend a couple of days talking about ways the picture in this class can be extended, finally, some actually unanswered questions that naturally arise.  The following two section titles are totally made up and probably won't last.}

\section{$E_\infty$ geometry}

\section{Knowns and unknowns}




\section{Erase me}

\section{Erase me}

\section{Erase me}


-----




\oldsection*{The modularity of the $M\String$ orientation}

$E_\infty$ orientations by $M\String$

$\tmf$, $\TMF$, and $\Tmf$ in terms of $\moduli{ell}$

Thom spectra and $\infty$--categories

The Bousfield--Kuhn functor and the Rezk logarithm

% free $E_\infty$--orientations off of $MU$

\oldsection*{Higher orientations}

$\TAF$ and friends

The $\alpha_{1/1}$ argument: Prop 2.3.2 of Hovey's $v_n$--elements of ring spectra

% The HLP calculations

\oldsection*{Equivariance}

This is tied up with the theory of power operations in a way I've never really thought about.  Seems complicated.

\oldsection*{Index theorems}

Connections with analysis

The Stolz--Teichner program

\oldsection*{Contexts for structured ring spectra}






-----

Difficulty in computing $\S_d \actson E_d^*$. (Gross--Hopkins and the period map.)

Barry's $p$--adic measures

Fixed point spectra and e.g. $L_{K(2)} \tmf$.

Blueshift, A--M--S, and the relationship to A--F--G?

Does $E_n$ receive an $E_\infty$ orientation?  Does $BP$?

Remark 12.13 of published $H_\infty$ AHS says their obstruction framework agrees with the $E_\infty$ obstruction framework (if you take everything in sight to have $E_\infty$ structures).  This is almost certainly related to the discussion at the end of Matt's thesis about the $MU$--orientation of $E_d$.\todo{Section 12.4 compares doing $H_\infty$ descent with doing $E_\infty$ descent and shows that they're the same (in the case of interest?).}

Hovey's paper on $v_n$--periodic elements in ring spectra.  He has a nice (and thorough!) exposition on why one should be interested in bordism spectra and their splittings: for instance, a careful analysis of $M\Spin$ will inexorably lead one toward studying $KO$.  It would be nice if studying $M\String$ (and potentially higher analogues) would lead one toward non-completed, non-connective versions of $EO_n$.  Talk about $BoP$, for instance.
