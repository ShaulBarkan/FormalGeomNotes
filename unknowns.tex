% -*- root: main.tex -*-

\chapter{Loose ends}

---

\todo[inline]{I'd like to spend a couple of days talking about ways the picture in this class can be extended, finally, some actually unanswered questions that naturally arise.  The following two section titles are totally made up and probably won't last.}

\oldsection{$E_\infty$ geometry}



\subsection*{The modularity of the $M\String$ orientation}

$E_\infty$ orientations by $M\String$

$\tmf$, $\TMF$, and $\Tmf$ in terms of $\moduli{ell}$

Thom spectra and $\infty$--categories

The Bousfield--Kuhn functor and the Rezk logarithm

% free $E_\infty$--orientations off of $MU$



\oldsection{Rational phenomena: character theory for Lubin--Tate spectra}

There's a sufficient amount of reliance on character theory in Matt's thesis that we should talk about it.  You should write that action and then backtrack here to see what you need for it.

See Morava's \textit{Local fields} paper

\begin{remark}
Theorem 2.6 of Greenlees--Strickland for a nice transchromatic perspective.  See also work of Stapleton and Schlank--Stapleton, of course.\todo{Flesh this out.}
\end{remark}


------

\begin{theorem}\citeme{Theorem A}
Let $E$ be any complex-oriented cohomology theory.  Take $G$ to be a finite group and let $\CatOf{Ab}_G$ be the full subcategory of the orbit category of $G$ built out of abelian subgroups of $G$.  Finally, let $X$ be a finite $G$--CW complex.  Then, each of the natural maps \[E^*(EG \times_G X) \to \lim_{A \in \CatOf{Ab}_G} E^*(EG \times_A X) \to \int_{A \in \CatOf{Ab}_G} E^*(BA \times X^A)\] becomes an isomorphism after inverting the order of $G$.  In particular, there is an isomorphism \[\frac{1}{|G|} E^* BG \to \lim_{A \in \CatOf{Ab}_G} \frac{1}{|G|} E^* BA. \qed\]
\end{theorem}

This is an analogue of Artin's theorem:
\begin{theorem}
There is an isomorphism \[\frac{1}{|G|} R(G) \to \lim_{C \in \CatOf{Cyclic}_G} \frac{1}{|G|} R(C). \qed\]
\end{theorem}


------

HKR intro material connecting Theorem A to character theory:

Recall that classical characters for finite groups are defined in the following situation: take $L = \Q^{\mathrm{ab}}$ to be the smallest characteristic $0$ field containing all roots of unity, and for a finite group $G$ let $Cl(G; L)$ be the ring of class functions on $G$ with values in $L$.  The units in the profinite integers $\widehat{\Z}$ act on $L$ as the Galois group over $\Q$, and since $G = \CatOf{Groups}(\widehat{\Z}, G)$ they also act naturally on $G$.  Together, this gives a conjugation action on $Cl(G; L)$: for $\phi \in \widehat{\Z}$, $g \in G$, and $\chi \in Cl(G; L)$, one sets \[(\phi \cdot \chi)(g) = \phi(\chi(\phi^{-1}(g))).\]  The character map is a ring homomorphism \[\chi: R(G) \to Cl(G; L)^{\widehat{\Z}},\] and this induces isomorphisms \[\chi: L \otimes R(G) \xrightarrow{\simeq} Cl(G; L)\] and even \[\chi: \Q \otimes R(G) \xrightarrow{\simeq} Cl(G; L)^{\widehat{\Z}}.\]

Now take $E = E_\Gamma$ to be a Morava $E$--theory of finite height $d = \height(\Gamma)$.  Take $E^*(B\Z_p^d)$ to be topologized by $B(\Z/p^j)^d$.  A character $\alpha: \Z_p^d \to S^1$ will induce a map $\alpha^*: E^* \CP^\infty \to E^* B\Z_p^d$.  We define $L(E^*) = S^{-1} E^*(B\Z_p^d)$, where $S$ is the set of images of a coordinate on $\CP^\infty_E$ under $\alpha^*$ for nonzero characters $\alpha$.  Note that this ring inherits an $\operatorname{Aut}(\Z_p^d)$ action by $E^*$--algebra maps.

The analogue of $Cl(G; L)$ will be $Cl_{d,p}(G; L(E^*))$, defined to be the ring of functions $\chi: G_{d, p} \to L(E^*)$ stable under $G$--orbits.  Noting that \[G_{d,p} = \operatorname{Hom}(\Z_p^d, G),\] one sees that $\operatorname{Aut}(\Z_p^d)$ acts on $G_{d,p}$ and thus on $Cl_{d,p}(G; L(E^*))$ as a ring of $E^*$--algebra maps: given $\phi \in \operatorname{Aut}(\Z_p^d)$, $\alpha \in G_{d,p}$, and $\chi \in Cl_{d,p}(G; L(E^*))$ one lets \[(\phi \cdot \chi)(\alpha) = \phi(\chi(\phi^{-1}(\alpha))).\]

Now we introduce a finite $G$--CW complex $X$.  Let \[\operatorname{Fix}_{d, p}(G, X) = \coprod_{\alpha \in \operatorname{Hom}(\Z_p^d, G)} X^{\operatorname{im} \alpha}.\]  This space has commuting actions of $G$ and $\operatorname{Aut}(\Z_p^d)$.  We set \[Cl_{d, p}(G, X; L(E^*)) = L(E^*) \otimes_{E^*} E^*(\operatorname{Fix}_{d,p}(G, X))^G,\] which is again an $E^*$--algebra acted on by $\operatorname{Aut}(Z_p^d)$.  We define the character map ``componentwise'': a homomorphism $\alpha \in \operatorname{Hom}(\Z_p^d, G)$ induces \[E^*(EG \times_G X) \to E^*(B\Z_p^d) \otimes_{E^*} E^*(X^{\operatorname{im} \alpha}) \to L(E^*) \otimes_{E^*} E^*(X^{\operatorname{im} \alpha}).\]  Taking the direct sum over $\alpha$, this assembles into a map \[\chi_{d,p}^G: E^*(EG \times_G X) \to Cl_{d,p}(G, X; L(E^*))^{\operatorname{Aut}(Z_p^d)}.\]
\todo{Nat taught you how to say all these things with $p$--adic tori, which was \emph{much} clearer.}
\begin{theorem}\citeme{Theorem C}
The invariant ring is $L(E^*)^{\operatorname{Aut}(\Z_p^d)} = p^{-1} E^*$, and $L(E^*)$ is faithfully flat over $p^{-1} E^*$.\todo{Checking this invariant ring claim is easiest done by comparing the functors the two things corepresent.}  The character map $\chi_{d,p}^G$ induces isomorphisms
\begin{align*}
\chi_{d,p}^G \co L(E^*) \otimes_{E^*} E^*(EG \times_G X) & \xrightarrow{\simeq} Cl_{d,p}(G, X; L(E^*)), \\
\chi_{d,p}^G \co p^{-1} E^*(EG \times_G X) & \xrightarrow{\simeq} Cl_{d,p}(G, X; L(E^*))^{\operatorname{Aut}(\Z_p^d)}.
\end{align*}
In particular, when $X = *$, there are isomorphisms
\begin{align*}
\chi_{d,p}^G \co L(E^*) \otimes_{E^*} E^*(BG) & \xrightarrow{\simeq} Cl_{d,p}(G; L(E^*)), \\
\chi_{d,p}^G \co p^{-1} E^*(BG) & \xrightarrow{\simeq} Cl_{d,p}(G; L(E^*))^{\operatorname{Aut}(\Z_p^d)}. \qed
\end{align*}
\end{theorem}

------

Jack gives an interpretation of this in terms of formal $\sheaf{O}_L$--modules.

------

I also have this summary of Nat's of the classical case:

It's not easy to decipher if you weren't there for the conversation, but here's my take on it. First, the map we wrote down today was the non-equivariant chern character: it mapped non-equivariant $KU \otimes \Q$ to non-equivariant $H\Q$, periodified. The first line on Nat's board points out that if you use this map on Borel-equivariant cohomology, you get nothing interesting: $K^0(BG)$ is interesting, but $H\Q^*(BG) = H\Q^*(*)$ collapses for finite $G$. So, you have to do something more impressive than just directly marry these two constructions to get something interesting.

That bottom row is Nat's suggestion of what ``more interesting'' could mean. (Not really his, of course, but I don't know who did this first. Chern, I suppose.) For an integer $n$, there's an evaluation map of (forgive me) topological stacks \[* \mmod (\Z/n) \times \operatorname{Hom}(* \mmod (\Z/n), * \mmod G) \xrightarrow{\mathrm{ev}} * \mmod G\] which upon applying a global-equivariant theory like $K_G$ gives \[K_{\Z/n}(*) \otimes K_G(\coprod_{\text{conjugacy classes of $g$ in $G$}} *) \xleftarrow{ev^*} K_G(*).\]

Now, apply the genuine $G$-equivariant Chern character to the $K_G$ factor to get \[K_{\Z/n}(*) \otimes H\Q_G(\coprod *) \from K_{\Z/n}(*) \otimes K_G(\coprod *),\] where the coproduct is again taken over conjugacy classes in G. Now, compute $K_{\Z/n}(*) = R(\Z/n) = \Z[x] / (x^n - 1)$, and insert this calculation to get \[K_{\Z/n}(*) \otimes H\Q_G(\coprod *) = \Q(\zeta_n) \otimes (\bigoplus_{\text{conjugacy classes}} \Q),\] where $\zeta_n$ is an $n${\th} root of unity.  As $n$ grows large, this selects sort of the part of the complex numbers $\C$ that the character theory of finite groups cares about, and so following all the composites we've built a map \[K_G(*) \to \C \otimes (\bigoplus_{\text{conjugacy classes}} \C).\]  The claim, finally, is that this map sends a $G$-representation (thought of as a point in $K_G(*)$) to its class function decomposition.


-----




\section{Knowns and unknowns}



\subsection*{Higher orientations}

$\TAF$ and friends

The $\alpha_{1/1}$ argument: Prop 2.3.2 of Hovey's $v_n$--elements of ring spectra

% The HLP calculations

\subsection*{Equivariance}

This is tied up with the theory of power operations in a way I've never really thought about.  Seems complicated.

\subsection*{Index theorems}

Connections with analysis

The Stolz--Teichner program








-----

Contexts for structured ring spectra

Difficulty in computing $\S_d \actson E_d^*$. (Gross--Hopkins and the period map.)

Barry's $p$--adic measures

Fixed point spectra and e.g. $L_{K(2)} \tmf$.

Blueshift, A--M--S, and the relationship to A--F--G?

Does $E_n$ receive an $E_\infty$ orientation?  Does $BP$?

Remark 12.13 of published $H_\infty$ AHS says their obstruction framework agrees with the $E_\infty$ obstruction framework (if you take everything in sight to have $E_\infty$ structures).  This is almost certainly related to the discussion at the end of Matt's thesis about the $MU$--orientation of $E_d$.\todo{Section 12.4 compares doing $H_\infty$ descent with doing $E_\infty$ descent and shows that they're the same (in the case of interest?).}

Hovey's paper on $v_n$--periodic elements in ring spectra.  He has a nice (and thorough!) exposition on why one should be interested in bordism spectra and their splittings: for instance, a careful analysis of $M\Spin$ will inexorably lead one toward studying $KO$.  It would be nice if studying $M\String$ (and potentially higher analogues) would lead one toward non-completed, non-connective versions of $EO_n$.  Talk about $BoP$, for instance.
