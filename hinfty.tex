% -*- root: main.tex -*-

\chapter{Power operations}

\todo{Write an introduction for me.}


\todo[inline]{There should be a context-based presentation of this chapter's material too.  What do contexts for structured ring spectra look like?  Why would you consider them --- what object are you trying to approximate?  How do you guess that the algebraic model is reasonable until you're aware of something like Strickland's theorem?}



\section{Isogenies}

\todo[inline]{Something that these notes routinely fail to do is to lead into the algebraic geometry in a believable way.  ``Today we're going to talk about isogenies'' --- and then, lo' and behold, isogenies appear the next day in algebraic topology.  This book would read much better if it showed how these structures were guessed to exist to begin with.}

Here's a definition of an isogeny.  Weierstrass preparation can be phrased as saying that a Weierstrass map is a coordinate change and a standard isogeny.\todo{Where does this come from?  It's Strickland's, I know that.}
\begin{definition}
Take $C$ and $D$ to be formal curves over $X$.  A map $f: C \to D$ is an \textit{isogeny} when the induced map $C \to C \times_X D$ exhibits $C$ as a divisor on $C \times_X D$ as $D$--schemes.
\end{definition}


In fact, every map in positive characteristic can be factored as a coordinate change and an isogeny, which is a weak form of preparation.


Lubin's finite quotients of formal groups. (Interaction with the Lubin--Tate moduli problem?  Or does this belong in the next day?)


Write out isogenies of the additive formal group, note that you just get the unstable Steenrod algebra again.  This is a remarkable accident.


Push and pull maps for divisor schemes


Moduli of subgroup divisors


The Drinfel'd moduli ring, level structures




\section{HKR characters}

There's a sufficient amount of reliance on character theory in Matt's thesis that we should talk about it.  You should write that action and then backtrack here to see what you need for it.







\section{Ando coordinates}

Ando's Theorem 3.4.4: Let $D_j$ be the ring extension of $E_n$ which trivializes the $p^j$--torsion subgroup of $\G_{E_n}$.  Let $H$ be a finite subgroup of $\G_{E_n}(D_k)$.  There is an unstable transformation of ring-valued functors \[E_n X \xrightarrow{\Psi^H} D_j \otimes E_n X,\] and if $F$ is an Ando coordinate then for any line bundle $\L \to X$ there is a formula \[\psi^H(e\L) = \prod_{h \in H} (h +_F e\L) \in D_j \otimes E_n(X).\]

$D_j$ is Galois over $E_n$ with Galois group $\GL_n(\Z/p^j)$.  If $\rho$ is a collection of finite subgroups weighted by elements of $E_n$ which is stable under the action of the Galois group, then $\Psi^\rho$ descends to take values in just $E_n$.  (For example, the entire subgroup has this property.)

This is built by a character map.  Take $H \subseteq F(D_j)[p^j]$ to be a finite subgroup again; then there is a map \[\chi^H: E_n(D_{H^*} X) \to D_j \otimes E_n(X),\] where $D_{H^*}$ denotes the extended power construction on $X$ using the Pontryagin dual of $H$.  This composes to give an operation \[Q^H: MU^{2*}(X) \xrightarrow{P_{H^*}} MU^{2|H|*}(D_{H^*} X) \to E_n^{2|H|*}(D_{H^*} X) \xrightarrow{\chi^H} D_j \otimes E_n^{2|H|*}(X).\]  Then $Q^H$ is a ring homomorphism with effects
\begin{align*}
Q^H F^{MU} & = F/H, &
Q^H(e_{MU} \L) & = \prod_{h \in H} h +_F e\L.
\end{align*}

Then we need to factor $Q^H: MU(X) \to D_j \otimes E_n(X)$ across the orienting map $MU \to E_n$.  Since $E_n$ is Landweber flat and $Q^H$ is a ring map, it suffices to do this for the one--point space, i.e., to construct a ring homomorphism \[\Psi^H: E_n \to D_j\] so that $\Psi^H = \Psi^H(*) \otimes Q^H$.  The first condition above then translates to $\Psi^H F^{MU} = F/H$.

Matt claims that 3.2.10 is the beating heart of the paper.  Certainly it deserves mention here, since we saw the version in Quillen's paper weeks ago.

Section 3.3 is the part that uses HKR character theory.  The beginning only uses the familiar description of $BA_{E_n}$ for finite abelian $A$, but then it gets serious.






\section{Strickland's theorem}

Following... the original? Following Nat?

Continuing on from the above, if we expected $E_n$ to be $E_\infty$ (or even $H_\infty$) so that it had power operations, then we would want to understand $E_n B\Sigma_{p^j}$ and match that with the operations we see.

---

There are union maps \[B\Sigma_j \times B\Sigma_k \to B\Sigma_{j+k},\] stable transfer maps \[B\Sigma_{j+k} \to B\Sigma_j \times B\Sigma_k,\] and diagonal maps \[B\Sigma_j \to B\Sigma_j \times B\Sigma_j.\]  These induce a coproduct $\psi$ as well as products $\times$ and $\bullet$ on $E^0 \P \S^0$, where $\P\S^0 = \coprod_{j=0}^\infty B\Sigma_j$ is the free $E_\infty$--ring on $\S^0$.  This is a Hopf ring, and under $\times$ alone it is a formal power series ring.  The $\times$--indecomposables (which, I guess, are analogues of considering additive unstable cooperations) are \[Q^\times E^0 \P\S^0 = \prod_{k \ge 0} \left( E^0 B\Sigma_{p^k} / \operatorname{tr} E^0 B\Sigma_{p^{k-1}}^p \right),\] where the $k${\th} factor in the product is naturally isomorphic to $\sheaf{O}_{\Sub_{p^k}(\G)}$.  The primitives are also accessible as the kernel of the dual restriction map.

Theorem 3.2 shows that $E^0 B\Sigma_k$ is free over $E^0$, Noetherian, and of rank controlled by generalized binomial coefficients.  Prop 3.4 is the only place where work gets done, and it's all in terms of $K$--theory and HKR characters.

There's actually an extra coproduct, coming from applying $D$ to the fold map $S^0 \vee S^0 \to S^0$.

The main content of Prop 5.1 (due to Kashiwabara) is that $K_0 \P \S^0$ injects into $K_0 \OS{BP}{0}$.  Grading $K_0 \P \S^0$ using the $k$--index in $B\Sigma_k$, you can see that it's of graded finite type, so we need only know it has no nilpotent elements to see that $K_0 \P \S^0$ is $\ast$--polynomial.  This follows from our computation that $K_0 \OS{BP}{0}$ is a tensor of power series and Laurent series rings.  Corollary 5.2 is about $K_0 Q S^0$, which is the group completion of $K_0 \P \S^0$, so it's the tensor of $K_0 \P \S^0$ with a graded field.

Prop 5.6, using a double bar spectral sequence method, shows that $K^0 Q S^2$ is a formal power series algebra.  Tracking the spectral sequences through, you'll find that $Q^\times K^0 Q S^0$ agrees with $P K^0 Q S^2$.  (You'll also notice that $K^0 Q S^2$ only has one product on it, cf.\ Remark 5.4.)

Snaith's theorem says $\Sigma^\infty QX = \Sigma^\infty \P X$ for connected spaces $X$.  You can also see (just after Theorem 6.2) the nice equivalences \[\P_k S^2 \simeq B\Sigma_k^{V_k} \simeq \P_k(S^0)^{V_k},\] where superscript denotes Thom complex.  So, for a complex-orientable cohomology theory, you can learn about $\P_k S^0$ from $\P_k S^2$.  In particular, we finally learn that $E^0 \P S^0$ is a formal power series $\times$--algebra (once checking that the Thom isomorphism is a ring map).  (We already knew the homological version of this claim.)

Section 8 has a nice discussion about indecomposables and primitives, to help move back and forth between homology and cohomology.  It probably helps most with the dimension count argument below that we aren't going to get into.

Start again with $D_{p^k} S^2 \simeq B\Sigma_{p^k}^{V_{p^k}}$.  We can associate to this a divisor $\D(V_{p^k})$ on $(B\Sigma_{p^k})_E$, which we know little about, but it is classified by a map to $\Div_{p^k} \CP^\infty_E$.  This receives a closed inclusion from $\Sub_{p^k} \CP^\infty_E$, so their pullback $Z_k$ is the largest subscheme of $(B\Sigma_{p^k})_E$ over which $\D(V_{p^k})$ is a subgroup divisor.
\begin{center}
\begin{tikzcd}
H_k \arrow{rr} \arrow{dd} & & \D(V_{p^k}) \\
& Z_k \arrow{rr} \arrow{rd} & & \Sub_{p^k} \CP^\infty_E \arrow{rd} \\
\Spf E^0 B\Sigma_{p^k} / \mathrm{tr} \arrow{rr} \arrow[densely dotted]{ru} & & (B\Sigma_{p^k})_E \arrow{rr} \arrow[crossing over,leftarrow]{uu} & & \Div_{p^k} \CP^\infty_E
\end{tikzcd}
\end{center}
We will show the existence of the dashed map, implying that the restricted divisor $H_k$ is a subgroup divisor on $Y_k = \Spf E^0 B\Sigma_{p^k} / \mathrm{tr}$.

(Prop 9.1:) This proof falls into two parts: first we construct a family of maps to $(B\Sigma_{p^k})_E$ on whose image $\D(V_{p^k})$ restricts to a subgroup divisor, and then we show that the union of their images is exactly $Y_k$.  Let $A$ be an abelian $p$--subgroup of $\Sigma_{p^k}$ that acts transitively on $\{1, \ldots, p^k\}$ (i.e., it is not boosted from some transfer).  The restriction of $V_{p^k}$ to $A$ is the regular representation, which splits as a sum of characters $V_{p^k}|_A = \bigoplus_{\L \in A^*} \L$.  Identifying $BA_E = \InternalHom{FormalGroups}(A^*, \CP^\infty_E)$, $\D(V_{p^k})$ restricts all the way to $\sum_{\L \in A^*} [\phi(\L)]$, with $\phi: A^* \to $``$\Gamma(\operatorname{Hom}(A^*, \G), \G)$''.  In Finite Subgroups of Formal Groups (see Props 22 and 32), we learned that the restriction of $\D(V_{p^k})$ further to $\Level(A^*, \CP^\infty_E)$ is a subgroup divisor.  So, our collection of maps are those of the form \[\Level(A^*, \CP^\infty_E) \to \InternalHom{FormalGroups}(A^*, \CP^\infty_E) = BA_E \to (B\Sigma_{p^k})_E.\]  Here, finally, is where we have to do some real work involving Chern classes and commutative algebra, so I'm inclined to skip it in the lectures.  Finally, you do a dimension count to see that $Z_k$ and $\Spf E^0 B\Sigma_{p^k} / \mathrm{tr}$ have the same dimension (which requires checking enough commutative algebra to see that ``dimension'' even makes sense), and so you show the map is injective and you're done.








\section{Interaction with $\Theta$--structures}

The Ando--Hopkins--Strickland result that the $\sigma$--orientation is an $H_\infty$--map

The main classical point is that an $MU\<0\>$--orientation is $H_\infty$ when the following diagram commutes for every choice of $A$:
\begin{center}
\begin{tikzcd}
(BA^* \times \CP^\infty)^{V_{reg} \otimes \L} \arrow{r} & D_n MU\<0\> \arrow{d} \arrow{r} & D_n E \arrow{d} \\
& MU\<0\> \arrow{r} & E
\end{tikzcd}
\end{center}
(This is equivalent to the condition given in the section on Matt's thesis.  In fact, maybe I should try writing this so that Matt's thesis uses the same language?)  If you write out what this means, you'll see that a given coordinate on $E$ pulls back to give two elements in the $E$--cohomology of that Thom spectrum (or: sections of the Thom sheaf), and the orientation is $H_\infty$ when they coincide.

Similarly, an $MU\<6\>$--orientation corresponds to a section of the sheaf of cubical structures on a certain Thom sheaf.  Using the $H_\infty$ structures on $MU\<6\>$ and on $E$ give two sections of the pulled back sheaf of cubical structures, and the $H_\infty$ condition is that they agree for all choices of group $A$.




\todo{Section 12.4 compares doing $H_\infty$ descent with doing $E_\infty$ descent and shows that they're the same (in the case of interest?).}




\subsection*{Other stuff that goes in this chapter}

Neil's \textit{Finite Subgroups of Formal Groups} has (in addition to lots of results) a section 14 where he talks about the action of a generalized Hecke algebra on the $E$--theory of a space.

Dyer--Lashof operations, the Steenrod operations, and isogenies of the formal additive group \citeme{See Neil's \textit{Steenrod algebra} note, maybe? Talk to Mike?}

Another augmentation to the notion of a context: working not just with $E_* X$ but with $E_*(X \times BG)$ for finite $G$.
