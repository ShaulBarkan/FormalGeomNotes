% -*- root: main.tex -*-

\chapter{Power operations}

\todo{I wish this had a better title.}

\todo{Write an introduction for me.}


\todo[inline]{There should be a context-based presentation of this chapter's material too.  What do contexts for structured ring spectra look like?  Why would you consider them --- what object are you trying to approximate?  How do you guess that the algebraic model is reasonable until you're aware of something like Strickland's theorem?}

\todo[inline]{Since you spend so much time talking about descent in other parts of these notes, maybe you should also read the end of the AHS $H_\infty$ paper where they claim to recast their results in the usual language of descent.}

\todo[inline]{Conversation with Nat on 2/9 suggests taking the following route in this chapter:
contexts for $E_\infty$ mapping spaces in general;
Subgroups and level structures;
the Drinfel'd ring and the universal level structure;
the isogenies pile;
power operations and Adams operations, after Ando (naturally indexed vs indexed on subgroups; have a look at the Screenshot you took on this day);
comparison of comodules $M$ for the isogenies pile with the action of $M_n(\Z_p)$ on $M \otimes_{E_n^*} D_\infty$ (this is a modern result due to Tomer, Tobi, Lukas, and Nat);
$H_\infty$ $MU$--orientations and Matt's thesis;
the analogous results for $\Theta^k$--structures.
In particular, leave character theory, $p$--divisible groups, and rational phenomena for spillover at the end of the year. They aren't strictly necessary to telling the story; you just need to know a little about the Drinfel'd ring to construct Matt's maps.  (If you have time, though, the point is that the rationalized Drinfel'd ring carries the universal level structure which is also an isomorphism.)}

\todo[inline]{The stuff around 4.3.1-2 of Matt's published thesis talks about $H_\infty$--maps being determined by their values on $*$ and $\CP^\infty$, which is an interesting result.  You might also compare with Butowiez--Turner.}

\todo[inline]{Work in height $1$ (and height $2$??) examples through this?  $K$--theory is pretty accessible, and the height $2$ examples are somewhat understood (Charles, Yifei), and they're both relevant for the elliptic $M\String$ story.  (There's also the pile of elliptic curves with isogenies...)}

\todo[inline]{Nat warns that the very end of Matt's thesis uses character theory for $S^1$, which you have to be very careful about to pull off correctly.  ($S^1$ is not a finite group, but in certain contexts it can be approximated by its torsion subgroups...)}

\todo[inline]{Yifei warned me that Matt's ``there exists a unique coordinate...''\ Lemma is specifically about lifting the \emph{Honda} formal group law over $\F_q$.  If you want to do this with elliptic cohomology or something, then you need a stronger statement (and it's clear what this statement should be, but no one has proven it).}






\section{$E_\infty$ ring spectra and their contexts}

\todo[inline]{Mike has suggested looking at the paper \textit{The $K$--theory localization of an unstable sphere}, by Mahowald and Thompson.  In it, they manually construct a resolution of $S^{2n+1}$ suitable for computing the unstable Adams spectral sequence for $K$--theory, but the resolution that they build is also exactly what you would use to compute the mapping spectral sequence for $E_\infty(K^{S^{2n-1}}, K)$.  Additionally, because the unstable $K$--theoretic operations are exhausted by the power operations, these two spectral sequences converge to the same target.

Purely in terms of the $E_\infty$ version, one can consider the composition of spectral sequences \[\Ext_{\Z[\theta]}(\Z, \operatorname{Der}_{K_*-alg}(K^* X, K^*)) \Rightarrow \operatorname{Der}_{K_*-Dyer-Lashof-alg}(K^* X, K^*) \Rightarrow E_\infty(\widehat{\S^0}^X, K^\wedge_p)\] and \[E_\infty(\widehat{\S^0}^X, K^\wedge_p)^{h\Z_p^*} = E_\infty(\widehat{\S^0}^X, \widehat{\S^0})\] where the first spectral sequence is a composition spectral sequence for derivations in $K_*$--algebras and then derivations respecting the Mandell's $\theta$--operation.  If $X$ is an odd sphere, then $K^* X$ has no derivations and this composite spectral sequence collapses, making the composition possible.

This is also related to recent work of Behrens--Rezk on the Bousfield--Kuhn functor...}

\todo[inline]{Another unpublished theorem of Hopkins and Lurie is that the natural map $Y = F(*, Y) \to E_\infty(E_n^Y, E_n)$ is an equivalence when $Y$ is a finite Postnikov tower in the range of degrees that $E_n$ can see.}




\section{Subgroups and level structures}

\todo[inline]{Something that these notes routinely fail to do is to lead into the algebraic geometry in a believable way.  ``Today we're going to talk about isogenies'' --- and then, lo' and behold, isogenies appear the next day in algebraic topology.  This book would read much better if it showed how these structures were guessed to exist to begin with.}

Here's a definition of an isogeny.  Weierstrass preparation can be phrased as saying that a Weierstrass map is a coordinate change and a standard isogeny.\citeme{Definition 5.17 of FSFG}
\begin{definition}
Take $C$ and $D$ to be formal curves over $X$.  A map $f: C \to D$ is an \textit{isogeny} when the induced map $C \to C \times_X D$ exhibits $C$ as a divisor on $C \times_X D$ as $D$--schemes.
\end{definition}
\todo{Be sure to compare this definition with the usual one for formal groups: surjections with finite cokernel.  These are easy to come up with lots of examples for!  (Don't ditch this definition, though, since it's the one that lets you prove something about Weierstrass preparation geometrically.)}




In fact, every map in positive characteristic can be factored as a coordinate change and an isogeny, which is a weak form of preparation.


Lubin's finite quotients of formal groups. (Interaction with the Lubin--Tate moduli problem?  Or does this belong in the next day?)


Write out isogenies of the additive formal group, note that you just get the unstable Steenrod algebra again.  This is a remarkable accident.


Push and pull maps for divisor schemes


Moduli of subgroup divisors


The Drinfel'd moduli ring, level structures

----

\begin{lemma}\citeme{Prop 6.2 of HKR}
The following conditions on a homomorphism \[\phi: \Lambda_r^* \to F[p^r](R)\] are equivalent:
\begin{enumerate}
\item For all $\alpha \ne 0$ in $\Lambda_r^*$, $\phi(\alpha)$ is a unit (resp., not a zero-divisor).
\item The Hopf algebra homomorphism \[R\ps{x} / [p^r](x) \to R^{\Lambda_r^*}\] is an isomorphism (resp., a monomorphism). \qed
\end{enumerate}
\end{lemma}

\begin{lemma}\citeme{Shortly after Prop 6.2 of HKR. Section 7?}
Let $\L_r(R)$ be the set of all group homomorphism \[\phi: \Lambda_r^* \to F[p^r](R)\] satisfying either of the conditions 1 or 2 above.  This functor is representable by a ring \[L_r(E^*) := S^{-1} E^*(B\Lambda_r)\] that is finite and faithfully flat over $p^{-1} E^*$.  (Here $S$ is generated by the $\phi(\alpha)$ with $\alpha \ne 0$, $\phi: \Lambda_r^* \to F[p^r](E^* B\Lambda_r)$ the canonical map.)
\end{lemma}

---

Section 2: complete local rings

``Galois'' means $R \to S$ a finite extension of integral domains has $R$ as the fixed subring for $\operatorname{Aut}_R(S)$ and $S$ is free over $R$.  Galois extension of rings implies the extension of fraction fields is Galois.  The converse holds for finite (finitely generated as a module) dominant (kernel of $f$ is nilpotent) maps of smooth (regular local ring) schemes.

Section 3: basic facts about formal groups

definition of height

Section 4: basic facts about divisors

Since $x -_F a \dot= x - a$, you can treat the divisor $[a]$ (defined in a coordinate by the ideal sheaf generated by $x - x(a)$) as generated just by $x - a$.

\begin{lemma}\citeme{Prop 4.6 of Finite Subgroups}
Let $D$ and $D'$ be two divisors on $\G$ over $X$.  There is then a closed subscheme $Y \le X$ such that for any map $a: Z \to X$ we have $a^* D \le a^* D'$ if and only if $a$ factors through $Y$. \qed
\end{lemma}

Section 5: quotient by a finite sbgp is again a fml gp

\begin{definition}
A \textit{finite subgroup} of $\G$ will mean a divisor $K$ on $\G$ which is also a subgroup scheme.  Let $\sheaf{O}_{\G/K}$ be the equalizer
\begin{center}
\begin{tikzcd}
\sheaf O_{\G/K} \arrow{r} & \sheaf O_{\G} \arrow[shift left=0.2cm]{r}{\mu^*} \arrow[shift right=0.2cm]{r}{\pi^*} & \sheaf O_K \otimes_{\sheaf O_X} \sheaf O_{\G}.
\end{tikzcd}
\end{center}
\end{definition}

\begin{lemma}\citeme{Theorem 5.3 of Finite Subgroups}
Write $y = N_\pi \mu^* x \in \sheaf O_{\G}$.\footnote{Remember that if $f: X \to Y$ is a finite flat map, then $N_f: \sheaf O_X \to \sheaf O_Y$ is the nonadditive map sending $u$ to the determinant of multiplication by $u$, considered as an $\sheaf O_Y$--linear endomorphism of $\sheaf O_X$.}  Then $y \equiv x^{p^m} \pmod{\m_X}$ and $\sheaf O_{\G/K} = \sheaf O_X\ps{y}$.  Moreover, the projection $\G \to \G/K$ is the categorical cokernel of $K \to \G$.  This all commutes with base change: given $f: Y \to X$ we have $f^* \G / f^* K = f^*(\G/K)$. \qed \todo{Expand this out in the case of a subgroup scheme given by a sum of point divisors.}
\end{lemma}

\todo{cf.\ also Prop 2.2.2 of Matt's thesis}

Section 6: coordinate-free lubin-tate theory

nothing you haven't already seen. in fact, most of it is done in coordinates, with only passing reference to the decoordinatization.

Section 7: level--$A$ structures: smooth, finite, flat

\todo{Be careful to distinguish the physical group $A$ from the associated \emph{constant group scheme}.}
As discussed long ago, for finite abelian $p$--groups there's a scheme \[\InternalHom{FormalGroups}(A, \G)(Y) = \InternalHom{Groups}(A, \G(Y)).\]  If $\G$ were a discrete group, we could decompose this as \[\text{``$\InternalHom{FormalGroups}(A, \G) = \coprod_{B \le A} \Mono(A/B, \G)$''}\] along the different kernel types of homomorphisms, but $\Mono$ does not exist as a scheme.\todo{Come up with a really compelling example.  You had one when you were talking to Danny and Jeremy.  Probably you got it \emph{from} Jeremy.}  Level structures approximate this as best one can be approximating $\G$ by something essentially discrete: an \'etale group scheme.

For a map $\phi: A \to \G(Y)$, we write $[\phi A] = \sum_{a \in A}[\phi(a)]$.  We also write $\Lambda = (\Q_p / \Z_p)^n$, so that $\Lambda[p^m] = (\Z/p^m)^{\times n}$.  Note \[|\CatOf{AbelianGroups}(A, \Lambda)| = |A|^n = \operatorname{rank} \left( \InternalHom{FormalGroups}(A, \G) \to X \right).\]

\begin{definition}
A \textit{level--$A$ structure} on $\G$ over an $X$--scheme $Y$ is a map $\phi: A \to \G(Y)$ such that $[\phi A[p]] \le G[p]$ as divisors.  A \textit{level--$m$ structure} means a level--$\Lambda[p^m]$ structure.
\end{definition}

\begin{lemma}\citeme{Prop 7.2-4 of Finite Subgroups}
The functor from schemes over $X$ to sets given by \[Y \mapsto \{\text{level--$A$ structures on $\G$ over Y}\}\] is represented by a finite flat scheme $\Level(A, \G)$ over $X$.  It is contravariantly functorial for monomorphisms of abelian groups.  Also, if $\phi: A \to \G$ is a level structure then $[\phi A]$ is a subgroup divisor and $[\phi A[p^k]] \le \G[p^k]$ for all $k$.  In fact, if $A = \Lambda[p^m]$ then $[\phi A] = \G[p^m]$.  \qed  \todo{I can't imagine proving this.  It's worth noting that it's proven by considering just the universal case, which we know to be smooth.}
\end{lemma}

In Section 26 of FPFP Neil says there's a decomposition into irreducible components \[\operatorname{Hom}(A, \G) = \operatorname{Hom}(A, \G_{\mathrm{red}}) = \bigcup_B \Level(A/B, \G)\] and this $\bigcup$ turns into a $\coprod$ after inverting $p$.  He also mentions this as motivation in Finite Subgroups, but he doesn't appear to prove it?

Section 8: maps among level--$A$ schemes, their Galois behavior

\begin{theorem}\citeme{Theorem 8.1 of Finite Subgroups}
Let $A$, $B$ be finite abelian $p$--groups of rank at most $n$, and let $u: A \to B$ be a monomorphism. Then:
\begin{enumerate}
\item \[\CatOf{FormalSchemes}_X(\Level(B, \G), \Level(A, \G)) = \operatorname{Mono}(A, B).\]
\item Such homomorphisms are detected by the behavior at the generic point.
\item The map $u^!: \Level(B, \G) \to \Level(A, \G)$ is finite and flat.
\item If $B \simeq \Lambda[p^m]$, then $u^!$ is a Galois covering.
\item The torsion subgroup of $\G(\Level(A, \G))$ is $A$. \qed
\end{enumerate}
\end{theorem}

Section 9: epimorphisms of groups become maps of level schemes, quotients by level structures

Let $\G_0$ be a formal group of height $n$ over $X_0 = \Spec \kappa$.  For every $m$, the divisor $p^m[0]$ is a subgroup of $\G_0$.  We write $\G_0\<p^m\>$ for the quotient group $\G_0 / p^m[0]$ and $\G\<m\> \to X\<m\>$ for the universal deformation of $\G_0\<m\> \to X_0$.  Note that $\G_0[p] = p^n[0]$, which induces isomorphisms $\G_0\<m+n\> \to \G_0\<m\>$, and we use this to make as many identifications as we can.

\begin{lemma}\citeme{9.1 of Finite Subgroups}
Let $u: A \to B$ be an epimorphism of abelian $p$--groups wit kernel $|\ker(u)| = p^\ell$.  Then $u$ induces a map \[u_!: \Level(A, \G\<m\>) \to \Level(B, \G\<m+\ell\>).\]  Also, if $A = \Lambda[p^m]$, then $u_!$ is a Galois covering with Galois group \[\Gamma = \{\alpha \in \operatorname{Aut}(A) \mid u\alpha = u\}. \qed\]
\end{lemma}

\begin{corollary}\citeme{Interstitial text between 9.1 and 9.2 of Finite Subgroups}
In particular, the map $A \to 0$ induces a map \[0_!: \Level(A, \G\<m\>) \to \Level(0, \G\<m+\ell\>) = X\<m+\ell\>\] which extracts quotient formal groups from level structures.  In the case $A = \Lambda[p^\ell]$, $0_!$ is just the projection $0^!$. \qed
\end{corollary}

Section 10: moduli of subgroup schemes

\begin{theorem}\citeme{Theorem 10.1 of Finite Subgroups}
The functor \[Y \mapsto \{\text{subgroups of $\G \times_X Y$ of degree $p^m$}\}\] is represented by a finite flat scheme $\Sub_{p^m}(\G)$ over $X$ of degree $|\Sub_{p^m}(\Lambda)|$.  The formation commutes with base change. \qed
\end{theorem}

We can at least give the construction: let $D$ be the universal divisor defined over $Y = \Div_{p^m}(\G)$ with equation $f_D(x) = \sum_{k=0}^{p^m} c_k x^k$.  There are unique elements $a_{ij} \in \sheaf O_Y$ such that \[f(x +_F y) = \sum_{i,j=0}^{p^m-1} a_{ij} x^i y^j \pmod{f(x), f(y)}.\]  Define \[\Sub_{p^m}(\G) = \Spf \sheaf O_Y / (c_0, a_{ij} \mid 0 \le i, j < p^m).\]  Finiteness, flatness, and rank counting are what take real work, starting with an arithmetic fracture square.

Section 13: deformation theory of isogenies

\begin{definition}
Suppose we have a morphism of formal groups
\begin{center}
\begin{tikzcd}
\G_0 \arrow{r}{q_0} \arrow{d} & \G'_0 \arrow{d} \\
X_0 \arrow{r}{f_0} & X'_0
\end{tikzcd}
\end{center}
such that the induced map $\G_0 \to f_0^* \G'_0$ is an isogeny of degree $p^m$.  By a deformation of $q_0$ we mean a prism
\begin{center}
\begin{tikzcd}
\mathbb H \arrow{rd}{q} \arrow{dd} & & \mathbb H_0 \arrow{ll} \arrow{rr} \arrow{rd} \arrow{dd} & & \G_0 \arrow{rd}{q_0} \arrow{dd} \\
& \mathbb H' & & \mathbb H'_0 \arrow[crossing over]{ll} \arrow[crossing over]{rr} & & \G'_0 \arrow{dd} \\
Y \arrow{rd}{1} & & Y_0 \arrow{ll} \arrow{rr} \arrow{rd}{1} & & X_0 \arrow{rd}{f_0} \\
& Y \arrow[crossing over,leftarrow]{uu} & & Y_0 \arrow{ll} \arrow{rr} \arrow[crossing over,leftarrow]{uu} & & X'_0,
\end{tikzcd}
\end{center}
where the middle face is the pullback of the left face, the back-right and front-right faces are pullbacks, so that $q$ is also an isogeny of degree $p^m$.
\end{definition}

Let $\G/X$ be the universal deformation of $\G_0$, let $a: \Sub_{p^m}(\G) \to X$ be the usual projection, and let $K < a^* \G$ be the universal example of a subgroup of degree $p^m$.  As $\Sub_{p^m}(\G)$ is a closed subscheme of $\Div_{p^m}(\G)$ and $\Div_{p^m}(\G)_0 = X_0$, we see that $\Sub_{p^m}(\G)_0 = X_0$.  There is a unique subgroup of order $p^m$ of $\G_0$ defined over $X_0$, viz.\ the divisor $p^m[0] = \Spf \sheaf O_{\G_0} / x^{p^m}$.  In particular, $K_0 = p^m[0] = \ker(q_0)$.  It follows that there is a pullback diagram as shown below:
\begin{center}
\begin{tikzcd}
(a^* \G/K)_0 \arrow{r}{\simeq} \arrow{d} & \G_0 / p^m[0] \arrow{r}{\overline q_0, \simeq} \arrow{d} & \G'_0 \arrow{d} \\
\Sub_{p^m}(\G)_0 \arrow{r}{a_0, \simeq} & X_0 \arrow{r}{f_0, \simeq} & X'_0.
\end{tikzcd}
\end{center}
We see that $a^* \G \to a^* \G/K$ is a deformation of $q_0$, and it is terminal in the category of such.

Now let $\G' / X'$ be the universal deformation of $\G'_0 / X'_0$.  The above construction also exhibits $a^* \G/K$ as a deformation of $\G'_0$, so it is classified by a map $b: \Sub_{p^m}(\G) \to X'$ extending the map $b_0 = f_0 \circ a_0: \Sub_{p^m}(\G)_0 \to X'_0$.

\begin{theorem}\citeme{Prop 13.1 of Finite Subgroups, \emph{hard}}
$b$ is finite and flat of degree $|\Sub_{p^m}(\Lambda)|$. \qed
\end{theorem}

\todo[inline]{Cf. Matt's thesis's Prop 2.5.1: $\Phi$ is a formal group over $\F_p$, $F$ a lift of $\Phi$ to $E_n$, $H$ a finite subgroup of $F(D_k)$, then $F/H$ is a lift of $\Phi$ to $D_k$.  (This is because the quotient map to $F/H$ reduces to $t \mapsto t^{p^r}$ for some $r$ over $\F_p$, which is an endomorphism of $\Phi$, so the quotient map over the residue field doesn't do anything!)  See also Prop 2.5.4, where he characterizes all isogenies of this sort as arising from this construction.}

Section 14: connections to AT

Neil's \textit{Finite Subgroups of Formal Groups} has (in addition to lots of results) a section 14 where he talks about the action of a generalized Hecke algebra on the $E$--theory of a space.  Let $a$ and $b$ be two points of $X$, with fibers $\G_a$ and $\G_b$, and let $q: \G_a \to \G_b$ be an isogeny.  Then there's an induced map $(Z_E)_a \to (Z_E)_b$, functorial in $q$ and natural in $Z$.  ``Certain $\Ext$ groups over this Hecke algebra form the input to spectral sequences that compute homotopy groups of spaces of maps of strictly commutative ring spectra, for example.''  \textbf{This sounds like the beginning of an answer to my context question.}

Section 11: flags of controlled rank ascending to $\G[p]$ and a map $\Level(1, \G) \to \operatorname{Flag}(\lambda, \G)$.
Section 12: the orbit scheme $\operatorname{Type}(A, \G) = \Level(A, \G) / \operatorname{Aut}(A)$: smooth, finite, flat
Section 15: formulas for computation
Section 16: examples

------

\begin{theorem}\citeme{See Theorem 2.4.1 of Ando's thesis, though he just cites other people}
Let $R$ be a complete local domain with positive residue characteristic $p$, and let $F$ be a formal group of finite height $d$ over $R$.  If $\sheaf O$ is the ring of integers in the algebraic closure of the fraction field of $R$, then $F(\sheaf O)[p^k] \cong (\Z/p^k)^d$ and $F(\sheaf O)_{\mathrm{tors}} \cong (\Q_p / \Z_p)^d$. \qed
\end{theorem}

------

Section 20 of FPFP is about ``full sets of points'' and the comparison with the cohomology of the flag variety of a vector bundle.

------

Talk with Nat:
\begin{itemize}
\item Definitions in terms of divisors.
\item Equalizer diagram for quotients by finite subgroups.
\item The image of a level structure $\ell$ is a subgroup divisor.
\item The schemes classifying subgroups and level structures (which are hard and easy respectively, and which have hard and easy connections to topology respectively).
\item It's easy to give explicit examples of the behavior of level structures based on cyclic groups.
\item Galois actions on the rings of level structures.
\end{itemize}




\section{The Drinfel'd ring and the universal level structure}

Talk with Nat:
\begin{itemize}
\item Recall the Lubin--Tate moduli problem.
\item Show that quotients of deformations by finite subgroups give deformations again.
\item Define the Drinfel'd ring.
\item As an $E^0$--algebra, it carries the universal level structure.
\item As an ind--(complete local ring), it corepresents deformations (by precomposition with the map $E^0 \to D_n$) \emph{equipped with level structures}.
\item Describe the action by $GL_n(\Z_p)$. (Hint at the action by $M_{n \times n}(\Z_p)$ with $\det \ne 0$.)
\item Describe the isogenies pile and its relation to all this?  (This doesn't really fit precisely, but it may be good to put here, on an algebraic day.)
\end{itemize}










\section{Descending coordinates along level structures}

% \begin{enumerate}
% \item If $V = (1 - \L)$ is the reduced canonical line bundle over $\CP^\infty$, then using all the above we have \[\ThomSheaf{V} \cong \pi^* 0^* \sheaf I (0) \otimes \sheaf I(0)^{-1} = \Theta^1(\sheaf I(0)),\] where $\pi: \G_E \to S_E$ is the structural map and $\Theta^1$ is the usual.
% \item Let $A$ be a finite abelian group.  An element $a \in A$ can be regarded as a character of $A^*$, and we let $V_a$ denote the associated line bundle over $BA^*$.  This gives a group homomorphism $\chi\co A \to \G(BA^*_E)$.  The line bundle $\ThomSheaf{V_a \otimes V \otimes \L}$ over $BA^*_E \times X_E \times \G$ is \[\ThomSheaf{V_a \otimes V \otimes \L} \cong T_a^* \sheaf I(D^{-1}),\] and taking $V$ to be the trivial line bundle over a point gives \[\ThomSheaf{V_a \otimes \L} \cong T_a^* \sheaf I(0) = \sheaf I(a^{-1}).\]
% \item Now let $V_{reg} = \bigoplus_{a \in A} V_a$ be the regular representation of $A^*$.  Over the scheme $(BA^*)_E \times \G$, the line bundle associated to the Thom complex of $V_{reg} \otimes V \otimes \L$ is \[\ThomSheaf{V_{reg} \otimes V \otimes \L} \cong \bigotimes_{a \in A} T_a^* \sheaf I(D^{-1}) \cong \sheaf I \left( \sum_{a \in A} T_a^* D^{-1} \right).\]  In particular, \[\ThomSheaf{V_{reg} \otimes \L} \cong \bigotimes_{a \in A} T_a \sheaf I(0) \cong \sheaf I(\chi).\]
% \item Suppose that the map \[\widetilde \chi\co (BA^*)_E \to \InternalHom{FormalGroups}(A, \G)\] is an isomorphism.  Given a level structure and cokernel pair \[A_T \xrightarrow{\ell} i^* \G \xrightarrow{q} \G',\] changing base along $T \times \G \xrightarrow{\chi_\ell} \InternalHom{FormalGroups}(A, \G) \times \G$ gives \[\chi_\ell^* \ThomSheaf{V_{reg} \otimes \L} \cong q^* N_q \sheaf I_{\G}(0) \cong q^* \sheaf I_{\G'}(0) \cong \sheaf I_{\G}(\ell).\]
% \item Restricting the above example to $BA^*$, we find \[\chi_\ell^* \ThomSheaf{V_{reg}} = 0_{\G}^* q^* \sheaf I_{\G'}(0) = 0_{\G'}^* \sheaf I_{\G'}(0) = \omega_{\G'}.\]
% \end{enumerate}

% \todo{Eqn 8.5 is nice and should appear in the Thom Sheaves section.  Item (8) in Section 8 has a nice interpretation of the splitting principle and lends some teeth to the phrase ``complex oriented descent''.}

% \begin{definition}
% For a divisor $D$ on a curve $C$ over $X$ with pointing $\zeta: X \to C$, set the \textit{Thom sheaf} of $D$ to be the line bundle $L(D) = \zeta^* J(D)$ over $X$, with $J(D)$ the ideal sheaf on $C$ defining $D$.  It satisfies $L(D + D') = L(D) \otimes L(D')$.  If $C$ has a coordinate $x$, then $J(D)$ has a generator $f_D(x)$ and $L(D)$ also has a generator $u_D$, called the \textit{Thom class}.  It satisfies $u_{D + D'} = u_D \otimes u_{D'}$.
% \end{definition}

\todo[inline]{It's not clear to me what theorems about level structures and so forth are best included on this day and which belong back in the lecture above.  We should be able to split things apart into stuff desired for character theory and stuff desired for descent.}

Ando's Theorem 3.4.4: Let $D_j$ be the ring extension of $E_n$ which trivializes the $p^j$--torsion subgroup of $\G_{E_n}$.  Let $H$ be a finite subgroup of $\G_{E_n}(D_k)$.  There is an unstable transformation of ring-valued functors \[E_n X \xrightarrow{\Psi^H} D_j \otimes E_n X,\] and if $F$ is an Ando coordinate then for any line bundle $\L \to X$ there is a formula \[\psi^H(e\L) = \prod_{h \in H} (h +_F e\L) \in D_j \otimes E_n(X).\]

$D_j$ is Galois over $E_n$ with Galois group $\GL_n(\Z/p^j)$.  If $\rho$ is a collection of finite subgroups weighted by elements of $E_n$ which is stable under the action of the Galois group, then $\Psi^\rho$ descends to take values in just $E_n$.  (For example, the entire subgroup has this property.)

This is built by a character map.  Take $H \subseteq F(D_j)[p^j]$ to be a finite subgroup again; then there is a map \[\chi^H: E_n(D_{H^*} X) \to D_j \otimes E_n(X),\] where $D_{H^*}$ denotes the extended power construction on $X$ using the Pontryagin dual of $H$.  This composes to give an operation \[Q^H: MU^{2*}(X) \xrightarrow{P_{H^*}} MU^{2|H|*}(D_{H^*} X) \to E_n^{2|H|*}(D_{H^*} X) \xrightarrow{\chi^H} D_j \otimes E_n^{2|H|*}(X).\]  Then $Q^H$ is a ring homomorphism with effects
\begin{align*}
Q^H F^{MU} & = F/H, &
Q^H(e_{MU} \L) & = \prod_{h \in H} h +_F e\L.
\end{align*}

Then we need to factor $Q^H: MU(X) \to D_j \otimes E_n(X)$ across the orienting map $MU \to E_n$.  Since $E_n$ is Landweber flat and $Q^H$ is a ring map, it suffices to do this for the one--point space, i.e., to construct a ring homomorphism \[\Psi^H: E_n \to D_j\] so that $\Psi^H = \Psi^H(*) \otimes Q^H$.  The first condition above then translates to $\Psi^H F^{MU} = F/H$.

\begin{theorem}\citeme{Theorem 2.5.7 of Matt's thesis}
For each $\star$--isomorphism class of lift $F$ of $\Phi$ to $E_n$, there is a unique choice of coordinate $x$ on $F$, lifting the preferred coordinate on $\Phi$, such that $\alpha^H_* F_x = F_x/H$, or equivalently that $l_H^x = f_H^x$, for all finite subgroups $H$.  (These morphisms are arranged in the following diagram:)
\begin{center}
\begin{tikzcd}
F_x \arrow{r}{f_H^x} \arrow[bend left]{rr}{l_H^x} & F_x / H \arrow{r}{g_H^x} & \alpha^H_* F_x \\
F \arrow{u}{x} \arrow{r} & F/H \arrow{u}{x_H} \arrow{r}{g_H} & \alpha^H_* F \arrow{u}{\alpha_* x} ,
\end{tikzcd}
\end{center}
where $\alpha_H: E_n \to D_k$ is the unique ring homomorphism such that there is a $\star$--isomorphism $g_H: F/H \to \alpha^H_* F$. \qed \todo{This is some serious work, and I don't think we'll prove it.  The main point is that $\alpha^p_* F_x = F_x / p$ can be reimagined as $f_p^x(t) = [p]_{F_x}(t)$, and this already is enough to determine what $x$ is by descending along the power of the maximal ideal in $E_n$, the length of a full level structure, and pieces of a smaller level structure inside of the full one.  It really is a long argument.}
\end{theorem}

\todo[inline]{Section 2.7 of Matt's thesis works the example of a normalized coordinate for $\G_m$. It's \emph{not} the $p$--typical coordinate. It \emph{is} the standard one! Cool.}

\begin{lemma}\citeme{Lemma 3.2.7 of Matt's thesis, BMMS86 page 25, AHS $H_\infty$ appendix}
$P_r(x + y) = \sum_{j=0}^r \operatorname{Tr}_{j,r}^{MU} d^*(P_j x \times P_{r-j}y)$.

\todo[inline]{This expresses the non-additivity of the power operations on $MU$. It's apparently needed in the proof that $Q^H$ acts as it should on Euler classes. It involves transfer formulas, which may mean we need to work that section of HKR into that day.}
\end{lemma}
\begin{proof}
Represent $x$ and $y$ by maps
\begin{align*}
U & \xrightarrow{f} X, &
V & \xrightarrow{g} Y.
\end{align*}
Then $P_r(x+y)$ is represented by \[D_r(U \sqcup V) \xrightarrow{D_r(f \sqcup g)} D_r X.\]  There is a decomposition \[D_r(U \sqcup V) = \coprod_{j=0}^r E\Sigma_r \times_{\Sigma_j \times \Sigma_{r-j}} (U^j \times V^{r-j}),\] and on the $j$ factor the map $D_r(f \sqcup g)$ restricts to
\begin{center}
\begin{tikzcd}
E\Sigma_r \times_{\Sigma_j \times \Sigma_{r-j}} (U^j \times V^{r-j}) \arrow{r}{E\Sigma_r \times_{\Sigma_j \times \Sigma_{r-j}} (f^j \times g^{r-j})} \arrow{d} & E\Sigma_r \times_{\Sigma_j \times \Sigma_{r-j}} X^r \arrow{d} \\
D_r(U \sqcup V) \arrow{r}{D_r(f \sqcup g)} & D_r X,
\end{tikzcd}
\end{center}
where the vertical maps are projections.  The counterclockwise composite represents the $j$ summand of $P_r(x+y)$ coming from the decomposition above; the clockwise composite represents the class $\operatorname{Tr}_{j,r}^{MU} d^*(P_j x \times P_{r-j} y)$. \qedhere
\end{proof}

\todo{There's also this useful naturality Lemma for power operations and Euler classes: $P_\pi(eV) = e(D_\pi V \to D_\pi X)$.  Does that come up in the Quillen chapter?  Maybe it should.}

\begin{lemma}\citeme{Prop 3.2.10 of Matt's thesis, see also p.\ 42 of Quillen}
Write $\Delta: B\pi \times X \to D_\pi X$ and let $\L$ be a complex line bundle on $X$.
\[\Delta^* P_\pi(e\L) = \prod_{u \in \pi^*} \left( e\left(\begin{array}{c} E\pi \times_u \C \\ \downarrow \\ B\pi \end{array} \right) +_{MU} e(\L) \right).\]
\end{lemma}

\needproof{Matt claims that 3.2.10, the above Lemma, is the beating heart of the paper.  Look how similar it looks to the formal group law quotient formula!  That's why an expanded formula must be included in the previous days, not just Neil's geometric scribblings}

\todo[color=red,inline]{Matt in and before Theorem 3.3.2 describes the ring $D_k$ as the \emph{image} of the localization map $E_n(B\Lambda_k) \to S^{-1} E_n(B\Lambda_k)$ rather than as the whole target.  Why??  He cites HKR for this, but the citation is meaningless because the theorem numbering scheme is so old.  Ah, comparing with Lemma 3.3.3 yields a clue: $D_k$ has a universal property as it sits under $E_n$, rather than under $E_n(B\Lambda_k)$...}

Now, suppose that we pass down to the $k${\th} Drinfel'd ring, so that the $p^k$--torsion in the formal group is presented as a discrete group $\Lambda^*[p^k]$.  Pick such a subgroup $H \subseteq \Lambda^*[p^k]$ with $|H| = r$, and consider also the dual map $\pi: \Lambda[p^k] \to H^*$.  We define the character map associated to $H$ to be the composite \[\chi^H \co E_n(D_{H^*} X) \xrightarrow{\Delta^*} E_n(BH^*) \otimes_{E_n} E_n(X) \xrightarrow{\chi_\pi \otimes 1} D_k \otimes_{E_n} E_n(X) =: D_k(X).\]  This definition is set up so that \[\chi^H \left( e \left( \begin{array}{c} EH^* \times_u \C \\ \downarrow \\ BH^* \end{array} \right) \right).\]  In the presence of a coordinate $x$, this sews together to give a cohomology operation:
\begin{align*}
Q^H \co MU^{2*}(X) & \xrightarrow{P^{MU}_G} MU^{2r*}(D_{H^*} X) \\
& \xrightarrow{\Delta^*} MU^{2r*}(BH^* \times X) \\
& \xrightarrow{t_x} E_n(BH^* \times X) \\
& \xrightarrow{\simeq} E_n BH^* \otimes_{E_n} E_n X \\
& \xrightarrow{\chi^H \otimes 1} D_k X.
\end{align*}
It turns out that $Q^H$ is a ring homomorphism (cf.\ careful manipulation of HKR's Theorem C, which may not be worth it to write out, but it seems like the main manipulation is the last line of Proof of Theorem 3.3.8 on pg.\ 466), so each choice of $H$ (and $x$) determines a new coordinate on $D_k$.
\begin{theorem}
The effect of $Q^H$ on Euler classes is \[Q^H e_{MU} \L = f_H^x e_x \L \in D_k(X),\] and its effect on coefficients is \[Q^H_* F_{MU} = F_x / H.\]
\end{theorem}
\begin{proof}
We chase through results established so far:
\begin{align*}
Q^H(e_{MU} \L) & = (\chi^H \otimes 1) \circ t_x \circ \Delta^* \circ P^{MU}_G (e_{MU} \L) \\
& = (\chi^H \otimes 1) \circ t_x \left( \prod_{u \in H^** = H} e_{MU} \left( \begin{array}{c} EH^* \times_u \C \\ \downarrow \\ BH^* \end{array} \right) +_{MU} e_{MU} \L \right) \\
& = (\chi^H \otimes 1) \left( \prod_{u \in H} e_{E_n} \left( \begin{array}{c} EH^* \times_u \C \\ \downarrow \\ BH^* \end{array} \right) +_{F_x} e_{E_n} \L \right) \\
& = \prod_{u \in H} (\phi_{univ}(u) +_{F_x} e_{E_n} \L) = f_H^x(e_{E_n} \L).
\end{align*}
Then, ``since $D_k$ is a domain, $F_x/H$ is completely determined by the functional equation'' \[f_H^x(F_x(t_1, t_2)) = F_x/H(f_H^x(t_1, f_H^x(t_2))).\]  Take $t_1$ and $t_2$ to be the Euler classes of the two tautological bundles $\L_1$ and $\L_2$ over $\CP^\infty \times \CP^\infty$, so that
\begin{align*}
Q^H(e_{MU} \L_1 +_{MU} e_{MU} \L_2) & = Q^H\left(e_{MU} \left( \begin{array}{c} \L_1 \otimes \L_2 \\ \downarrow \\ \CP^\infty \times \CP^\infty \end{array} \right)\right) \\
& = f_H^x \left(e_{E_n} \left( \begin{array}{c} \L_1 \otimes \L_2 \\ \downarrow \\ \CP^\infty \times \CP^\infty \end{array} \right)\right) = f_H^x(t_1 +_{F_x} t_2).
\end{align*}
On the other hand, $Q^H$ is a ring homomorphism, so we can also split it over the sum first:
\begin{align*}
Q^H(e_{MU} \L_1 +_{MU} e_{MU} \L_2) & = Q^H(e_{MU} \L_1) +_{Q^H_* F^{MU}} Q^H(e_{MU} \L_2) \\
& = f_H^x(t_1) +_{Q^H_* F^{MU}} f_H^x(t_2),
\end{align*}
hence $f_H^x(t_1) +_{Q^H_* F^{MU}} f_H^x(t_2) = f_H^x(t_1 +_{F_x} t_2)$ and $Q^H_* F^{MU} = F_x / H$.
\end{proof}

Finally, we would like to produce a factorization \[MU \xrightarrow{\Psi^H} E_n \to D_k\] of the long natural transformation $Q^H$.  Since $E_n$ was built by Landweber flatness, it suffices to do this on coefficient rings, i.e., when applying the functors in the diagram to the one-point space.  On a point, our calculations above show that $\Psi^H$ exists exactly when $\alpha^H_* F_x = F_x / H$.  We did this algebraic calculation earlier: given any coordinate, there is a unique coordinate $P$ that is $\star$--isomorphic to it and through which the operations $Q^H$ factor to give ring operations $\Psi^H$ for all subgroups $H \subseteq \Lambda_k^* = F_P(D_k)[p^k]$.  This solves the problem of giving the operations the right \emph{source}.

\todo[inline]{Leave a remark in here about this: McClure in BMMS works along similar lines to show that the Quillen idempotent is not $H_\infty$, but he doesn't get any positive results (and, in particular, he can't complete his analysis as we do because he doesn't have access to the $BP$--homology of finite groups and to HKR character theory).  One wonders whether the stuff here does say something about $BP$ as the height tends toward $\infty$.  So far as I know, no one has written much about this.  Surely it remains a bee in Matt's bonnet.}

Now we focus on giving the operations the right \emph{target}.  This is considerably easier.  The group $\Aut(\Lambda_k^*)$ acts on the set of subgroups of $\Lambda_k^*$, and we define a ring $Op^k$ by the fixed points of $\Aut(\Lambda_k^*)$ acting on the polynomial ring $E_n[\text{subgroups of $\Lambda_k^*$}]$.  Note that $Op^k \subseteq Op^{k+1}$, and define $Op = \colim_k Op^k$, which consists of elements $\rho = \sum_{i \in I} a_i \prod_{H \in \alpha_i} H$, $I$ a finite set, $a_i \in E_n$, and $\alpha_i$ are certain $\Aut(\Lambda_k^*)$--stable lists of subgroups of $\Lambda_k^*$, $k \gg 0$, with possible repetitions.  For such a $\rho$, we define the associated operations
\begin{align*}
Q^\rho \co MU^{2*}(X) & \xrightarrow{\sum_{i \in I} a_i \prod_{H \in \alpha_i} Q^H} D_k(X), \\
\Psi^\rho \co E_n(X) & \xrightarrow{\sum_{i \in I} a_i \prod_{H \in \alpha_i} \Psi^H} D_k(X).
\end{align*}
The theorem is that these actually land in $E_n(X)$, as they definitely land in $D_k^{\Aut(\Lambda_k^*)} \otimes_{E_n} E_n(X)$, and Galois descent for level structures says that left--hand factor is just $E_n$.

\todo{Matt runs the example of the subgroups $\G_m[p^j]$ in $p$--adic $K$--theory and he compares it with some Hopf ring analysis of $E_n \OS{E_n}{*}$ due to Wilson}





\section{The moduli of subgroup divisors}

Following... the original? Following Nat?

Continuing on from the above, if we expected $E_n$ to be $E_\infty$ (or even $H_\infty$) so that it had power operations, then we would want to understand $E_n B\Sigma_{p^j}$ and match that with the operations we see.

---

There are union maps \[B\Sigma_j \times B\Sigma_k \to B\Sigma_{j+k},\] stable transfer maps \[B\Sigma_{j+k} \to B\Sigma_j \times B\Sigma_k,\] and diagonal maps \[B\Sigma_j \to B\Sigma_j \times B\Sigma_j.\]  These induce a coproduct $\psi$ as well as products $\times$ and $\bullet$ on $E^0 \P \S^0$, where $\P\S^0 = \coprod_{j=0}^\infty B\Sigma_j$ is the free $E_\infty$--ring on $\S^0$.  This is a Hopf ring, and under $\times$ alone it is a formal power series ring.  The $\times$--indecomposables (which, I guess, are analogues of considering additive unstable cooperations) are \[Q^\times E^0 \P\S^0 = \prod_{k \ge 0} \left( E^0 B\Sigma_{p^k} / \operatorname{tr} E^0 B\Sigma_{p^{k-1}}^p \right),\] where the $k${\th} factor in the product is naturally isomorphic to $\sheaf{O}_{\Sub_{p^k}(\G)}$.  The primitives are also accessible as the kernel of the dual restriction map.

Theorem 3.2 shows that $E^0 B\Sigma_k$ is free over $E^0$, Noetherian, and of rank controlled by generalized binomial coefficients.  Prop 3.4 is the only place where work gets done, and it's all in terms of $K$--theory and HKR characters.

There's actually an extra coproduct, coming from applying $D$ to the fold map $S^0 \vee S^0 \to S^0$.

The main content of Prop 5.1 (due to Kashiwabara) is that $K_0 \P \S^0$ injects into $K_0 \OS{BP}{0}$.  Grading $K_0 \P \S^0$ using the $k$--index in $B\Sigma_k$, you can see that it's of graded finite type, so we need only know it has no nilpotent elements to see that $K_0 \P \S^0$ is $\ast$--polynomial.  This follows from our computation that $K_0 \OS{BP}{0}$ is a tensor of power series and Laurent series rings.  Corollary 5.2 is about $K_0 Q S^0$, which is the group completion of $K_0 \P \S^0$, so it's the tensor of $K_0 \P \S^0$ with a graded field.

Prop 5.6, using a double bar spectral sequence method, shows that $K^0 Q S^2$ is a formal power series algebra.  Tracking the spectral sequences through, you'll find that $Q^\times K^0 Q S^0$ agrees with $P K^0 Q S^2$.  (You'll also notice that $K^0 Q S^2$ only has one product on it, cf.\ Remark 5.4.)

Snaith's theorem says $\Sigma^\infty QX = \Sigma^\infty \P X$ for connected spaces $X$.  You can also see (just after Theorem 6.2) the nice equivalences \[\P_k S^2 \simeq B\Sigma_k^{V_k} \simeq \P_k(S^0)^{V_k},\] where superscript denotes Thom complex.  So, for a complex-orientable cohomology theory, you can learn about $\P_k S^0$ from $\P_k S^2$.  In particular, we finally learn that $E^0 \P S^0$ is a formal power series $\times$--algebra (once checking that the Thom isomorphism is a ring map).  (We already knew the homological version of this claim.)

Section 8 has a nice discussion about indecomposables and primitives, to help move back and forth between homology and cohomology.  It probably helps most with the dimension count argument below that we aren't going to get into.

Start again with $D_{p^k} S^2 \simeq B\Sigma_{p^k}^{V_{p^k}}$.  We can associate to this a divisor $\D(V_{p^k})$ on $(B\Sigma_{p^k})_E$, which we know little about, but it is classified by a map to $\Div_{p^k} \CP^\infty_E$.  This receives a closed inclusion from $\Sub_{p^k} \CP^\infty_E$, so their pullback $Z_k$ is the largest subscheme of $(B\Sigma_{p^k})_E$ over which $\D(V_{p^k})$ is a subgroup divisor.
\begin{center}
\begin{tikzcd}
H_k \arrow{rr} \arrow{dd} & & \D(V_{p^k}) \\
& Z_k \arrow{rr} \arrow{rd} & & \Sub_{p^k} \CP^\infty_E \arrow{rd} \\
\Spf E^0 B\Sigma_{p^k} / \mathrm{tr} \arrow{rr} \arrow[densely dotted]{ru} & & (B\Sigma_{p^k})_E \arrow{rr} \arrow[crossing over,leftarrow]{uu} & & \Div_{p^k} \CP^\infty_E
\end{tikzcd}
\end{center}
We will show the existence of the dashed map, implying that the restricted divisor $H_k$ is a subgroup divisor on $Y_k = \Spf E^0 B\Sigma_{p^k} / \mathrm{tr}$.

(Prop 9.1:) This proof falls into two parts: first we construct a family of maps to $(B\Sigma_{p^k})_E$ on whose image $\D(V_{p^k})$ restricts to a subgroup divisor, and then we show that the union of their images is exactly $Y_k$.  Let $A$ be an abelian $p$--subgroup of $\Sigma_{p^k}$ that acts transitively on $\{1, \ldots, p^k\}$ (i.e., it is not boosted from some transfer).  The restriction of $V_{p^k}$ to $A$ is the regular representation, which splits as a sum of characters $V_{p^k}|_A = \bigoplus_{\L \in A^*} \L$.  Identifying $BA_E = \InternalHom{FormalGroups}(A^*, \CP^\infty_E)$, $\D(V_{p^k})$ restricts all the way to $\sum_{\L \in A^*} [\phi(\L)]$, with $\phi: A^* \to $``$\Gamma(\operatorname{Hom}(A^*, \G), \G)$''.  In Finite Subgroups of Formal Groups (see Props 22 and 32), we learned that the restriction of $\D(V_{p^k})$ further to $\Level(A^*, \CP^\infty_E)$ is a subgroup divisor.  So, our collection of maps are those of the form \[\Level(A^*, \CP^\infty_E) \to \InternalHom{FormalGroups}(A^*, \CP^\infty_E) = BA_E \to (B\Sigma_{p^k})_E.\]  Here, finally, is where we have to do some real work involving Chern classes and commutative algebra, so I'm inclined to skip it in the lectures.  Finally, you do a dimension count to see that $Z_k$ and $\Spf E^0 B\Sigma_{p^k} / \mathrm{tr}$ have the same dimension (which requires checking enough commutative algebra to see that ``dimension'' even makes sense), and so you show the map is injective and you're done.


-----

Here's Neil's proof of the joint images claim.  It seems like a clear enough use of character theory that we should include it, if we can make character theory itself clear.

Recall from [18, Theorem 23] that $\Level(A^*,\G)$ is a smooth scheme, and thus that $D(A) = \sheaf O_{\Level(A^*,\G)}$ is an integral domain. Using [18, Proposition 26], we see that when $\L \in A^*$ is nontrivial, we have $\phi(\L) \ne 0$ as sections of $\G$ over $\Level(A^*, \G)$, and thus $e(\L) = x(\phi(\L)) \ne 0$ in $D(A)$. It follows that that $c_{p^k} = \prod_{\L \ne 1} e(\L)$ is not a zero-divisor in $D(A)$. On the other hand, if $A'$ is an Abelian $p$-subgroup of $\Sigma_{p^k}$ which does not act transitively on $\{1, \ldots, p^k\}$, then the restriction of $V_{p^k} − 1$ to $A'$ has a trivial summand, and thus $c_{p^k}$ maps to zero in $D(A')$. Next, we recall the version of generalised character theory described in [8, Appendix A].
\[p^{-1} E^0 BG = \left(\prod_A p^{-1} D(A)\right)^G\]
where $A$ runs over all Abelian $p$-subgroups of $G$. As $\overline R_k = E^0(B\Sigma_{p^k} )/ ann(c_{p^k} )$ and everything in sight is torsion-free, we see that $p^{−1} \overline R_k$ is the quotient of $p^{−1}E^0B\Sigma_{p^k}$ by the annihilator of the image of $c_{p^k}$ . Using our analysis of the images of $c_{p^k}$ in the rings $D(A)$, we conclude that
\[p^{-1} \overline R_k = \left(\prod_A p^{−1}D(A)\right)^{\Sigma_{p^k}},\]
where the product is now over all transitive Abelian $p$-subgroups. This implies that for such $A$, the map $E^0B\Sigma_{p^k} \to D(A)$ factors through $\overline R_k$, and that the resulting maps $\overline R_k \to D(A)$ are jointly injective. This means that $Y_k = \Spf \overline R_k$ is the union of the images of the corresponding schemes $\Level(A^*,\G)$, as required.








\section{Interaction with $\Theta$--structures}

The Ando--Hopkins--Strickland result that the $\sigma$--orientation is an $H_\infty$--map

The main classical point is that an $MU\<0\>$--orientation is $H_\infty$ when the following diagram commutes for every choice of $A$:
\begin{center}
\begin{tikzcd}
(BA^* \times \CP^\infty)^{V_{reg} \otimes \L} \arrow{r} & D_n MU\<0\> \arrow{d} \arrow{r} & D_n E \arrow{d} \\
& MU\<0\> \arrow{r} & E
\end{tikzcd}
\end{center}
(This is equivalent to the condition given in the section on Matt's thesis.  In fact, maybe I should try writing this so that Matt's thesis uses the same language?)  If you write out what this means, you'll see that a given coordinate on $E$ pulls back to give two elements in the $E$--cohomology of that Thom spectrum (or: sections of the Thom sheaf), and the orientation is $H_\infty$ when they coincide.

Similarly, an $MU\<6\>$--orientation corresponds to a section of the sheaf of cubical structures on a certain Thom sheaf.  Using the $H_\infty$ structures on $MU\<6\>$ and on $E$ give two sections of the pulled back sheaf of cubical structures, and the $H_\infty$ condition is that they agree for all choices of group $A$.

Then you also need to check that the $\sigma$--orientation actually satisfies this.

\todo[inline]{The AHS document really restrictions attention to $E_2$.  Is there a version of this story that gives non-supersingular orientations too, or even the $K_{\Tate}$ orientation?  I can't tell if the restriction in AHS's exposition comes from not knowing that $K_{\Tate}$ has an $E_\infty$ structure or if it comes from a restriction on the formal group.  (At one point it looks like they only need to know that $p$ is regular on $\pi_0 E$, cf.\ 16.5...)}

Section 3.1: Intrinsic description of the isogenies story for an $H_\infty$ \emph{complex orientable} ring spectrum, without mention of a specific orientation / coordinate.  This is nice: it means that a complex orientation has to be a coordinate which is compatible with the descent picture already extant on the level of formal groups, which is indeed the conclusion of Matt's thesis.

Section 3.2: They define an abelian group indexed extended power construction \[D_A(X) = \L(U^{A^*}, U) \sm_{A^*} X^{(A^*)},\] where $\L(U^{A^*}, U)$ is the space of linear isometries from the $A^*${\th} power of a universe $U$ down to itself.  Yuck.  Then, given a level structure $(i\co \Spf R \to S_E, \ell\co A_{\Spf R} \to i^* \G)$, they construct a map \[\psi^E_\ell \co \pi_0 E \xrightarrow{D_A} \pi_0 \CatOf{Spectra}(D_A S^0, E) = \pi_0 E^{BA^*_+} \to \sheaf{O}((BA^*)_E) \xrightarrow{\chi_\ell} R,\] where $\chi_\ell$ is the map classifying the homomorphism $\ell$.  This is a continuous map of rings: it's clearly multiplicative, it's additive up to transfers (but those vanish for an abelian group), and it's continuous by an argument in Lemma 3.10.  (You don't actually need an abelian group here; you can work in the scheme of subgroups --- i.e., in the cohomology of $B\Sigma_k$ modulo transfers --- and this will still work.)  This construction is natural in $H_\infty$ maps $f\co E \to F$:
\begin{center}
\begin{tikzcd}
i^* S_F \arrow[bend left]{rrd}{\psi_\ell^F} \arrow[densely dotted]{rd}{\psi_\ell^{F/E}} \arrow[bend right]{rdd} \\
& \Spf R \times_{i, S_E, S_f} S_F \arrow{r}{\psi_\ell^F} \arrow{d} & S_F \arrow{d}{S_f} \\
& \Spf R \arrow{r}{\psi_\ell^E} & S_E,
\end{tikzcd}
\end{center}
begetting the relative map $\psi_\ell^{F/E}\co i^* S_F \to (\psi_\ell^E)^* S_F$ as indicated.  For example, take $F = E^{\CP^\infty_+}$, so that $\G = S_F$, giving the (group) map \[\psi_\ell^{\G/E}\co i^* \G \to (\psi_\ell^E)^* \G.\]  One of the immediate goals is to show that this is an isogeny.  A different construction we can do is take $V$ to be a virtual bundle over $X$ and set $F = E^{X_+}$.  Given $m \in \pi_0 \CatOf{Spectra}(X^V, E)$ applying the construction of $D_A$ above gives an element \[\psi_\ell^V(m) \in R \underset{\chi_\ell, \hat\pi_0 E^{BA^*_+}}{\widehat{\otimes}} \hat\pi_0 \CatOf{Spectra}((BA^* \times X)^{V_{reg} \otimes V}, E).\]  This map is additive and also $\psi_\ell^V(xm) = \psi_\ell^F(x) \psi_\ell^V(m)$, so we can interpret this as a map \[\psi_\ell^V\co (\psi_\ell^F)^* \ThomSheaf{V} \to \chi_\ell^*\ThomSheaf{V_{reg} \otimes V}\] of line bundles over $i^* S_F = i^* X_E$.

\begin{lemma}\citeme{Lemma 3.19 of AHS $H_\infty$}
The map $\psi_\ell^V$ has the following properties:
\begin{enumerate}
\item If $m$ trivializes $\ThomSheaf{V}$ then $\psi_\ell^V(m)$ trivializes $\chi_\ell^* \ThomSheaf{V_{reg} \otimes V}$.
\item $\psi_\ell^{V_1 \oplus V_2} = \psi_\ell^{V_1} \otimes \psi_\ell^{V_2}$.
\item For $f\co Y \to X$ a map, $\psi_\ell^{f^* V} = f^* \psi_\ell^V$. \qed 
\end{enumerate}
\end{lemma}

In particular, we can apply this to $X = \CP^\infty$ and $\ThomSheaf{\L - 1} = \sheaf I(0)$.  Then 8.11 gives \[\psi_\ell^{\L - 1} \co (\psi_\ell^F)^* \sheaf I_{\G}(0) \to \chi_\ell^* \ThomSheaf{V_{reg} \otimes (\L - 1)} = \sheaf I_{i^* \G}(\ell).\]

\begin{theorem}\citeme{Prop 3.21}
The map $\psi_\ell^{\G/E} \co i^* \G \to (\psi_\ell^E)^* \G$ of 3.15 is an isogeny with kernel $[\ell(A)]$.  Using $\psi_\ell^{\G/E}$ to make the identification \[(\psi_\ell^{\G/E})^* \sheaf I_{(\psi_\ell^E)^* \G}(0) \cong \sheaf I_{i^* \G}(\ell),\] the map $\psi_\ell^{\L - 1}$ sends a coordinate $x$ on $\G$ to the trivialization $(\psi_\ell^{\G/E})^* (\psi_\ell^E)^* x$ of $\sheaf I_{i^*\G}(\ell)$. \qed
\end{theorem}

3.24 might be interesting.

\todo[inline]{So far, it seems like the point is that the identity map on $MU\<0\>$ classifies a section of the ideal sheaf at zero of the universal formal group which is compatible with descent for level structures, so any $H_\infty$ map out of $MU\<0\>$ classifies not just a section of the ideal sheaf at zero of whatever other formal group but does so in a way that is, again, compatible with descent for level structures.}

\begin{theorem}\citeme{Prop 4.13} \todo{The discussion leading up to this theorem seems interesting, especially equations 4.10,12.}
Let $g: MU\<0\> \to E$ be a homotopy multiplicative map, and let $s = s_g$ be the corresponding trivialization of $\sheaf I_{\G}(0)$.  If the map $g$ is $H_\infty$, then for any level structure $\ell: A \to i^* \G$ the section $s$ satisfies the identity \[N_{\psi_\ell^{\G/E}} i^* s = (\psi_\ell^E)^* s,\] in which the isogeny $\psi_\ell^{\G/E}$ has been used to make the identification \[N_{\psi_\ell^{\G/E}} i^* \sheaf I_{\G}(0) \cong \sheaf I_{(\psi_\ell^E)^* \G}(0). \qed\]
\end{theorem}

\begin{lemma}\citeme{Eqn 5.3, generalizes Quillen's splitting formula}
For $V$ a vector bundle on a space $X$ and $V_{reg}$ the (vector bundle over $BA^*$ induced from) the regular representation on $A$, there is an isomorphism of sheaves over $(BA^* \times X)_E$ \[\ThomSheaf{V_{reg} \otimes V} \cong \bigotimes_{a \in A} \widetilde T_a \ThomSheaf{V}.\]
\end{lemma}

Eqn 5.4 claims to use 5.3 but seems to be using something about the behavior of the norm map on line bundles vs the translated sum of divisors appearing in 5.3.

The beginning of the proof of 6.1 appears to be a simplification of some of the descent arguments appearing in the algebraic parts of Matt's thesis's main calculations.  On the other hand, I can't even read what the McClure reference in 6.1 is doing.  What's $\Delta^*$??

\begin{lemma}\citeme{Prop 7.5}
Take $\pi_0 E$ to be a complete local ring and $\G_E$ to be of finite height.  If $B^* \subset A^*$ is a proper subgroup, then the following composite map of $\pi_0 E$--modules is zero: \[\pi_0 E^{BB^*_+} \xrightarrow{transfer} \pi_0 E^{BA^*_+} \xrightarrow{\chi_\ell} \sheaf O(T).\]
\end{lemma}
\begin{proof}
It suffices to consider the tautological level structure over $\Level(A, \G)$.  We may take $A$ to be a $p$--group, and indeed for now we set $A = \Z/p$, $B = 0$.  For $t \in \pi_0 E^{\CP^\infty_+}$ a coordinate with formal group law $F$, we have \[\pi_0 E^{BA^*_+} \cong \pi_0 E \ps{t} / [p]_F(t)\] and $\tau: \pi_0 E^{BB^*_+} = \pi_0 E \to \pi_0 E^{BA^*_+}$ is given by $\tau(1) = \<p\>_F(t)$, where $\<p\>_F(t) = [p]_F(t) / t$ is the ``reduced $p$--series''.  The result then follows from the isomorphism $\sheaf O(\Level(\Z/p, \G_E)) \cong \pi_0 E\ps{t} / \<p\>_F(t)$.  The result then follows in general by induction: $B^*$ can be taken to be a \emph{maximal} proper subgroup of $A^*$, with cokernel $\Z/p$.
\end{proof}

\begin{example}
Let $\G_m$ be the formal multiplicative group with coordinate $x$ so that the group law is \[x +_{\G_m} y = x + y - xy, \quad [p](x) = 1 - (1 - x)^p.\]  The monomorphism $\Z/p \to \G_m(\Z\ps{y} / [p](y))$ given by $j \mapsto [j](y)$ becomes the zero map under the base change
\begin{align*}
\Z\ps{y} / [p](y) & \to \Z/p, \\
y & \mapsto 0.
\end{align*}
\end{example}

\begin{remark}\citeme{Corollary 9.21, Prop 9.17}
If $R$ is a domain of characteristic $0$, then a level structure over $R$ actually induces a monomorphism on points.
\end{remark}

\begin{lemma}\citeme{Prop 9.24} \todo{One of the reduction steps in Prop 6.1 is handled by 9.24, which is in turn equivalent to a basic case of an HKR theorem, so should be stated on that day (or in the algebraic day).}
The natural map \[\sheaf O(\InternalHom{FormalGroups}(\Z/p, \G)) \to R \times \sheaf O(\Level(\Z/p, \G))\] is injective.
\end{lemma}
\begin{proof}
\todo{Fill this.}
\end{proof}

\todo[color=red,inline]{I left off at Section 10.}


------ Descent along level structures, simplicially (Section 11) ------

\todo[inline]{Actually, this section appears \emph{not} to be about $\FGps$, and instead it's about the \emph{coarse moduli quotient} to the functor of formal groups, which is not locally representable.  I'm a little confused about this --- I intend to ask Mike what's going on.}

Write $\Level(A) \to \FGps$ for the parameter space of a formal group equipped with a level--$A$ structure, together with its structure map (to the \emph{coarse moduli of formal groups!!!}).  We define a sequence of schemes by: $\Level_0 = \FGps$, $\Level_1 = \coprod_{A_0} \Level(A_0)$ for finite abelian groups $A_0$, and most generally \[\Level_n = \coprod_{0 = A_n \subseteq \cdots \subseteq A_0} \Level(A_0).\]  There are two maps $\Level_1 \to \Level_0$.  One is the structural one, where we simply peel off the formal group and forget the level structure.  The other comes from the quotient map: $\ell\co A \to \G$ yields a quotient isogeny $q\co \G \to \G/\ell$, and we take the second map $\Level_1 \to \Level_0$ to send $\ell$ to $\G / \ell$.  Then, consider the following Lemma:

\begin{lemma}\citeme{AHS Lemma 11.3}
For $\ell\co A \to \G$ a level structure and $B \subseteq A$ a subgroup, the induced map $\ell|_B\co B \to \G$ is a level structure and the quotient $\G / \ell|_B$ receives a level structure $\ell'\co A/B \to \G/\ell|_B$. \qed
\end{lemma}

This gives us enough compatibility among quotients to use the two maps above to assemble the $\Level_*$ schemes into a simplicial object.  Most face maps just omit a subgroup, except for the last face map, since the zero subgroup is not permitted to be omitted.  Instead, the last face map sends the string of subgroups $0 = A_n \subseteq A_{n-1} \subseteq \cdots \subseteq A_0$ and level structure $\ell\co A_0 \to \G$ to the quotient string $0 = A_{n-1} / A_{n-1} \subseteq \cdots \subseteq A_0 / A_{n-1}$ and quotient level structure $\ell\co A_0 / A_{n-1} \to \G/\ell|_{A_{n-1}}$.  The degeneracy maps come from lengthening one of these strings by an identity inclusion.

\begin{definition}\citeme{Definition 11.10, Remark 11.11}
Let $\G\co F \to \FGps$ be a functor over formal groups, and define schemes $\Level(A, F) = \Level(A) \times_{\G} F$ and $\Level_n(F) = \Level_n \times_{\G} F$.  Then, \textit{descent data for level structures on $F$} is the structure of a simplicial scheme on $\Level_*(F)$, together with a morphism of simplicial schemes $\Level_*(F) \to \Level_*$.  It is enough to specify a map $d_1\co \Level_1(F) \to F$, use that to build the simplicial scheme structure as in the above Lemma, and assert that the following square commutes:
\begin{center}
\begin{tikzcd}
\Level_1(F) \arrow{r} \arrow{d}{d_1} & \Level_1 \arrow{d}{d_1} \\
F \arrow{r} & \FGps.
\end{tikzcd}
\end{center}
\end{definition}

\begin{example}
Let $\G\co S \to \FGps$ be a formal group of finite height over a $p$--local formal scheme $S$.  The functor $\Level(A, \G)$ is exactly the functor defined in Section 9 (see above), and in particular it is represented by an $S$--scheme.  The maps $\psi_\ell$ and $f_\ell$ from Definition 3.1 amount to giving a map $d_1\co \Level_1(\G) \to S$ and an isogeny $q\co d_0^* \G \to d_1^* \G$ whose kernel on $\Level(A, \G)$ is $A$.  The other conditions on Definition 3.1 exactly ensure that $(\Level_*(\G), d_*, s_*)$ is a simplicial functor and over $\Level_2(\G)$ the relevant hexagonal diagram commutes:
\begin{center}
\begin{tikzcd}
& d_0^* d_0^* \G \arrow[-,double]{ld} \arrow{rd}{d_0^* q} \\
d_1^* d_0^* \G \arrow{d}{d_1^* q} & & d_0^* d_1^* \G \arrow[-,double]{d} \\
d_1^* d_1^* \G \arrow[-,double]{rd} & & d_2^* d_0^* \G \arrow{ld}{d_2^* q} \\
& d_2^* d_1^* \G.
\end{tikzcd}
\end{center}
\end{example}

\begin{example}
We now further package this into a single object.  Let $\underline{\G}$ be the functor over $\FGps$ whose value on $R$ is the set of pullback diagrams
\begin{center}
\begin{tikzcd}
\G' \arrow{r}{f} \arrow{d} & \G \arrow{d} \\
\Spf R \arrow{r}{i} & S
\end{tikzcd}
\end{center}
such that the map $\G' \to i^* \G$ induced by $f$ is a homomorphism (hence isomorphism) of formal groups over $\Spf R$.  For a finite abelian group $A$, write $\Level(A, \underline{\G})(R)$ for the set of diagrams
\begin{center}
\begin{tikzcd}
A_{\Spf R} \arrow{r}{\ell} \arrow{rd} & \G' \arrow{r}{f} \arrow{d} & \G \arrow{d} \\
& \Spf R \arrow{r}{i} & S
\end{tikzcd}
\end{center}
where the square forms a point in $\underline{\G}(R)$ and $\ell$ is a level--$A$ structure.  Giving a map of functors $d_1\co \Level_1(\underline{\G}) \to \underline{\G}$ making the above square commute is to give a pullback diagram
\begin{center}
\begin{tikzcd}
\G / \ell \arrow{r} \arrow{d} & \G \arrow{d} \\
\Level_1(\G) \arrow{r} & S,
\end{tikzcd}
\end{center}
or equivalently a map of formal schemes $\Level_1(\G) \to S$ and an isogeny $q\co d_0^* \G d_1^* \G$ whose kernel on $\Level(A, \G)$ is $A$.  Therefore, descent data for level structures on the formal group $\G$ (in the sense of Section 3) are equivalent to descent data for level structures on the functor $\underline{\G}$.
\end{example}

------ Section 12: Descent for level structures on Lubin--Tate groups ------

Let $k$ be perfect of positive characteristic $p$, and let $\Gamma$ be a formal group of finite height over $k$.  Recall that this induces a relative Frobenius
\begin{center}
\begin{tikzcd}
\Gamma \arrow{r}{F} \arrow[bend left]{rr}{\phi_\Gamma} \arrow{rd} & \phi_k^* \Gamma \arrow{r} \arrow{d} & \Gamma \arrow{d} \\
& \Spec k \arrow{r}{\phi_k} & \Spec k.
\end{tikzcd}
\end{center}
The map $F$ is an isogeny of degree $p$, with kernel the divisor $p \cdot [0]$.  Recall also that a deformation $H$ of $\Gamma$ to $T$ induces a map $\underline{H} \to \Def(\Gamma)$, and there is a universal such $\G$ over the ground scheme $S \cong \Spf \W(k)\ps{u_1, \ldots, u_{d-1}}$ such that $\underline{\G} \to \Def(\Gamma)$ is an isomorphism of functors over $\FGps$.

Now consider a point in $\Level(A, \Def \Gamma)$:
\begin{center}
\begin{tikzcd}
A_T \arrow{r}{\ell} \arrow{rd} & H \arrow{d} & H_0 \arrow{l} \arrow{r}{f} \arrow{d} & \Gamma \arrow{d} \\
& T & T_0 \arrow{l} \arrow{r}{j} & \Spec k.
\end{tikzcd}
\end{center}
The level structure $\ell$ gives rise to a quotient isogeny $q\co H \to H'$.  Since $A$ is sent to $0$ in $\sheaf O_{T_0}$, there is a canonical map $\bar q$ fitting into the diagram
\begin{center}
\begin{tikzcd}
H \arrow{rr}{q} \arrow{rdd} & & H' \arrow{ldd} \\
& & & H_0 \arrow{rdd} \arrow[crossing over]{lllu} \arrow{r} & H_0' \arrow[crossing over]{llu} \arrow{dd} \arrow[densely dotted]{r}{\bar q} & (\phi^r)^* H_0 \arrow{ldd} \arrow{r} \arrow[leftarrow, bend right, crossing over]{ll} & H_0 \arrow{dd} \arrow{r}{f} & \Gamma \arrow{dd} \\
& T \\
& & & & T_0 \arrow{lllu} \arrow{rr}{\phi^r} & & T_0 \arrow{r}{j} & \Spec k.
\end{tikzcd}
\end{center}
The map $\bar q$ combines with the rest of the maps to exhibit $H'$ as a deformation of $\Gamma$, and hence we get a natural transformation \[d_1\co \Level_1(\Def(\Gamma)) \to \Def(\Gamma).\]  Since $\phi^r \phi^s = \phi^{r+s}$, this gives descent data for level structures on $\Def(\Gamma)$.  Identifying this functor with $\underline{\G}$ using Lubin--Tate theory, we equivalently have shown the existence of descent data for level structures on $\underline{\G}$.

Incidentally, the descent data constructed here is also the descent data that would come from the structure of an $E_\infty$--orientation on the Morava $E$--theory $E_d$, essentially because the divisor associated to the kernel of the relative Frobenius on the special fiber is forced to be $p[0]$, and everything is dictated by how the deformation theory \emph{has} to go (and the fact that the topological operations we're studying induce deformation-theoretic-describable operations on algebra).

------Section 15: Level structures on elliptic curves, and the relation to the $\sigma$--orientation / the corresponding section of the $\Theta^3$--sheaf------








\subsection*{Other stuff that goes in this chapter}

Dyer--Lashof operations, the Steenrod operations, and isogenies of the formal additive group \citeme{See Neil's \textit{Steenrod algebra} note, maybe? Talk to Mike?}

Another augmentation to the notion of a context: working not just with $E_* X$ but with $E_*(X \times BG)$ for finite $G$.

Charles's \textit{The congruence criterion} paper codifies the Hecke algebra picture Neil is talking about, and in particular it talks about sheaves over the pile of isogenies.

If we're going to talk about that Hecke algebra, then maybe we can also talk about the period map, since one of the main points of it is that it's equivariant for that action.

Section 3.7 of Matt's thesis also seems to deal with the context question: he gives a character-theoretic description of the total power operation, which ties the behavior of the total power operation to a formula of type ``decomposition into subgroups''.  Worth reading.

\todo{This is Nat's claim. Check back with him about how this is visible.}The rational story: start with a sheaf on the isogenies pile.  Tensor everything with $\Q$.  That turns this thing into a rational algebra under the Drinfel'd ring together with an equivariant action of $\GL_n \Q_p$.

Matt's Section 4 talks about the $E_\infty$ structure on $E_n$ and compatibility with his power operations.  It's not clear how this doesn't immediately follow from the stuff he proves in Section 3, but I think I'm just running out of stream in reading this thesis.  One of the neat features of this later section is that it relies on calculations in $E_n D_\pi \OS{MU}{2*}$, which is an interesting way to mix operations coming from instability and from an $H_\infty$--structure.  This is yet another clue about what the relevant picture of a context should look like.  He often cites VIII.7 of BMMS.

Mike says that Mahowald--Thompson analyzed $L_{K(n)} \Loops S^{2n+1}$ by writing down some clever finite resolution.  The resolution that they produce by hand is actually exactly what you would get if you tried to understand the mapping spectral sequence for $E_\infty(E_n^{\Loops S^{2n+1}}, E_n)$.

Mike also says that a consequence of the unpublished Hopkins--Lurie ambidexterity follow-up is that the comparison map $\CatOf{Spaces}(*, Y) \to E_\infty(E_n^Y, E_n^*)$ is an equivalence if $Y$ is a finite Postnikov tower living in the range of degrees visible to Morava $E$--theory.
