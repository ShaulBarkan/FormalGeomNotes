\documentclass{article}

\usepackage{amsmath,amssymb}

\usepackage{fullpage}
\usepackage[urw-garamond]{mathdesign}

\newcommand{\Z}{\mathbb Z}
\renewcommand{\S}{\mathbb S}
\newcommand{\F}{\mathbb F}
\newcommand{\G}{\mathbb G}
\newcommand{\R}{\mathbb R}
\newcommand{\RP}{\R\mathrm P}

\newcommand{\<}{\langle}
\renewcommand{\>}{\rangle}
\newcommand{\sm}{\wedge}

\newcommand{\context}[1]{\mathcal{M}_{#1}}
\newcommand{\moduli}[1]{\mathcal{M}_{\mathbf{#1}}}

\newcommand{\Spin}{\mathit{Spin}}
\newcommand{\String}{\mathit{String}}
\newcommand{\TMF}{\mathit{TMF}}
\newcommand{\tmf}{\mathit{tmf}}
\newcommand{\BP}{\mathit{BP}}
\newcommand{\MU}{\mathit{MU}}
\newcommand{\Tate}{\mathrm{Tate}}
\newcommand{\gl}{\mathit{gl}}
\newcommand{\GL}{\mathit{GL}}

\DeclareMathOperator{\Spec}{Spec}

\title{Formal Geometry in Algebraic Topology}
\author{Eric Peterson}
\date{Spring 2016, MWF 12pm--1pm}

\begin{document}

\maketitle

\section{Class information}

\subsection{Goals}

The primary goal of this class is to teach students to view results in algebraic topology through the lens of (formal) algebraic geometry.

\subsection{Grading}

This class won't have any official assignments. I'll give references as readings for those who would like a deeper understanding, though I'll do my best to ensure that no extra reading is required to follow the arc of the class.

I do want to assemble course notes from this class, but it's unlikely that I will have time to type \emph{all} of them up. Instead, I would like to ``crowdsource'' this somewhat: I'll type up skeletal notes for each lecture, and then we as a class will try to flesh them out as the semester progresses. As incentive to help, those who contribute to the document will have their name included in the acknowledgements, and those who contribute \emph{substantially} will have their name added as a coauthor. Everyone could use more CV items.

\section{Week-by-week}

\quad \quad \; \textsc{Overview}
\renewcommand{\labelenumi}{
\ifnum\value{enumi}=1%
  Jan 25\fi%
\ifnum\value{enumi}=2%
  Jan 27\fi%
\ifnum\value{enumi}=3%
Jan 29\fi%
\ifnum\value{enumi}=4%
Feb 1\fi%
\ifnum\value{enumi}=5%
Feb 3\fi%
\ifnum\value{enumi}=6%
Feb 5\fi%
\ifnum\value{enumi}=7%
Feb 8\fi%
\ifnum\value{enumi}=8%
Feb 10\fi%
\ifnum\value{enumi}=9%
Feb 12\fi%
\ifnum\value{enumi}=10%
Feb 17\fi%
\ifnum\value{enumi}=11%
Feb 19\fi%
\ifnum\value{enumi}=12%
Feb 22\fi%
\ifnum\value{enumi}=13%
Feb 24\fi%
\ifnum\value{enumi}=14%
Feb 26\fi%
\ifnum\value{enumi}=15%
Feb 28\fi%
\ifnum\value{enumi}=16%
Mar 2\fi%
\ifnum\value{enumi}=17%
Mar 4\fi%
\ifnum\value{enumi}=18%
Mar 7\fi%
\ifnum\value{enumi}=19%
Mar 9\fi%
\ifnum\value{enumi}=20%
Mar 11\fi%
\ifnum\value{enumi}=21%
Mar 21\fi%
\ifnum\value{enumi}=22%
Mar 23\fi%
\ifnum\value{enumi}=23%
Mar 25\fi%
\ifnum\value{enumi}=24%
Mar 28\fi%
\ifnum\value{enumi}=25%
Mar 30\fi%
\ifnum\value{enumi}=26%
Apr 1\fi%
\ifnum\value{enumi}=27%
Apr 4\fi%
\ifnum\value{enumi}=28%
Apr 6\fi%
\ifnum\value{enumi}=29%
Apr 8\fi%
\ifnum\value{enumi}=30%
Apr 11\fi%
\ifnum\value{enumi}=31%
Apr 13\fi%
\ifnum\value{enumi}=32%
Apr 15\fi%
\ifnum\value{enumi}=33%
Apr 18\fi%
\ifnum\value{enumi}=34%
Apr 20\fi%
\ifnum\value{enumi}=35%
Apr 22\fi%
\ifnum\value{enumi}=36%
Apr 25\fi%
\ifnum\value{enumi}=37%
Apr 27\fi%
\ifnum\value{enumi}=38%
Apr 29\fi%
\ifnum\value{enumi}=39%
May 2\fi%
\ifnum\value{enumi}=40%
May 4\fi%
\ifnum\value{enumi}>40%
PAST END\fi%
    :}
\begin{enumerate}
\item Overview of the class. (Orientations and theories of integration. Statement of the $\sigma$--orientation.)

\textsc{Case study: mod--$2$ homology}
\item Sheaves and formal schemes. The Steenrod algebra and $\context{H\F_2}$.
\item The mod--$2$ Adams spectral sequence. Sheaf cohomology. 
\item The sheaf $\context{H\F_2}(MO)$ and $\pi_* MO$.

\textsc{Introduction to the chromatic program}

\item Neil's $X_E$ construction for a general $E$. Formal schemes and formal groups. Basic theorems on formal varieties.
\item Simplicial presheaves, definition of the context. Homological and cohomological versions. Thom isomorphisms, and Quillen's theorem on $\context{\MU}$.
\item Structure theorems on $\moduli{fg}$. The picture. The definition of $K$-- and $E$--theories.
\item Group schemes and Hopf algebras. Finite dimensional Hopf algebras form an abelian category. Dieudonn\'e theory.
\item The periodicity and thick subcategory theorems. Bousfield localization, chromatic localizations and their properties, chromatic convergence.
\item $E(1)$--local homotopy of the sphere.

\textsc{The $\sigma$--orientation}

\item Thom spectra, line bundles, and divisors
\item The nonrigid, complex $\sigma$--orientation
\item Cohomological versions of AHS: $BU[2k, \infty)_E$.
\item The real version of the $\sigma$--orientation: $B\String_E$
\item Singer--Stong calculation of $H^* BU[2k, \infty)$.

\textsc{Power operations}

\item Ando, Hopkins, Strickland on $H_\infty$--orientations and the norm condition
\item The rigid, real $\sigma$--orientation: AHR. Its effect in homology.
\item The Rezk logarithm and the Bousfield--Kuhn functor
\item Statement of Lurie's characterization of $\TMF$, using this to determine a map from $M\String$ by AHR
\item Dylan's paper on String orientations
\item Matt's calculation of $E_\infty$--orientations of $K(1)$--local spectra using the short free resolution of $MU$ in the $K(1)$--local category

---------------------
\item Cartier duality
\item Subschemes and divisors
\item Coalgebraic formal schemes
\item \textit{Forms of $K$--theory}, Elliptic spectra, Tate $K$--theory, $\TMF$
\item The Ravenel--Wilson calculation, Weil pairings, Neil's MO answer about $H_* K(\Z, 3)$
\item $\sigma$ restricted to $K_{\Tate}$
\item What are $\Theta$--structures for geometers studying abelian varieties?
\item What are Weil pairings for geometers?
\item The Atiyah--Bott--Shapiro orientation (Is there a complex version of this? I understand it as a splitting of $M\Spin$...)
\item The HLP conjecture
\item Sinkinson's calculation and $M\BP\<m\>$--orientations
\item Hovey--Ravenel on nonorientations of $E_n$ by $MO[k, \infty)$. Other things in H--R?
\item Wood's cofiber sequence and $KO_{(p \ge 3)}$
\item The Serre--Tate theorem
\item The thick subcategory theorem.  Nilpotence and periodicity.
\item The chromatic spectral sequence, computations of $\pi_* L_{E(n)} \S$ for low $n$.
\item The fundamental domain of $\pi_{GH}$
\item Orientations and the functor $\gl_1$.
\end{enumerate}

\section{------------------------}



\section{Resources}

Ando, Hopkins, Strickland (Theorem of the Cube)

Ando, Hopkins, Strickland ($H_\infty$ map)

Ando, Strickland

Ando, Hopkins, Rezk

Barry Walker's thesis

Bill Singer's thesis, Bob Stong's \textit{Determination}

Hughes, Lau, Peterson

Morava's \textit{Forms of $K$--theory}

Neil's Functorial Philosophy for Formal Phenomena

Ravenel, Wilson

Kitchloo, Laures, Wilson


\section{$\context{H\F_2}(MO)$}

Hood made the following nice observation. $MO^{H\F_2}$ is the scheme of coordinates on $\RP^\infty_{H\F_2}$, with coordinate ring $\F_2[x_1, x_2, \ldots]$ and corresponding series $f(t) = \sum_{n=1}^\infty x_{n-1} t^n$ and $x_0 = 1$ implicit. This identification is equivalent to Adams's observation that $MO$ is the ``free homotopy ring spectrum'' on $MO(1)$ his sense. Then, $\Spec \mathcal{A}_* = \underline{\operatorname{Aut}}(\widehat{\G}_a)$ acts on this by coordinate changes (and we can pick a left-- or right--action as we see fit). If we pick an action by postcomposition, then we can do the following nice thing: set $f(t) = f_2(t) + g(t)$, where $f_2(t)$ contains just the terms in degrees perfect powers of $2$. Then $f_2^{-1}(f(t))$ is another coordinate with no terms in degrees perfect powers of $2$, and any nontrivial automorphism applied to this ``reduced'' series will re-introduce terms in degrees perfect powers of $2$.  So, this is a canonical form for the series under the $\underline{\operatorname{Aut}}(\widehat{\mathbb G}_a)$--action which admits no further automorphisms. It should follow that $H^*(\context{H\F_2}; \context{H\F_2}(MO))$ has amplitude $0$ and takes the form $\F_2[x_j \mid j \ne 2^n - 1]$, i.e., whose generating function is arbitrary other than having no terms in degrees perfect powers of $2$.

This means that the Adams spectral sequence degenerates and this computes $\pi_* MO$.  (It would be nice to interpret this in terms of a logarithm on $\RP^\infty_{H\F_2}$.)  It also means that the Hurewicz map is injective, hence that $MO$ is a retract of $H\F_2 \sm MO$, hence that $MO$ is an $H\F_2$--module, hence that $MO$ is a wedge of shifts of $H\F_2$.



\end{document}
