% -*- root: main.tex -*-

\setcounter{chapter}{-1}
\chapter{Introduction}

\section{Introduction}\label{IntroductionSection}

The goal of this class is to communicate a certain \textit{weltanschauung} uncovered in pieces by many different people working in bordism theory, and the goal just for today is to tell a story about one theorem where it is especially apparent.

To begin, we will define a homology theory called ``bordism homology''.  Recall that the singular homology of a space $X$ is defined by considering the collection of continuous maps $\sigma: \Delta^n \to X$, taking the free $\Z$--module on each of these sets, and constructing a chain complex \[\cdots \xrightarrow{\partial} \Z\{\Delta^n \to X\} \xrightarrow{\partial} \Z\{\Delta^{n-1} \to X\} \xrightarrow{\partial} \cdots.\]  Bordism homology is constructed analogously, but using manifolds $M$ as the sources instead of simplices:
\begin{align*}
\cdots & \xrightarrow{\partial} \{M^n \to X \mid \text{$M^n$ an $n$--manifold}\} \\
& \xrightarrow{\partial} \{M^{n-1} \to X \mid \text{$M^{n-1}$ an $(n-1)$--manifold}\} \\
& \xrightarrow{\partial} \cdots.
\end{align*}

\begin{lemma}
This forms a chain complex of monoids under direct sum of manifolds, ands it homology is written $MO_*(X)$.  These are naturally abelian groups, and moreover they satisfy the axioms of a generalized homology theory. \qed
\end{lemma}

\todo{dshi: I'm confused about this paragraph. What is $X$ here? A sequence of groups? How does $O(n)$ relate to the $MO$ story above? 10min later: So $X$ is a structure group. There is a potential for confusion here cuz $X$ is a space above. Can you explain this part of the story to me (again) at office hours on Tuesday.} In fact, we can define a bordism theory $MX$ for any suitable family of structure groups $X_n \to O(n)$.  The coefficient ring of $MX$, or its value $MX_*(*)$ on a point, gives the ring of $X$--bordism classes, and generally $MX_*(Y)$ of some space $Y$ gives a kind of ``bordism in families (over $Y$)''.  There are evident comparison morphisms for the most ordinary kinds of bordism, given by replacing a chain of manifolds with an equivalent simplicial chain:\todo{dshi:I don't follow here. How does this replacement go explicitly? Somehow I understood it when you explained this to me in person but now I don't see it anymore.} \[MO \to H\Z/2, \quad MSO \to H\Z.\] In both cases, we can evaluate on a point to get ring maps $MO_*(*) \to \Z/2$ and $MSO_*(*) \to \Z$, called ``genera'' --- neither of which is very interesting, since they're both zero in positive degrees.\todo{A comparison of this with the usual spectrum definition of $MX$ appears in Switzer 12.35.}

However, having maps of homology theories (rather than just maps of coefficient rings) is considerably more data then just the genus.  In fact, we can extract a theory of integration.  Consider the following special case of oriented bordism, where we evaluate $MSO_*$ on an infinite loopspace:
\begin{align*}
MSO_n K(\Z, n) & = \left\{ \text{oriented $n$--manifolds mapping to $K(\Z, n)$} \right\} / \sim \\
& = \left\{ \begin{array}{c}\text{oriented $n$--manifolds $M$} \\ \text{with a specified class $\omega \in H^n(M; \Z)$} \end{array}\right\} / \sim.
\end{align*}
Associated to such a representative $(M, \omega)$, the yoga of stable homotopy theory then allows us to build a composite
\begin{align*}
\S & \xrightarrow{\mathmakebox[2.5em]{(M, \omega)}} MSO \sm (\S^{-n} \sm \Susp^\infty_+ K(\Z, n)) \\ 
& \xrightarrow{\mathmakebox[2.5em]{\colim}} MSO \sm H\Z \\
& \xrightarrow{\mathmakebox[2.5em]{\phi \sm 1}} H\Z \sm H\Z \\
& \xrightarrow{\mathmakebox[2.5em]{\mu}} H\Z,
\end{align*}
\todo{I changed $\S^0$ to $\S$ here, because that's what you used below, but it seems that the notation for the sphere spectrum has been inconsistent elsewhere too.}
where $\phi$ is the orientation map.  Altogether, this composite gives us an element of $\pi_0 H\Z$, i.e., an integer.

\begin{lemma}
The integer obtained by the above process is $\int_M \omega$. \qed
\end{lemma}

\noindent This definition of $\int_M \omega$ via stable homotopy theory is pretty nice, in the sense that many theorems accompany it for free.  For instance, the relation ``$\sim$'' automatically imposes a Stokes' theorem on it.\todo{I thought I knew how this worked, but now I'm not so sure.  How does this work?}

Now take $X = e$ to be the trivial structure group, which is the bordism theory of manifolds with trivialized tangent bundle.  In this case, the Pontryagin--Thom construction gives an equivalence $Me \xrightarrow{\simeq} \S$.  It is thus possible (and some people have indeed taken up this viewpoint) that stable homotopy theory can be done solely through the lens of ``framed bordism''.  We will prefer to view this the other way: the sphere spectrum $\S$ often appears to us as a natural object, and we will occasionally replace it by $Me$, the framed bordism spectrum.  For example, given a ring spectrum $E$ with unit map $\S \to E$, we can reconsider this as a ring map $\S = Me \to E$.  Following along the lines of the previous paragraph, we learn that any ring spectrum $E$ is automatically equipped with a theory of integration for framed manifolds.

Sometimes, as in the examples above, this unit map factors: \[\S = Me \to MO \to H\Z/2.\]  This is a witness to the overdeterminacy of $H\Z/2$'s integral for framed bordism: if the framed manifold is pushed all the way down to an unoriented manifold, there is still enough residual data to define the integral.  Given any ring spectrum $E$, we can ask the analogous question: If we filter $O$ by a system of structure groups, at what stage does the unit map $Me \to E$ factor through?  For instance, the map \[\S = Me \to MSO \to H\Z\] considered above does \emph{not} factor further through $MO$ --- an orientation is \emph{required} to define the integral of an integer--valued cohomology class.  In the more general case, the map $SO \to O$ is the beginning of the Postnikov filtration of $O$, \todo{danny: do you mean postnikov filtration for $MO$? I asked this in class but I think it's good to say how does the filtration go. The classical postnikov filtration for $X$ builds the homotopy groups of $X$ up from the bottom, to get a sequence $\cdots \to X_2 \to X_1 \to X_0$, with $X$ the limit. I think the situation here is the opposite? We'll talk about this.} and we now present a diagram of this filtration and some interesting integration theories related to it:
\begin{center}
\begin{tikzcd}
Me \arrow{r} \arrow{rrd} \arrow{rrrd} \arrow{rrrrd} & \cdots \arrow{r} & M\Spin \arrow{r} \arrow[crossing over]{d} & MSO \arrow{r} \arrow[crossing over]{d} & MO \arrow[crossing over]{d} \\
& & kO & H\Z & H\Z/2.
\end{tikzcd}
\end{center}
\todo{Given that you mentioned string in the theorem below.  Might wanna add string into the diagram}

This is the situation homotopy theorists found themselves in some decades ago, when Ochanine and Witten proved the following mysterious theorem using analytical and physical methods:

\begin{theorem}[Ochanine, Witten]
There is a map of rings \[\sigma: M\Spin_*\todo{Is this $M\Spin_*(*)$?  On second thought, ignore this comment; I'm being silly.} \to \C(\!(q)\!).\]  Moreover, if $M$ is a Spin manifold such that twice its first Pontryagin class vanishes --- that is, if $M$ lifts to a $\String$--manifold --- then $\sigma(M)$ lands in the subring $MF \subseteq \Z\ps{q}$ of modular forms with integral coefficients. \qed
\end{theorem}

\noindent However, neither party gave indication that their result should be valid ``in families'', and no theory of integration was produced.  From the perspective of the homotopy theorist, it wasn't even totally clear what such a claim would mean: to give a topological enrichment of these theorems would mean finding a ring spectrum $E$ such that $E_*(*)$ had something to do with modular forms.  Around the same time, Landweber, Ravenel, and Stong began studying ``elliptic cohomology'' for independent reasons; sometime much earlier, Morava had constructed an object ``$K^{\Tate}$'' associated to the Tate elliptic curve; and a decade later Ando, Hopkins, and Strickland put all these things together in the following theorem:

\begin{theorem}[Ando--Hopkins--Strickland]
If $E$ is an ``elliptic cohomology theory'', then there is a canonical map $M\String \to E$ called the $\sigma$--orientation.  In particular, the map $M\String_* \to K^{\Tate}_*$ is Witten's genus. \qed
\end{theorem}

We now come to the motivation for this class.  The homotopical $\sigma$--orientation was actually first constructed using formal geometry.  The original proof of Ando--Hopkins--Strickland begins with a reduction to maps of the form \[MU[6, \infty) \to E.\]  They then work to show that in especially good cases they can complete the missing arrow in the diagram
\begin{center}
\begin{tikzcd}
MU[6, \infty) \arrow{r} \arrow{rd} & M\String \arrow[densely dotted]{d} \\
& E.
\end{tikzcd}
\end{center}
Leaving aside the extension problem for the moment, their main theorem is the following description of the cohomology ring $E^* MU[6, \infty)$:
\begin{theorem}[Ando--Hopkins--Strickland]
For $E$ an even--periodic cohomology theory, \[\Spec E_* MU[6, \infty) \cong C^3(\G_E; \sheaf I(0)),\] where ``$C^3(\G_E; \sheaf I(0))$'' is a certain scheme.  When $E$ is taken to be elliptic, so that there is a specified isomorphism $\G_E \cong C^\wedge_0$ for $C$ an elliptic curve, the theory of elliptic curves furnishes the scheme with a canonical point.  Hence, there is a preferred class $MU[6, \infty) \to E$, natural in the choice of elliptic $E$. \qed
\end{theorem}

\noindent Our real goal is to understand theorems like this last one, where algebraic geometry asserts some real control over something in the domain of homotopy theory.  The structure of the class will be to work through a sequence of case studies where this perspective shines through most brightly.  We'll start by working through Thom's calculation of the homotopy of $MO$, which simultaneously holds the attractive features of being free of technical complexity while revealing essentially all of the structural complexity.  Having seen that through to the end, we'll then venture on to other examples: the complex bordism ring, structure theorems for finite spectra, unstable cooperations, and, finally, the theorem above and its extensions.  The overriding theme of the class will be that algebraic geometry is a good organizing principle that gives us one avenue of insight into how homotopy theory functions.  In particular, it allows us to organize ``operations'' of various sorts between spectra derived from bordism theories.

We should also mention that we will specifically \emph{not} discuss the following aspects of this story:
\begin{itemize}
\item Analytic techniques will be completely omitted.  Much of modern research stemming from the above problem is an attempt to extend index theory across Witten's genus, and this often means heavy analytic work.  We will strictly confine ourselves to the domain of homotopy theory.
\item As sort of a sub-point (and despite the motivation provided in this Introduction), we will also mostly avoid manifold geometry.  (We do give a proof of Quillen's theorem on the structure of $MU_*$ which invokes some mild amount of manifold geometry.)  Again, much of the contemporary research about $\tmf$ is an attempt to find a geometric model, so that geometric techniques can be imported --- including equivariance and the geometry of quantum field theories, to name two.
\item In a different direction, our focus will not linger on actually computing bordism rings $MX_*$, nor will we consider geometric constructions on manifolds and their behavior after imagining\todo{imagining??} into the bordism ring.  This is also the source of active research: the structure of the symplectic bordism ring remains, to large extent, mysterious, and what we do understand of it comes through a mix of formal geometry and raw manifold geometry.  This could be a topic that fits logically into this document, were it not for time limitations and the author's inexpertise.
\item The geometry of $E_\infty$ rings will also be avoided.  These really are inescapable at the conclusion of the story we will tell here, but there are better resources from which to learn about $E_\infty$ rings, and the pre--$E_\infty$ story is not told so often these days.  So, we will focus on the unstructured part and leave $E_\infty$ rings to other authors.
\end{itemize}

Finally, we will also mention good companions to these notes.  Essentially none of the material here is original --- it's almost all cribbed either from published or unpublished sources --- but the source documents are quite scattered and dense.  We will make a point to cite useful references as we go.  One document stands out above all others, though: Neil Strickland's \textit{Functorial Philosophy for Formal Phenomena}~\cite{StricklandFPFP}.  These lecture notes can basically be viewed as an attempt to make it through this paper in the span of a semester.

\todo{Akhil wrote a couple of blog posts about Ochanine's theorem: \texttt{https://amathew.wordpress.com/2012/05/30/ochanines-theorem-on-elliptic-genera/} and \texttt{https://amathew.wordpress.com/2012/05/31/the-other-direction-of-ochaines-theorem/}. Mentioning a more precise result might lend to a more beefy introduction.}
