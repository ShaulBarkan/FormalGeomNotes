% -*- root: main.tex -*-

\chapter{Complex bordism}



\section{Formal varieties}

Having totally dissected unoriented bordism, we can now turn our attention to other sorts of bordism theories, and there are many available: oriented, $\Spin$, $\String$, complex, \ldots.  We would like to replicate the results above for these other contexts, but we quickly see that only one of the listed bordism theories supports this program.  The space $\RP^\infty = BO(1)$ was a key player in the unoriented bordism story, and the only other bordism theory with a similar ground object is complex bordism, with $\CP^\infty = BU(1)$.  So, we will focus on it.

The content of the first lecture can be replicated essentially \textit{mutatis mutandis}, resulting in the omnibus theorem
\begin{theorem}
There is a complex $J$--homomorphism \[J_{\C}: BU \to B \GL_1 \S.\]  It has a Thom spectrum $MU$, for a complex vector bundle $\xi$ on a space $X$ and a ring spectrum $E$ under $MU$, there is a natural equivalence \[E \sm T(\xi) \simeq E \sm \Susp^\infty_+ X.\]  In particular, it follows that $E^* \CP^\infty$ is isomorphic to a one--dimensional power series ring. \qed
\end{theorem}

In light of this theorem, it seems prudent to develop some of the theory of formal schemes and formal varieties.



\begin{definition}
Formal affine $n$--space is defined by \[\A^n = \Spf R\llbracket x_1, \ldots, x_n\rrbracket.\]  A \textit{formal affine variety} is a formal scheme $V$ which is (noncanonically) isomorphic to $\A^n$ for some $n$.  The two maps in an isomorphism pair \[V \to \A^n, \quad V \leftarrow \A^n\] are called a coordinate (system) and a parameter (system) respectively.
\end{definition}

Maps between affine $n$--spaces

\begin{definition}
Let $V$ be a formal variety and let $I_V = \A^1(V)$ be the ideal of functions vanishing at the origin.  Then, we define the \textit{cotangent space of $V$} at the origin by \[T^* V = I_V / I_V^2.\]  Similarly, a point $f \in V(R[t] / t^2)$ can be written as $f = f(0) + v(f) t$ where $v(f)$ is determined by the image of $f - f(0)$ in $T^* V$.  Hence, we define the \textit{tangent space of $V$} by \[TV = V(R[t] / t^2) \cong \CatOf{Modules}_R(T^* V, R).\]
\end{definition}

\begin{theorem}
A map $f: V \to W$ of formal varieties is an isomorphism if and only if the induced map $Tf: TV \to TW$ is an isomorphism of $R$--modules.
\end{theorem}
\begin{proof}
This is 3.1.8 in the Crystals notes.
\end{proof}

\begin{remark}
You can get formal varieties by completing varieties.
\end{remark}

\todo{You can also get formal varieties using this detection theorem...}

\begin{theorem}
\todo{I think this theorem is motivated by Artin--Mazur formal groups, and the Crystals notes use it to extract a formal group from a Dieudonn\'e module.  Some motivation could go here.}Let $A$ be a Noetherian ring and $G: \CatOf{AdicAlgebras}_A \to \CatOf{AbelianGroups}$ be a functor such that
\begin{enumerate}
\item $G(A) = 0$.
\item $G$ takes surjective maps to surjective maps.
\item There is a finite, free $A$--module $M$ and a functorial isomorphism \[I \otimes_A M \to G(B) \to G(B')\] whenever $I$ belongs to a square-zero extension of adic $A$--algebras \[I \to B \to B'.\]
\end{enumerate}
Then, $G \cong \A^n$ as a functor to sets, where $n = \dim M$.
\end{theorem}
\begin{proof}
This is 9.6.4 in the Crystals notes.
\end{proof}

\begin{definition}
A formal group is a formal variety endowed with an abelian group structure.\footnote{Formal groups in dimension $1$ are automatically commutative if and only if the ground ring has no elements which are simultaneously nilpotent and torsion.  \textbf{Cite this.}}
\end{definition}

\begin{remark}
\todo{Discuss the connection between formal groups with coordinates and formal group laws.  Introduce FGL notation.}
\end{remark}

\begin{remark}
Formal groups automatically have inverses.
\end{remark}

\begin{remark}
You can get formal groups from completions of algebraic groups.
\end{remark}

Quillen's theorem needs to know that rational formal groups have logarithms.  Now is a good time to prove that?  Or maybe this comes later.  If it comes here, it would be good to give the version that uses the tangent space:
\begin{theorem}
There is a unique isomorphism \[\G \xrightarrow{\log} \operatorname{Lie} \G \otimes \G_a.\]
\end{theorem}

\begin{example}
$\CP^\infty_{H\Z P} \cong \G_a$.  $\CP^\infty_{KU} \cong \G_m$.
\end{example}








\section{The splitting principle}

the splitting principle: gives Chern classes generally from knowing the calculation for $\CP^\infty$

the cohomology rings $E^* BU(n)$: also gives Chern classes generally, plus there's the divisor scheme structure

% divisors and line bundles on $1$--dimensional objects?

poincare duality for manifolds with oriented tangent bundle

wrong-way maps: $\zeta^* \zeta_* 1$ gives the Euler class of the bundle

Explicit Thom isomorphism map for universal cohomology: $\xi: X \to BU(n)$ Thomifies to $T(\xi) \to MU(n) \to MU$, representing a class $g \in MU^* T(\xi)$, and this gives a map
\begin{align*}
MU^*(X) & \to MU^* T(\xi) & \cong MU^*(E, E_0)\\
x & \mapsto & g \smile p^*(x),
\end{align*}
where $p: E \to X$ is the projection and $E_0$ is the image of the zero section.

The first part of Adams's blue book is about the Landweber--Novikov operations.  In particular, he talks about the choice of funny total Chern class, saying that duals to monomials in $H_* BU$ are what span the cohomology of $H^* BU$ and that those are what interact well with the Whitney sum.

The wrong--way maps come from conjugating by Poincar\'e duality: \[E^* X \cong E_{d_X-*} X \to E_{d_X-*} Y \cong E^*{*-d_X+d_Y} Y.\]  Poincar\'e duality comes from asserting that the stable normal bundle is oriented for the theory, and then Atiyah duality says \[D(X_+) \simeq \Susp^{-n} T(\nu) [\simeq T(\nu - n\eps) \simeq T(-\tau)]. \]

---

Here's some old stuff from when I thought it was good to prove the splitting principle for real bundles in the first case study.  This \emph{may} all be needed earlier than this, when we go through a proof of Quillen's theorem on $MU$.  Maybe I'll defer it then too.

---

Our first goal for today is to show the following freeness property of the ring spectrum $MO$:
\begin{theorem}
Let $E$ be a ring spectrum.  Homotopy classes of ring maps $MO \to E$ are in natural bijection with factorizations \[\S \to MO(1) \to E\] of the unit map for $E$. \qed
\end{theorem}

\noindent This proof falls into two halves, and one half is much easier than the other.  The data of a ring map $MO \to E$ appears to be considerly more data than a factorization, and showing that one begets the other turns out to be the easier direction of the proof.  Suppose that we're given such a ring map $MO \to E$, so that we can apply the Thom isomorphism machinery from the beginning of this story.  Then, recall the definition of $MO(1)$ as a Thom spectrum: \[MO(1) = T(\L - 1 \downarrow \RP^\infty).\]  Restricting the base space all the way to a point gives \[T(\L - 1 \downarrow *) = \S,\] and this fits into the following commutative diagram with the ring map we were given:
\begin{center}
\begin{tikzcd}
\S \arrow{r} \arrow{d} & MO \arrow{d} \\
MO(1) \arrow{ru} \arrow[densely dotted]{r} & E.
\end{tikzcd}
\end{center}
The horizontal arrow across the top is the unit map for $MO$, so the long composite is the unit map for $E$, and the dotted composite is the desired factorization of the unit.  In terms of cohomology classes, the Thom isomorphism gives \[E^* \RP^\infty \cong \widetilde E^* MO(1).\]  The left--hand group has the canonical element ``$1$'', and the data of ``$\cong$'' is the Thom isomorphism, sending $1$ to a canonical map $MO(1) \to E$.  This, too, is the dotted arrow.\todo{Can this all be phrased more clearly?}

Remembering that $MO(1) \simeq \Susp^{-1} \Susp^\infty \RP^\infty$, we see that this was what powered our computation of $H\F_2^* \RP^\infty$ from earlier, and in fact this map $MO(1) \to E$ is enough to deduce an Thom isomorphism in $E$--cohomology for $MO(1)$ alone.  The other direction of the proof then sounds more serious: we have to show that if we have a Thom isomorphism for the bundle involved in forming $MO(1)$, then we can extract from that compatible Thom isomorphisms for all bundles.  This kind of reduction is famous enough to have a name: it is called ``the splitting principle''.

\todo{Keep talking.}

------

The main point is that the fiber map $\CP^{n-1} \to P(V) \to X$ postcomposes by $P(V) \to \CP^\infty$ to the skeletal inclusion, so the Serre spectral sequence degenerates since its fiber is unmolested (or: the Leray--Hirsch theorem). The Chern classes are \emph{defined} by the one remaining multiplicative extension.





\section{Divisors and divisor schemes}

Divisors and schemes of divisors on $1$-dimensional formal groups

$BU(n)_E$, $BU_E$, and $(BU \times \Z)_E$

be sure to talk about the total chern class $c_{\mathbf t}$:
\begin{definition}
There is a sequence of characteristic classes $c_\alpha(E)$ wrapping into $c_{\mathbf t}(E) = \sum_{\alpha} \mathbf t^\alpha c_\alpha(E)$ satisfying
\begin{align*}
c_{\mathbf t}(E \oplus E') & = c_{\mathbf t}(E) \cdot c_{\mathbf t}(E'), \\
c_{\mathbf t}(\L) & = \sum_{j=0}^\infty t_j e(\L)^j, \\
c_{\mathbf t}(E) & = \operatorname{Norm}\left( \sum_{j \ge 0} t_j e(\mathcal O(1))^j \right),
\end{align*}
where ``$\operatorname{Norm}(\alpha)$'' denotes the determinant of multiplication by $\alpha$, which is well-defined since $U^*(\P E)[\mathbf t]$ is a finite dimensional module over $U^*(X)[\mathbf t]$.
\end{definition}









\section{Quillen's theorem}

--- we don't have this language yet ---

Today we will give an analysis of $\context{MU}$, due to Quillen.

\begin{theorem}[Quillen]
There are isomorphisms
\begin{align*}
\Spec MU_* & \to \moduli{fgl}, \\
\Spec MU_* MU & \to \moduli{fgl} \times \moduli{ps}^{\gpd}, \\
\context{MU} & \to \left. \left. \moduli{fgl} \middle/\!\!\!\middle/ \left( \moduli{fgl} \times \moduli{ps}^{\gpd} \right) = \moduli{fg} \right. \right. . \qed
\end{align*}
\end{theorem}

Q: Once you know that $MU$ has the universal formal group law on it, does the description of $MU_* MU$ follow immediately from evenness?  Probably?

--- here's some of the real proof ---

Need to talk about power operations and the construction of the Steenrod operations for a sufficiently geometric cohomology theory

Need to talk about Gysin pushforwards in complex bordism and in ordinary cohomology.  Compare these with the theory of Thom isomorphism in general.  They're equivalent, right?  A complex orientation makes proper maps induce shriek maps, and shriek maps can be used to deduce what Chern classes are by push-pull: if $\zeta: X \to E$ is the zero section of a complex bundle $\xi$, then $e(\xi) = i^* i_*(1)$ I think.

Need to talk about characteristic classes and the $c_{\mathbf t}(x) = \sum_\alpha c_\alpha(x) \cdot \mathbf t^\alpha$ ``total Chern class''.  This will be hard, since you want to talk about Chern classes way later.  Maybe you'll just have to do it now\ldots

\begin{definition}
Using complex oriented maps $Z \to X$ of complex manifolds to model complex cobordism cocycles, pullback along a map is given by transverse perturbation and categorical pullback / intersection, and pushforward along a \emph{proper} map is given by postcomposition.
\end{definition}

\begin{definition}\todo{Where do these even come from? Ask Mike. Or look at Buchstaber's papers on infinitesimal flows\ldots}
Let $f: Z \to X$ be a complex-oriented map of even dimension and whose orientation is represented by a factorization of an embedding $i$ into a complex vector bundle \[Z \xrightarrow i E \to X\] with complex normal bundle $\nu_i$.  Then, setting $\nu_f = f^* E - \nu_i \in K(Z)$, the \textit{Landweber--Novikov operations} \[s_{\mathbf t} = \sum_\alpha \mathbf t^\alpha s_\alpha: U^*(X) \to U^*(X)[\mathbf t]\] are defined by \[s_{\mathbf t}(f_* 1) = f_* c_{\mathbf t}(\nu_f),\] or more generally \[s_{\mathbf t}(f_* x) = f_*(c_{\mathbf t}(\nu_f) \cdot s_{\mathbf t}(x)).\]
\end{definition}

\begin{theorem}\todo{You really need these Riemann--Roch formulas.  Where do they come from?}
Let $f: Z \to X$ be a $G$--map and let $z \in h(Z)$.  Then there is the square
\begin{center}
\begin{tikzcd}
Z^G \arrow{r}{r_Z} \arrow{d}{f^G} & Z \arrow{d}{f} \\
X^G \arrow{r}{r_X} & X
\end{tikzcd}
\end{center}
and corresponding push-pull formula \[e(\mu(E)) \cdot r_X^* f_* z = f^G_*(e(\mu_i) \cdot r_Z^* z).\]
\end{theorem}
\begin{proof}
This is 3.8. I'm not even sure what $\mu(E)$ means. In 3.4-6, they talk about the case of an inclusion, and there $\mu_i$ denotes the part of a certain $G$--equivariant vector bundle that has nontrivial $G$--action.  (I think maybe he defines $\mu_i$ and $\mu_f$, labeled by maps $i$ and $f$, and since the projection map from $E$ doesn't have a name he just calls it $\mu(E)$ instead?  That's silly.)
\end{proof}

\begin{theorem}
Suppose $G$ acts transitively on $\{1, \ldots, k\}$ and let $\rho$ denote the corresponding representation of $G$ on the subspace of $(z_1, \ldots, z_k)$ in $\C^k$ such that $\sum_i z_i = 0$ and $G$ permutes the coordinates.  Suppose $f: Z \to X$ is a proper complex-oriented map of dimension $2q$ and that $m$ is an integer larger than the dimension of $Z$, so that $m \eps + \nu_f$ is a vector bundle over $Z$, well-defined up to isomorphism, where $\eps$ is the trivial complex line bundle.  Then in $h^{2m(k-1) - 2qk}(X)$ we have \[e(\rho)^m P(f_* 1) = f_* e(\rho \otimes (m \eps + \nu_f)).\]
\end{theorem}
\begin{proof}
This is 3.12.  It follows from letting $G$ act on $f^{\times k}: Z^{\times k} \to X^{\times k}$ and using the push-pull formula for the invariant spaces (which are $Z$ and $X$) sitting inside of $Z^{\times k}$ and $X^{\times k}$: $P(f_* 1) = \Delta^* f^{\times k}_* 1$.
\end{proof}

\begin{definition}
Let $f: Z \to X$ represent a complex-oriented cobordism class, and let $\L$ be any line bundle over $Z$ on which $G$ acts trivially.  Then:
\begin{align*}
e(\rho \otimes \L) & = e \left( \bigoplus_{i=1}^{k-1} \eta^i \otimes \L \right) \\
& = \prod_{i=1}^{k-1} e(\eta^i \otimes \L) \\
& = \prod_{i=1}^{k-1} \left( [i]_F(v) +_F e(\L) \right) \\
& = w + \sum_{j \ge 1} a_j(v) e(\L)^j.
\end{align*}
The series $a_j(T) \in C\llbracket T \rrbracket$ are defined by this last relation.  We also have \[w = e(\rho) = (k-1)! v^{k-1} + \sum_{j \ge k} b_j v^j\] with $b_j \in C$.  More generally, if $E$ is any vector bundle over $Z$ on which $G$ acts trivially, then \[e(\rho \otimes E) = \sum_{|\alpha| \le r} w^{r - |\alpha|} (a(v))^\alpha c_\alpha(E).\]
\end{definition}
\begin{proof}
This is the discussion around 3.16.
\end{proof}

\begin{theorem}
Let $Q \to B$ be a principal $\Z/k$--bundle and let \[P: U^{-2q}(X) \to U^{-2qk}(B \times X)\] be the $k${\th} Steenrod power operation. Let $v$ be the Euler class of the line bundle on $B$ given by tensoring $Q$ over the character $\Z/k \cong U^1[k] \subseteq U^1$, and let $w$ be the Euler class of the bundle similarly induced from $Q$ by the reduced regular representation $\rho$.  Then, the Steenrod operation entwined with $Q$ is related to the Landweber--Novikov operations by the formula \[w^{n+q} P(x) = \sum_{|a| \le n} w^{n - |\alpha|} a(v)^\alpha s_\alpha(x),\] where $n \gg 0$ and the $a_j(T)$ are power series given above with coefficients in the subring $C$ generated by the coefficients of the standard formal group law on $U^*$.
\end{theorem}
\begin{proof}
This is 3.17.
\end{proof}

\begin{theorem}
If $X$ is finite, then
\begin{align*}
U^*(X) & = C \cdot \sum_{q \ge 0} U^q(X), \\
\widetilde U^*(X) & = C \cdot \sum_{q > 0} U^q(X),
\end{align*}
where $C$ is the subring of $U^*(*)$ generated by the coefficients of the canonical formal group law.
\end{theorem}
\begin{proof}
This is 5.1 in Quillen's paper. Use induction and the Landweber--Novikov--Steenrod relationship.  It's the nicest result in the whole thing.  The point is that if you apply the $p$th power operation to a class in nonzero degree, then up to filtration it acts like the identity but also there's a formula in terms of lower classes and the Landweber--Novikov power series, and this reduction to lower classes lets you power an induction.
\end{proof}

\begin{corollary}
$U^{ev}(*) = C$ and $U^{odd}) = 0$. \qed
\end{corollary}

\begin{definition}(6.2)
Let $\eps: U^*(X) \to H^*(X)$ be the reduction map and let the Boardman map $\beta: U^*(X) \to H^*(X)[\mathbf t]$ define Gysin maps in ordinary cohomology by $\beta = \eps \circ s_{\mathbf t}$ and hence for a proper complex oriented map $f: Z \to X$: \[\beta(f_* z) = f_*(c_{\mathbf t}^H(\nu_f) \cdot \beta z).\] 
\end{definition}

\begin{definition}
It follows that \[\beta e^U(\L) = \sum_{j \ge 0} t_j(e^J(\L))^{j+1}\] so setting \[\theta_{\mathbf t}(T) = \sum_{j \ge 0} t_j T^{j+1}\] we have \[(\beta \otimes^*)(\theta_{\mathbf t}(T_1), \theta_{\mathbf t}(T_2)) = \theta_{\mathbf t}(T_1 + T_2).\]  It further follows that there are ring homomorphisms
\begin{center}
\begin{tikzcd}
L \arrow{r}{\delta} & U^*(*) \arrow{r}{\beta} & \Z[t] \\
F_{univ} \arrow[|->]{r} & F \arrow[|->]{r} & \theta_{\mathbf t}^*(T_1 + T_2),
\end{tikzcd}
\end{center}
where ``$\theta_{\mathbf t}^*$'' denotes conjugation by a power series.
\end{definition}

\begin{theorem}
The homomorphism $\delta$ is an isomorphism and the homomorphism $\beta$ is injective.
\end{theorem}
\begin{proof}
We know that $\delta$ is surjective by the arguments about the subring $C$.  On the other hand, the rational composite $\Q \otimes (\beta \circ \delta)$ induces an isomorphism $\Q \otimes L \to \Q[\mathbf t]$ using the logarithm.  By Lazard's theorem, $L$ is torsion-free, so $\beta \circ \delta$ is injective.  The conclusion follows.
\end{proof}


\section{Lazard's theorem}

This should have a clever ``The structure of $\moduli{fg}$'' name.
