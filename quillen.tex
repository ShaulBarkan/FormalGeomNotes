% -*- root: main.tex -*-

\chapter{Complex bordism}\label{ComplexBordismChapter}


\todo{Write an introduction for me.}


\section{Formal varieties}\label{FormalVarietiesLecture}

\todo[inline]{I think this lecture may be too long.  On the other hand, the rational stuff at the end will go rather quickly --- which I know from experience in the Pittsburgh talks.}

Having totally dissected unoriented bordism, we can now turn our attention to other sorts of bordism theories, and there are many available: oriented, $\Spin$, $\String$, complex, \ldots.  We would like to replicate the results from \Cref{UnorientedBordismChapter} for these other contexts, but we quickly see that only one of the listed bordism theories supports this program.  The space $\RP^\infty = BO(1)$ was a key player in the unoriented bordism story, and the only other bordism theory with a similar ground object is complex bordism, with $\CP^\infty = BU(1)$.  So, we will focus on it.

The contents of \Cref{LectureThomSpectra} can be replicated essentially \textit{mutatis mutandis}, resulting in the following theorems:
\begin{theorem}\citeme{Give a reference from Lecture 1.1}
There is a complex $J$--homomorphism \[J_{\C}: BU \to B \GL_1 \S. \qed \]
\end{theorem}

\begin{definition}\citeme{Give a reference from Lecture 1.1}
The associated Thom spectrum is written ``$MU$'' and called \textit{complex bordism}.  A map $MU \to E$ of ring spectra is said to be a \textit{complex orientation of $E$}.\todo{Something I've seen more than once is an equivalence $MU(k) \simeq BU(k) / BU(k-1)$. It's not immediately obvious to me where this comes from. Where does it come from? Is it helpful to think about?}
\end{definition}

\begin{theorem}\label{ThomIsomOverC}\citeme{Give a reference from Lecture 1.1}
For a complex vector bundle $\xi$ on a space $X$ and a complex-oriented\todo{I don't remember discussing orientations in the last chapter (probably because it's not needed), but maybe you can say something about where the complex orientation is used in this theorem?} ring spectrum $E$, there is a natural equivalence \[E \sm T(\xi) \simeq E \sm \Susp^\infty_+ X. \qed\]
\end{theorem}

\begin{corollary}\label{CPinftyNiceCalculation}\citeme{Give a reference from Lecture 1.1}
In particular, for a complex-oriented ring spectrum $E$ it follows that $E^* \CP^\infty$ is isomorphic to a one--dimensional power series ring. \qed
\end{corollary}
\todo{Maybe I'm confused about grading issues, but I thought $E^* \C P^\infty$ was a polynomial ring and $EP^0 \C P^\infty$ is the power series ring?}
\todo{Also, this is a nice argument.  Usually this computation proceeds through the AHSS.  Can this method be adapted to spaces other than $\C P^\infty$?}

In light of these results, it seems prudent to develop some of the theory of formal schemes and formal varieties outside of the context of $\F_2$--algebras.

\begin{definition}\todo{Erick has been complaining about this definition for a while, and I think he's right. His suggestion is for each scheme to be a nilpotent thickening over its reduction, which is of finite type (over whatever base).  I kept compulsively writing ``Artinian'', but he pointed out that $\Z[x] / x^n$ is \emph{not} Artinian, and so this can't be the right assumption.  I'm not sure why I was so stuck on this word... am I forgetting some important case?}\todo{Here is a very relevant MathOverflow question, replete with stacks project reference: http://mathoverflow.net/questions/236351/are-all-formal-schemes-really-ind-schemes .}
Fix a scheme $S$.  A formal $S$--scheme $X = \{X_\alpha\}_\alpha$ is an ind-system of Artinian $S$--schemes $X_\alpha$.\todo{Why Artinian? Also, you used to say finite instead of Artinian, so maybe you can standardize the terminology across chapters.} $S$--schemes $X_\alpha$.
\end{definition}

\begin{remark}
In the case $S = \Spec k$ for a field $k$, ``Artinian'' means that $\sheaf{O}_{X_\alpha}$ is a finite--dimensional $k$--vector space.
\end{remark}

These ind-systems arise when studying completions of rings.  To address the geometric situation, we first owe ourselves a definition of a closed subscheme:
\begin{definition}\label{DefnCompletion}
Let $X$ be an affine formal scheme\todo{I'm guessing this means isomorphic to the constant ind-system $\{\Spec R\}$?}, and pick a chart $\Spec R \to X$.  A subscheme $Y \subseteq X$ is called \textit{closed} when it has the form
\begin{center}
\begin{tikzcd}
Y \arrow{r} \arrow[-,double,densely dotted]{d} & X \arrow[-,double]{d} \\
\Spec (R/I) \arrow{r} & \Spec R.
\end{tikzcd}
\end{center}
\end{definition}
\todo{I don't understand your decorations for the vertical arrows.  Shouldn't they be arrows?  And why is the left one densely dotted?}

There's a complementary notion of an open subscheme, which we will continue to avoid for now.  These definitions are both best stated in a coordinate--free way, but the open subscheme version really \emph{requires} it, so we will postpone it until later.\oweproof{Definition of open scheme}\oweproof{Definition of closed subscheme without chart}  For now, we will proceed with the geometry:

\begin{definition}
Consider such a closed subscheme $Y$ of an affine $S$--scheme $X$, modeled by a map $\Spec R/I \to \Spec R$.  We define the \textit{$n${\th} order neighborhood of $Y$ in $X$} to be the scheme $\Spec R/I^{n+1}$.  The \textit{formal neighborhood of $Y$ in $X$} is then the ind-system \[X^\wedge_Y := \left\{ \Spec R/I \to \Spec R/I^2 \to \Spec R/I^3 \to \cdots \right\}.\]\todo{When is this system one of Artinian schemes?  The condition we came to in class was that $I$ is its own radical and $R$ is Noetherian.  I'm very mildly uncomfortable with this condition on $I$.}
\end{definition}
\todo{I thought in class you said that $\sqrt{I}$ is maximal instead of $I$ is radical?}

\noindent So, formal schemes arise naturally when studying the local geometry of $X$ near a subscheme $Y$.  An exceedingly common situation is for $X$ to be a variety and $Y$ to be a smooth point, so that $X^\wedge_Y$ looks like ``a small piece of affine space''.  We pin this important case down with a definition:

\begin{definition}
In the case that $S = \Spec R$ is affine, formal affine $n$--space over $S$ is defined by \[\A^n = \Spf R\llbracket x_1, \ldots, x_n\rrbracket.\]  \todo{I have very little intuition for $\Spf R$.  Could you define this more precisely, especially as an ind-system from the functors of points perspective?} A \textit{formal affine variety} is a formal scheme $V$ which is (noncanonically) isomorphic to $\A^n$ for some $n$.  The two maps in an isomorphism pair \[V \to \A^n, \quad V \leftarrow \A^n\] are called a \textit{coordinate (system)} and a \textit{parameter (system)} respectively.
\end{definition}

\begin{lemma}\label{MapsOfFVarsArePowerSeries}
A pointed map $\A^n \to \A^m$ is identical to an $m$--tuple of $n$--variate power series with no constant term. \qed
\end{lemma}

\begin{remark}
In some sense, \Cref{MapsOfFVarsArePowerSeries} is a full explanation for why anyone would even think to involve formal geometry in algebraic topology (nevermind how useful the program has been in the long run).  Calculations in algebraic topology are frequently expressed in terms of power series rings, and with this Lemma we are provided geometric interpretations for such statements.
\end{remark}

\Cref{MapsOfFVarsArePowerSeries} shows how formal varieties are especially nice, because maps between them can be boiled down to statements about power series.  In particular, this allows local theorems from analytic differential geometry to be imported, including a version of the inverse function theorem, which we will now work towards.\todo{Work an example first of $\A^1$, rather than an abstract formal variety.}

\begin{definition}
Let $V$ be a formal variety and let $I_V = \A^1(V)$ \todo{Strictly speaking, I don't think this notation has been introduced.} be the ideal of functions vanishing at the origin.  Then, we define the \textit{cotangent space of $V$} at the origin by \[T^* V = I_V / I_V^2.\]
\end{definition}

\begin{lemma}
There is an isomorphism \[TV \cong \CatOf{Modules}_R(T^* V, R).\]
\end{lemma}
\begin{proof}
A point $f \in V(R[\eps] / \eps^2)$ is given by a map $f\co \sheaf O_V \to R[\eps] / \eps^2$.  If $f$ is pointed, then it carries the ideal $I(0)$ of functions vanishing at zero to the ideal $(\eps)$, and hence also carries $I(0)^2$ to $(\eps)^2 = 0$.  Hence, $f$ induces a map
\begin{center}
\begin{tikzcd}
I(0) / I(0)^2 \arrow{r}{f} \arrow[-,double]{d} & (\eps) / (\eps)^2 \arrow[-,double]{d} \\
T^* V \arrow{r}{f} & R,
\end{tikzcd}
\end{center}
hence a point in $\CatOf{Modules}_R(T^* V, R)$.  This assignment is visibly bijective.
\end{proof}

\begin{theorem}\label{InverseFunctionTheoremForFVars}\citeme{This is 3.1.8 in the Crystals notes.}
A map $f\co V \to W$ of finite--dimensional formal varieties is an isomorphism if and only if the induced map $Tf\co TV \to TW$ is an isomorphism of $R$--modules.
\end{theorem}
\begin{proof}
First, reduce to the case where $V \cong \A^n$ and $W \cong \A^n$ have the same dimension \todo{I don't know how you are defining $\Spf$, but there may be something to show here, since you're not actually ``reducing'' to the case but showing that it `is'' the case.}, and select charts for both.  Then, $Tf$ is a matrix of dimension $n \times n$.  If $Tf$ fails to be invertible, we are done, and if it is invertible, we replace $f$ by $f \circ (Tf)^{-1}$ so that $Tf$ is the identity matrix.

We now construct the inverse function by induction on degree.  Set $g^{(1)}$ to be the identity function, so that $f$ and $g^{(1)}$ are mutual inverses when restricted to the first-order neighborhood.  So, suppose that $g^{(r-1)}$ has been constructed, and consider its interaction with $f$ on the $r${\th} order neighborhood: \[g_i^{(r-1)}(f(x)) = x_i + \sum_{|J| = r} c_J x_1^{J_1} \cdots x_n^{J_n} + o(r+1). \]  By adding in the correction term \[g_i^{(r)} = g_i^{(r-1)} - \sum_{|J| = r} c_J x_1^{J_1} \cdots x_n^{J_n},\] we have $g_i^{(r)}(f(x)) = x_i + o(r)$.
\end{proof}

We now return to our motivating example of $\CP^\infty_E$ for $E$ a complex-oriented cohomology theory, where we saw that the complex-orientation determines an isomorphism $\CP^\infty_E \cong \A^1$.  However, the object ``$E^* \CP^\infty$'' is something that exists independent of the orientation map $MU \to E$, and we now have the language to tease apart this situation:
\begin{lemma}
A cohomology theory $E$ is \textit{complex orientable} (i.e., it is able to receive a ring map from $MU$) precisely when $\CP^\infty_E$ is a formal curve\todo{Is this just a formal variety of dimension 1?}.  A choice of map $MU \to E$ determines a coordinate $\CP^\infty_E \cong \A^1$. \qed
\end{lemma}

As we saw in the first case study, $\CP^\infty_E$ has more structure than just a formal scheme: it also carries the structure of a group.  We close today with some remarks about such objects.

\begin{definition}\citeme{Theorem 2.2.6 of the Crystals notes}
A formal group is a formal variety endowed with an abelian group structure.\footnote{Formal groups in dimension $1$ are automatically commutative if and only if the ground ring has no elements which are simultaneously nilpotent and torsion.}
\end{definition}

\begin{remark}
As with formal schemes, formal groups can arise as formal completions of an algebraic group at its identity point.  It turns out that there are many more formal groups than come from this procedure, a phenomenon that is of keen interest to stable homotopy theorists.\todo{Add some kind of reference to a complaint about this?  It's not like we're going to talk about $\TMF$ much.}
\end{remark}

\begin{corollary}
As with physical groups, the formal group addition map on $\G$ determines the inverse law.
\end{corollary}
\begin{proof}
Consider the shearing map
\begin{align*}
\G \times \G & \xrightarrow{\sigma} \G \times \G, \\
(x, y) & \mapsto (x, x + y).
\end{align*}
The induced map $T\sigma$ on tangent spaces is evidently invertible, so by \Cref{InverseFunctionTheoremForFVars} there is an inverse map $(x, y) \mapsto (x, y - x)$.  Setting $y = 0$ and projecting to the second factor gives the inversion map.
\end{proof}

\begin{definition}\label{FGLDefinition}
Let $\G$ be a formal group.  In the presence of a coordinate $\phi \co \G \cong \A^n$, the addition law on $\G$ begets a map
\begin{center}
\begin{tikzcd}
\G \times \G \arrow{r} \arrow[-,double]{d} & \G \arrow[-,double]{d} \\
\A^n \times \A^n \arrow{r} & \A^n,
\end{tikzcd} \todo{Again, the vertical arrows should be arrows, not equal signs?}
\end{center}
and hence a $n$-tuple of $(2n)$-variate power series ``$+_\phi$'', satisfying
\begin{align*}
\underline{\smash x} +_\phi \underline{\smash y} & = \underline{\smash y} +_\phi \underline{\smash x}, & \text{(commutativity)} \\
\underline{\smash x} +_\phi \underline{\smash 0} & = \underline{\smash x}, & \text{(unitality)} \\
\underline{\smash x} +_\phi (\underline{\smash y} +_\phi \underline{\smash z}) & = (\underline{\smash x} +_\phi \underline{\smash y}) +_\phi \underline{\smash z}. & \text{(associativity)}
\end{align*}
Such a tuple $+_\phi$ is called a \textit{formal group law}.
\end{definition}

Let's now consider two examples of $E$ which are complex-orientable and describe $\CP^\infty_E$ for them.

\begin{example}
There is an isomorphism $\CP^\infty_{H\Z P} \cong \G_a$.  This follows from reasoning identical to that given in \Cref{RPinftyExampleForReal}.
\end{example}

\begin{example}\todo{This calculation ignores the grading, which isn't great. If you're careful and distinguish $c$ from $c_1$, things should fall into place better.}\todo{Make it clear that we're using the coordinate $1 - t$ on $\Gm$.}
There is also an isomorphism $\CP^\infty_{KU} \cong \G_m$.  Given a complex line bundle $\L$ over a space $X$, we use the complex orientation of $KU$ where the total Chern class of $\L$ is given by \[c(\L) = 1 - [\L].\]  Given this definition, we perform a manual computation:
\begin{align*}
c(\L_1 \otimes \L_2) & = 1 - [\L_1 \otimes \L_2] = 1 - [\L_1] [\L_2] \\
& = -1 + [\L_1] + [\L_2] - [\L_1] [\L_2] + 1 - [\L_1] + 1 - [\L_2] \\
& = (1 - [\L_1]) + (1 - [\L_2]) - (1 - [\L_1])(1 - [\L_2]) \\
& = c(\L_1) + c(\L_2) - c(\L_1) c(\L_2).
\end{align*}
In this coordinate on $\CP^\infty_{KU}$, the group law is then $x +_! y = x + y - xy$. \todo{Why is the notation $+_!$?}
\end{example}

We will close today by showing that the rational theory of formal groups is highly degenerate, similar to the rational theory of spectra.

\begin{definition}
The module of \textit{K\"ahler differentials} on a $k$--algebra $R$ is an $R$--module $\Omega^1_{R/k}$.  It is generated by symbols $dr$ for each element $r \in R$, subject to the two families of relations
\begin{align*}
ds & = 0, \, s \in k & \text{(differentiation is linear for ``scalars'')} \\
d(rr') & = rdr' + r'dr. & \text{($d$ is a derivation)}
\end{align*}
Elements of $\Omega^1_{R/k}$ are referred to as \textit{$1$--forms}.
\end{definition}

\begin{lemma}\todo{Do we ever use this? Does it provide intuition? Hmph.}
The module $\Omega^1_{R/k}$ is universal for derivations into $R$--modules: \[\CatOf{Derivations}_k(R, M) = \CatOf{Modules}_R(\Omega^1_{R/k}, M). \qed\]
\end{lemma}

These definitions are interesting in this level of generality, but suppose now that $k$ is a $\Q$--algebra and that $R = k\llbracket x \rrbracket$ is the coordinatized ring of functions on a formal line over $k$.  What's special about this rational curve case is that differentiation gives an isomorphism between $\Omega^1_{R/k}$ and the ideal $(x)$ of functions vanishing at the origin, i.e., the ideal sheaf selecting the closed subscheme $\Spec k \to \Spf R$.  Its inverse is formal integration: \[\int \co \left(\sum_{j=0}^\infty c_j x^j \right) dx \mapsto \sum_{j=0}^\infty \frac{c_j}{j+1} x^{j+1}.\]

Taking a cue from classical Lie theory, we attempt to define exponential and logarithm functions for a given formal group law $F$.  \todo{We should show here that the submodule of invariant $1$--forms is equivalent to the module of tangent vectors.}This is typically accomplished by studying left--invariant differentials: a $1$--form $f(x) dx$ is said to be left--invariant under $F$\todo{The condition $\omega = T_y^* \omega$ looks more reasonable.}\todo{Also define the \emph{sheaf} of invariant differentials, since you use that in a couple of days.} when \[f(x) dx = f(y +_F x) d(y +_F x) = f(y +_F x) \frac{\partial(y +_F x)}{\partial x} dx.\]  Restricting to the origin by setting $x = 0$, we deduce the condition \[f(0) = f(y) \cdot \left. \frac{\partial(y +_F x)}{\partial x} \right|_{x=0}.\]  If $R$ is a $\Q$--algebra, then setting the boundary condition $f(0) = 1$ and integrating against $y$ yields \[\log_F(y) = \int f(y) \, dy = \int \left( \left. \frac{\partial(y +_F x)}{\partial x} \right|_{x=0} \right)^{-1} dy.\]  To see that the series $\log_F$ has the claimed homomorphism property\todo{While the notation $\log$ is suggestive, I don't think you claimed anything yet at this point.}, note that \[\frac{\partial \log_F(y +_F x)}{\partial x} dx = f(y +_F x) d(y +_F x) = f(x) dx = \frac{\partial \log_F(x)}{\partial x} dx,\] so $\log_F(y +_F x)$ and $\log_F(x)$ differ by a constant.  Checking at $y = 0$ shows that the constant is $\log_F(x)$, hence \[\log_F(x +_F y) = \log_F(x) + \log_F(y).\]  In all, this argument bundles into the following coordinate-free theorem:
\begin{theorem}\label{RationalFGLsHaveLogarithms}\todo{This feels like a bit of a jump.  ``$\otimes$'', for instance?}
There is a unique isomorphism \[\G \xrightarrow{\log} \operatorname{Lie} \G \otimes \G_a. \qed\] \todo{First you have  map $\G \otimes (\Omega^1)^F \to \G_a$, and then you move $(\Omega^1)^F$ to the right, changing it to $\operatorname{Lie} \G$.}
\end{theorem}
\todo{It would also be good to put the example of the standard logarithm for $\G_m$ here.}







\section{Divisors on formal curves}\citeme{Section 2.4.2 of the AHS preprint}

We now have a solid foundation for the most important case of the complex-oriented cohomology of a space: $E^* \CP^\infty$.  We turn next to an algebro-geometric model for the other topological operation complex-oriented cohomology theories are well-suited for: the formation of Thom complexes.  Recall the theorem from the beginning of last time:
\begin{theorem}[{\Cref{ThomIsomOverC}}]
For a complex vector bundle $\xi$ on a space $X$ and a complex-oriented ring spectrum $E$, there is a natural equivalence \[E \sm T(\xi) \simeq E \sm \Susp^\infty_+ X. \qed\]
\end{theorem}
Recalling also the perspective on modules as quasicoherent sheaves from \Cref{HopfAlgebraLecture}, we are thus moved to study sheaves of modules on $X_E$ which are $1$--dimensional --- i.e., line bundles.  Having said all that, we will leave the topology for tomorrow and focus on the algebra today.  We fix the following three pieces of data:
\begin{itemize}
\item $S$ is our ``base'' formal scheme.
\item $C$ is a formal curve over $S$.
\item $\zeta: S \to C$ is a distinguished point on $C$.
\end{itemize}

Recall that yesterday we defined what it meant for a subscheme to be closed.  The notion of a divisor on a formal curve is a particular sort of closed subscheme:

\begin{definition}\citeme{Def 2.33 of AHS preprint}
An \textit{effective Weil divisor} $D$ on $C$ is a closed $S$--subscheme of $C$ whose structure map $D \to S$ is flat and whose ideal sheaf $\sheaf I_D$ is free of rank $1$ as an $\sheaf{O}_S$--module.  We say that the \textit{rank} of $D$ is $n$ when its ring of functions $\sheaf O_D$ is free of rank $n$ over $\sheaf O_S$.
\end{definition}
\todo{Is there a distinction between free and locally free?  I guess not because everything is affine.}

Consider the case of interest to us, where we have selected a coordinate $x$ on $C$.  In that case, there are isomorphisms $S = \Spec E_*$ and $C \cong \Spf E^*\ps{x}$, so that a divisor $D$ must be of the form $D \cong \Spf E^*\ps{x} / f$ for some $f$ not a zero-divisor.  We see then that $\sheaf I_D$ corresponds to the principal ideal $E^*\ps{x} \cdot f \cong E^*\ps{x}$, and $D$ is a divisor exactly when $E^*\ps{x} / f$ is a flat $E^*$--module.\todo{Is this last condition easy to unpack?  I'd hope it means something about monicity.  Maybe see Lemma 17.1 of FPFP?} \todo{How do you even show that $f$ can be chosen to be a polynomial, and not just a power series?  Weierstrass?}
 
Before considering their connection to line bundles, we will study the concept of a divisor in isolation.

\begin{lemma}
The scheme of such effective Weil divisors of rank $n$ exists: $\Div_n^+ C$.  It is a formal variety of dimension $n$.  In fact, a coordinate $x$ on $C$ determines an isomorphism $\Div_n^+ C \cong \A^n$.\todo{I wonder if it's possible to frame this argument with \Cref{DetectingFormalVarieties}.  The proof given here is Prop 5.2 of FSFG.}
\end{lemma}
\begin{proof}
Begin with the definition \[\Div_n^+(C)(R) = \left\{(a, D) \middle| \begin{array}{c} a: \Spec R \to S, \\ \text{$D$ is an effective divisor on $C \times_S \Spec R$} \end{array} \right\}.\]  To show that it is a formal variety, we pick a coordinate $x$ on $C$ and consider a point $(a, D) \in \Div_n^+(C)(R)$.  In this case, $C \times \Spec R$ is presented as \[C \times_X \Spec R = \Spf R\llbracket x \rrbracket\] and hence $D$ can be presented as the closed subscheme \[D = \Spf R\llbracket x \rrbracket / (x^n - g(x)), \quad g(x) = \sum_{j=0}^{n-1} a_j(D) x^j.\]  One checks that $a_j(D)$ is a nilpotent element of $R$ for all $j$\todo{This statement has real content! If $a_j(D)$ were not nilpotent, then Weierstrass factorization would strip off a smaller monic polynomial.  But, we haven't talked about Weierstrass preparation yet... and we were intending to leave it for much later.  Maybe we should have done this today.}\todo{Actually, Jeremy and Jun Hou point out that Weierstrass preparation requires hypotheses on the ground scheme (like: it's $\Spf$ of a complete and local ring) that aren't necessarily satisfied here.  So, what geometric thing do we \emph{really} mean?}, and hence determines a map $\Spec R \to \A^n$.  Conversely, given such a map, we can form the polynomial $g(x)$ and hence the divisor $D$.
\end{proof}

This proof lays bare the moral value of this scheme: it parametrizes collections of points on $C$ which arise as zero loci of polynomials.  It's well-known how basic operations on polynomials affect their zero loci, and these operations are also reflected on the level of divisor schemes.  For instance, there is a unioning map:
\begin{lemma}
There is a map
\begin{align*}
\Div_n^+ C \times \Div_m^+ C & \to \Div_{n+m}^+ C, \\
(D_n, D_m) & \mapsto D_n \sqcup D_m. \qed
\end{align*}
\end{lemma}
\begin{remark}\label{DescriptionOfSqCupMapOnPolynomials}
On the level of the polynomials $g_n$, $g_m$, and $g_{n+m}$, this map is given by \[(g_n, g_m) \mapsto x^{n+m} - (x^n - g_n(x)) \cdot (x^m - g_m(x)) =: g_{n+m}(x).\] \todo{What was wrong with $g_{n+m} := g_n g_m$ from class?  The current $g_{n+m}$ has degree $<n+m$\ldots}
\end{remark}

Note that there is a canonical isomorphism $C \to \Div_1^+ C$.  Iterating the above addition map gives the vertical map in the following triangle:
\begin{center}
\begin{tikzcd}
& C^{\times n} \arrow{ld} \arrow{d} \\
C^{\times n}_{\Sigma_n} \arrow[densely dotted]{r}{\cong} & \Div_n^+ C.
\end{tikzcd}
\end{center}
\begin{lemma}\todo{Make a point that the other arrow is not surjective.}
The object $C^{\times n}_{\Sigma_n}$ exists, it factors the iterated addition map, and the dotted arrow is an isomorphism. 
\end{lemma}
\begin{proof}
The first assertion is a consequence of Newton's theorem on symmetric polynomials: the subring of symmetric polynomials in $R[x_1, \ldots, x_n]$ is itself polynomial on generators \[\sigma_j(x_1, \ldots, x_n) = \sum_{\substack{S \subseteq \{1, \ldots, n\} \\ |S| = j}} x_{S_1} \cdots x_{S_j},\] and hence \[R[\sigma_1, \ldots, \sigma_n] \subseteq R[x_1, \ldots, x_n].\]  Picking a coordinate on $C$ allows us to import this fact into formal geometry to deduce the existence of $C^{\times n}_{\Sigma_n}$.  The factorization then follows by noting that the iterated $\sqcup$ map is symmetric.  Finally, \Cref{DescriptionOfSqCupMapOnPolynomials} shows that the horizontal map pulls the coordinate $a_j$ back to $\sigma_j$, so the third assertion follows.
\end{proof}

We now consider the effects of maps $q\co C \to C'$ between curves.
\begin{lemma}\label{DivHasPushforwards}
Let $q\co C \to C'$ be a map of formal curves over $S$, and let $D \subseteq C$ be a divisor on $C$.  Then the composite $D \to C \to C'$, denoted $q_* D$, is also a divisor.
\end{lemma}
\begin{proof}
The structure map map $D \to S$ is unchanged and hence still flat, and the ideal sheaf $\sheaf I_{q_* D} \subseteq \sheaf O_{C'}$ is given by tensoring up the original ideal sheaf: \[\sheaf I_{q_* D} = \sheaf I_D \otimes_{\sheaf O_C} \sheaf O_{C'}.\]  Hence, it is still free of rank $1$.
\end{proof}

\begin{remark}
For a general map $q$, the pullback $D \times_{C'} C$ of a divisor $D \subseteq C'$ will not be a divisor on $C$.  However, conditions on $q$ can be imposed so that this is so, and in this case $q$ is called an \textit{isogeny}.  We will return to this in the future.
\end{remark}

Now we use the pointing $\zeta: S \to C$.  Together with the $\sqcup$ map, this gives a composite
\begin{center}
\begin{tikzcd}
\Div_n^+ C \arrow{r} & C \times \Div_n^+ C \arrow{r} & \Div_1^+ C \times \Div_n^+ C \arrow{r} & \Div_{n+1}^+ C, \\
D \arrow[|->,r] & (\zeta, D) \arrow[|->,r] & ([\zeta], D) \arrow[|->,r] & {[\zeta] \sqcup D}.
\end{tikzcd}
\end{center}

\begin{definition}\label{StableDivisorSchemeDefn}
We define the following variants of ``stable divisor schemes'':
\begin{align*}
\Div^+ C & = \coprod_{n \ge 0} \Div_n^+ C, \\
\Div_n C & = \colim \left( \Div_n^+ C \xrightarrow{[\zeta] + -} \Div_{n+1}^+ C \xrightarrow{[\zeta] + -} \cdots \right), \\
\Div C & = \colim \left( \Div^+ C \xrightarrow{[\zeta] + -} \Div^+ C \xrightarrow{[\zeta] + -} \cdots \right) \\
& \cong \coprod_{n \in \Z} \Div_n C.
\end{align*}
\end{definition}

\begin{theorem}\label{DivConstructionsAreFree}
The scheme $\Div^+ C$ models the free formal monoid on the formal curve $C$.  The scheme $\Div C$ models the free formal group on the formal curve $C$.\footnote{That is, the group-completion of $\Div^+ C$ gives $\Div C$, even in absence of a pointing on $C$.}  The scheme $\Div_0 C$ \emph{simultaneously} models the free formal monoid and the free formal group on the \emph{pointed} formal curve $C$. \qed \todo{Can you at least see where the negation maps would come from w/o reference to coalgebras? (See FPFP Prop 18.1?)}
\end{theorem}
\noindent We will postpone the proof of this theorem until later, once we've developed a theory of coalgebraic formal schemes.\oweproof{The various $\Div$ constructions give free monoids / groups}

\begin{remark}
This gives another way to interpret \Cref{DivHasPushforwards}.  A map $q\co C \to C'$ postcomposes to give a map $C \to C' \to \Div C'$.  Since the target of this map is a formal group scheme, universality induces a map $q_*\co \Div C \to \Div C'$.
\end{remark}

To close today, we finally link divisors to the study of line bundles.

\begin{definition}\citeme{Def 2.38 of AHS preprint}
Suppose that $\L$ is a line bundle on $C$ and select a section $u$ of $\L$.  There is a largest closed subscheme $D \subseteq C$ where the condition $u|_D = 0$ is satisfied.  If $D$ is a divisor, $u$ is said to be \textit{divisorial} and $D = \div u$.\todo{Is being divisorial much of a condition at all? The zero section is not divisorial --- are there other sections of other line bundles which are not divisorial?  (Maybe if the base scheme is sufficiently nasty, then you can build things that literally aren't this counterexample, but which are close enough and ubiquitous enough as to be hard to avoid by naming properties, so we invent a new name...)}
\end{definition}

\begin{lemma}\citeme{Def 2.38 of AHS preprint}
Let $u$ be a divisorial section of $\L$.  Then, $u$ gives a trivialization of $\L \otimes \sheaf I_D$, so that $\L \cong \sheaf I_D^{-1}$. \qed
\end{lemma}

\begin{lemma}
This construction is suitably monoidal: if $u$ and $v$ are divisorial sections of $\sheaf L$ and $\sheaf M$ respectively, then $u \otimes v$ is a divisorial section of $\sheaf L \otimes \sheaf M$ and $\div(u \otimes v) = \div u + \div v$. \qed
\end{lemma}

This Lemma induces us to consider the extension of this concept to meromorphic functions:
\begin{definition}
A \textit{meromorphic divisorial section} of a line bundle $\sheaf L$ is a decompositon $\sheaf L \cong \sheaf L_1 \otimes \sheaf L_2^{-1}$ together with an expression of the form $u_+ / u_-$, where $u_+$ and $u_-$ are divisorial sections of $\sheaf L_1$ and $\sheaf L_2$ respectively.  We set $\div(u_+ / u_-) = \div u_+ - \div u_-$.
\end{definition}

The fundamental theorem is that, in the case of a curve $C$, meromorphic functions (sometimes called ``Cartier divisors'') and Weil divisors essentially agree.

\begin{definition}
The ring of meromorphic functions on $C$, $\sheaf{M}_C$, is obtained by inverting all coordinates in $\sheaf{O}_C$.\footnote{In fact, it suffices to invert any single one.}
\end{definition}

A particular meromorphic function spans a $1$--dimensional $\sheaf O_C$--submodule sheaf of $\sheaf M_C$, and hence it determines a line bundle.  Conversely, a line bundle is determined by local gluing data, which is exactly the data of a meromorphic function.  However, it is clear that there is some overdeterminacy in this first operation: scaling a meromorphic function by a nowhere vanishing entire function will not modify the submodule sheaf.  This suggests the following operation: to a meromorphic function $f$, we assign the difference of its zero locus and its infinite locus, considered as a divisor.  This determines a map \[\sheaf M_C^\times \to (\Div C)(S).\]

\begin{definition}
We then augment this to a scheme $\operatorname{Mer}(C, \mathbb G_m)$ of meromorphic functions on $C$ by \[\operatorname{Mer}(C, \mathbb G_m)(R) := \left\{ (u, f) \middle| \begin{array}{c} u: \Spec R \to S, \\ f \in \sheaf{M}^\times_{C \times_S \Spec R} \end{array} \right\}.\]
\end{definition}

\begin{theorem}\citeme{This is Prop 5.26 in FSFG.}
In the case of a formal curve $C$, there is a short exact sequence of formal groups \[0 \to \InternalHom{FormalSchemes}(C, \mathbb G_m) \to \operatorname{Mer}(C, \mathbb G_m) \to \Div(C) \to 0. \qed\]
\end{theorem}







\section{Projectivization and Thom spaces}\citeme{Section 8 of the $H_\infty$ AHS paper.}
\todo{This title needs improvement.}
\todo{You are not consistent about calling vector bundles $V$ or $\xi$.}

Today we will exploit all of the algebraic geometry we set up yesterday to deduce a load of topological results.  

\begin{definition}
Let $E$ be a complex-orientable theory and let $V \to X$ be a complex vector bundle over a space $X$.  According to \Cref{ThomIsomOverC}, the cohomology of the Thom space $E^* T(V)$ forms a $1$--dimensional $E^* X$--module.  We denote the associated line bundle over $X_E$ by $\ThomSheaf{V}$.
\end{definition}

This construction enjoys many properties already established.
\begin{corollary}\citeme{Make backreferences.} \todo{this corollary seems to be stated in the wrong order, as the first half uses virtual bundles which are defined in the second half. AY}
If $f \co X \to Y$ is a map and $V$ is a virtual bundle over $Y$, then there is an isomorphism \[\ThomSheaf{f^* V} \cong (f_E)^* \ThomSheaf{V}.\]  Using \Cref{ThomSpacesAreMonoidal}, there is also is a canonical isomorphism \[\ThomSheaf{V \oplus W} = \ThomSheaf{V} \otimes \ThomSheaf{W}.\]  Finally, this property can then be used to extend the definition of $\ThomSheaf{V}$ to virtual bundles: \[\ThomSheaf{V - W} = \ThomSheaf{V} \otimes \ThomSheaf{W}^{-1}. \qed\]
\end{corollary}

\begin{remark}
One of the main utilities of this definition is that it only uses the \emph{property} that $E$ is complex-orientable, and it begets only the \emph{property} that $\ThomSheaf{V}$ is a line bundle.
\end{remark}

The following example connects this topic with that of \Cref{FormalVarietiesLecture}:
\begin{example}\label{Pi2AndInvariantDiffls}
\todo{Make this clearer: point out that the cofiber definition of the Thom space helps you make the first identification (that's what the ``zero section'' refers to), and then point out that $\zeta^*(M) = M / \sheaf I(0) \cdot M$ helps you over the last hump.}\todo{In class you wrote $\L - 1$ instead of $\L$ almost everywhere in this example.} 
If $\L$ denotes the canonical line bundle over $\CP^\infty$, then the zero section identifies $E^0 (\CP^\infty)^{\L}$ \footnote{What does this notation mean?  I would guess it's something like maps $\L \to E^0(\CP^\infty)$, but this doesn't seem to make sense.}with the augmentation ideal in $E^0 \CP^\infty$, and so we have an isomorphism $\ThomSheaf{\L} \cong \sheaf I(0)$.  Then, consider the map $\eps: * \to \CP^\infty$, which classifies a line bundle that Thomifies to $\CP^1 \to \CP^\infty$.  Using naturality, we see \todo{I think you should mention that the tilde means the sheaf associated to the module.  Also, in class you wrote $\pi_{-2} E$ instead of $\pi_2 E$}\[\widetilde{\pi_2 E} \cong \ThomSheaf{* \to \CP^\infty} \cong 0^* \sheaf I(0) \cong \omega_{\G_E},\] where $\G_E = \CP^\infty_E$ is the formal group associated to $E$--theory and $\omega_{\G_E}$ is its sheaf of invariant differentials\footnote{The identification of this with the sheaf of invariant differentials is something of a choice.  Certainly it is naturally isomorphic to $T_0^* \CP^\infty_E$, and this in turn is naturally isomorphic to $\omega_{\G_E}$, but deciding which of these two to write is a decision to be borne out as ``correct''.}.  More generally, if $k \eps$ \todo{It might be confusing using $\eps$ for two different things in the same paragraph.} is the trivial bundle of dimension $k$ over a point, then $\ThomSheaf{k \eps} \cong \omega_{\G_E}^{\otimes k}$.  If $f \co E \to F$ is an $E$--algebra (e.g., $F = E^{X_+}$), then this gives an interpretation of $\pi_{2k} F$ as $f_E^* \omega_{\G_E}^{\otimes k}$.
\end{example}

Aside from this example, though, this construction on its own does not allow for the ready manipulation of line bundles.  However, our discussion yesterday centered on an equivalent presentation of line bundles on a formal curve: their corresponding divisors.  Following that cue, we now seek out a topological construction on vector bundles $V \to X$ which produces finite schemes over $X_E$.  A quick browse through the literature will lead one to the following:
\begin{definition}
Let $\xi$ be a complex vector bundle of rank $n$ over a base $X$.  Define $\P(\xi)$, the \textit{projectivization of $\xi$}, to be the $\CP^{n-1}$--bundle over $X$ whose fiber of $x \in X$ is the space of complex lines in the original fiber $\xi|_x$.
\end{definition}

\begin{theorem}\label{CohomologyOfProjectivization}
Take $E$ to be complex-oriented.  The $E$--cohomology of $\P(\xi)$ is given by the formula \[E^* \P(\xi) \cong \left. E^*(X) \llbracket t \rrbracket \middle/ c(\xi) \right.\] for a certain monic polynomial \[c(\xi) = t^n - c_1(\xi) t^{n-1} + c_2(\xi) t^{n-2} - \cdots + (-1)^n c_n(\xi).\]
\end{theorem}
\begin{proof}
We fit all of the fibrations we have into a single diagram:
\begin{center}
\begin{tikzcd}
& \C^\times \arrow[-,double]{dd} \arrow{rd} \\
\C^n \arrow{dd} & & \C^n \setminus \{0\} \arrow[crossing over]{ll} \arrow{r} \arrow{dd} & \CP^{n-1} \arrow{r} \arrow{dd} & \CP^\infty \arrow[-,double]{dd} \\
& \C^\times \arrow{rd} \\
\xi \arrow{d} & & \xi \setminus \zeta \arrow{ll} \arrow{r} \arrow{d} & \P(\xi) \arrow{r} \arrow{d}{\pi} & \CP^\infty \arrow{d} \\
X \arrow[-,double]{rr} \arrow[bend left, densely dotted]{u}{\zeta} & & X \arrow[-,double]{r} & X \arrow{r} & *.
\end{tikzcd}
\end{center}
We read this diagram as follows: on the far left, there's the vector bundle we began with, as well as its zero-section $\zeta$.  Deleting the zero-section gives the second bundle, a $\C^n \setminus \{0\}$--bundle over $X$.  Its quotient by the scaling $\C^\times$--action gives the third bundle, a $\CP^{n-1}$--bundle over $X$.  Additionally, the quotient map $\C^n \setminus \{0\} \to \CP^{n-1}$ is itself a $\C^\times$--bundle, and this induces the structure of a $\C^\times$--bundle on the quotient map $\xi \setminus \zeta \to \P(\xi)$.  Thinking of these as complex line bundles, they are classified by a map to $\CP^\infty$, which can itself be thought of as the last vertical fibration, fibering over a point.

Note that the map between these two last fibers is surjective on $E$--cohomology.  It follows that the Serre spectral sequence for the third vertical fibration is degenerate, since all the classes in the fiber must survive.\footnote{This is called the Leray--Hirsch theorem.}  We thus conclude that $E^* \P(\xi)$ is a free $E^*(X)$--module on the classes $\{1, t, t^2, \ldots, t^{n-1}\}$ spanning $E^* \CP^{n-1}$.  To understand the ring structure, we need only compute $t^{n-1} \cdot t$, which must be able to be written in terms of the classes which are lower in $t$--degree: \[t^n = c_1(\xi) t^{n-1} - c_2(\xi) t^{n-2} + \cdots + (-1)^{n-1} c_n(\xi)\] for some classes $c_i(\xi) \in E^* X$.  The theorem follows.
\end{proof}

In coordinate-free language, we have the following Corollary:
\begin{corollary}[{\Cref{CohomologyOfProjectivization} redux}]
Take $E$ to be \emph{complex-orientable}.  The map \[\P(\xi)_E \to X_E \times \CP^\infty_E\] is a closed inclusion of $X_E$--schemes, and the structure map $\P(\xi)_E \to X_E$ is free and finite of rank $n$.  It follows that $\P(\xi)_E$ is a divisor on $\CP^\infty_E$ (considered over $X_E$). \qed \todo{Be more careful about this ``over $X_E$'' thing. Maybe just emphasize that having a Chern polynomial with coefficients in $E^* X$ really forces you to take this perspective to make things typecheck.}
\end{corollary}

The next major theorems concerning projectivization are the following:

\begin{corollary}
The sub-bundle of $\pi^*(\xi)$ consisting of vectors $(v, (\ell, x))$ such that $v$ lies along the line $\ell$ splits off a canonical line bundle. \qed
\end{corollary}

\begin{corollary}[``Splitting principle'' / ``Complex--oriented descent'']\label{OriginalSplittingPrinciple}
Associated to any $n$--dimensional complex vector bundle $\xi$ over a base $X$, there is a canonical map $i_\xi\co Y_\xi \to X$ such that $(i_\xi)_E\co (Y_\xi)_E \to X_E$ is finite and faithfully flat, and there is a canonical splitting into complex line bundles: \[i_\xi^*(\xi) \cong \bigoplus_{i=1}^n \L_i. \qed\]
\end{corollary}

This last Corollary is extremely important.  Its essential contents is to say that any question about characteristic classes can be checked for sums of line bundles.  Specifically, because of the injectivity of $i_\xi^*$, any relationship among the characteristic classes deduced in $E^* Y_\xi$ must already be true in the ring $E^* X$.  The following theorem is a consequence of this principle:

\begin{theorem}\todo{Proposition 8.31 in FSFG shows that the isomorphism $BU(n)_E \cong \Div_n^+ \CP^\infty_E$ is independent of coordinate. Read it.}
Again take $E$ to be complex-oriented.  The coset fibration \[U(n-1) \to U(n) \to S^{2n-1}\] deloops to a spherical fibration \[S^{2n-1} \to BU(n-1) \to BU(n).\]  The associated Serre spectral sequence \[E_2^{*, *} = H^*(BU(n); E^* S^{2n-1}) \Rightarrow E^* BU(n-1)\] degenerates at $E_{2n}$ and induces an isomorphism \[E^* BU(n) \cong E^* \llbracket \sigma_1, \ldots, \sigma_n\rrbracket.\] \todo{Shouldn't this be a polynomial ring instead of a power series ring?  Or are you considering the periodified version $EP^0 BU(n)$?} Now, let $\xi\co X \to BU(n)$ classify a vector bundle $\xi$.  Then the coefficient $c_j$ in the polynomial $c(\xi)$ is selected by $\sigma_j$: \[c_j(\xi) = \xi^*(\sigma_j).\]
\end{theorem}
\begin{proof}[Proof sketch]
The first part is a standard calculation.  To prove the relation between the Chern classes and the $\sigma_j$, the splitting principle states that we can factor complete the map $\xi\co X \to BU(n)$ to a square
\begin{center}
\begin{tikzcd}
Y_\xi \arrow{d}{f_\xi} \arrow[densely dotted]{r}{\bigoplus_{i=1}^n \L_i} & BU(1)^{\times n} \arrow{d}{\oplus} \\
X \arrow{r}{\xi} & BU(n).
\end{tikzcd}
\end{center}
The equation $c_j(f_\xi^* \xi) = \xi^*(\sigma_j)$ can be checked in $E^* Y_\xi$.
\end{proof}

We now see that not only does $\P(\xi)_E$ produce a point of $\Div_n^+(\G_E)$, but actually the scheme $\Div_n^+(\G_E)$ itself appears internally to topology:

\begin{corollary}\label{IdentificationOfBUnWithDivn}
For a complex orientable cohomology theory $E$, there is an isomorphism \[BU(n)_E \cong \Div_n^+ \CP^\infty_E,\] so that maps $\xi\co X \to BU(n)$ are transported to divisors $\P(\xi)_E \subseteq \CP^\infty_E \times X_E$.  Selecting a particular complex orientation of $E$ begets two isomorphisms
\begin{align*}
BU(n)_E & \cong \A^n, &
\Div_n^+ \CP^\infty_E & \cong \A^n,
\end{align*}
and these are compatible with the centered isomorphism above.\footnote{Something to take away from this Theorem is the \emph{faithfulness} of this interpretation of the $E$--cohomology of vector bundles.  That this map is an isomorphism means that $\Div_n^+$ captures everything that $E$--cohomology can see.  There's nothing left in the theory of characteristic classes that is left untouched.} \qed
\end{corollary}

What's most remarkable about the description in this theorem is its coherence with topological facts we know about $BU(n)$.  The theorem follows from the projectivization construction, but there are natural operations on both sides of the isomorphism that continue to match up.  For instance, the Whitney sum map $BU(n) \times BU(m) \to BU(n+m)$ has the following behavior:

\begin{lemma}\label{WhitneySumOfDivisors}
The sum map \[BU(n) \times BU(m) \xrightarrow\oplus BU(n+m)\] induces on Chern polynomials the identity \[c(\xi \oplus \zeta) = c(\xi) \cdot c(\zeta).\]  In terms of divisors, \[\P(\xi \oplus \zeta)_E = \P(\xi)_E \sqcup \P(\zeta)_E,\] and hence there is an induced square
\begin{center}
\begin{tikzcd}
BU(n)_E \times BU(m)_E \arrow{r}{\oplus} \arrow[-,double]{d} & BU(n+m) \arrow[-,double]{d} \\
\Div_n^+ \CP^\infty_E \times \Div_m^+ \CP^\infty_E \arrow{r}{\sqcup} & \Div_{n+m}^+ \CP^\infty_E. \qed
\end{tikzcd}
\end{center}
\end{lemma}

The following is a consequence of combining this Lemma with the splitting principle:

\begin{corollary}
The map $Y_E \xrightarrow{f_\xi} X_E$ pulls $\P(\xi)_E$ back to give \[Y_E \times_{X_E} \P(\xi)_E \cong \bigoplus_{i=1}^n \left\{ c_1(\L_i) \right\}. \qed\]
\end{corollary}

\noindent This says that the splitting principle is a topological enhancement of the claim that a divisor can be base-chnaged along a finite flat map where it splits as a sum of points.  The other theorems from yesterday are also easily matched up with topological counterparts:

\begin{corollary}\todo{Draw a table comparing the different notions of vector bundles (stable vs unstable, rank $n$ vs virtual rank $n$) to the different notions of Weil divisors.}
There are natural isomorphisms $BU_E \cong \Div_0 \CP^\infty_E$ and $(BU \times \Z)_E \cong \Div \CP^\infty_E$. Additionally, $(BU \times \Z)_E$ is the free formal group on the curve $\CP^\infty_E$. \qed
\end{corollary}

\begin{corollary}
There is a commutative diagram
\begin{center}
\begin{tikzcd}
BU(n)_E \times BU(m)_E \arrow{r}{\otimes} \arrow[-,double]{d} & BU(nm)_E \arrow[-,double]{d} \\
\Div_n^+ \CP^\infty_E \times \Div_m^+ \CP^\infty_E \arrow{r}{\cdot} & \Div_{nm}^+ \CP^\infty_E,
\end{tikzcd}
\end{center}
where the bottom map acts by \[(D_1, D_2 \subseteq \CP^\infty_E \times X_E) \mapsto (D_1 \times D_2 \subseteq \CP^\infty_E \times \CP^\infty_E \xrightarrow{\mu} \CP^\infty_E),\] and $\mu$ is the map induced by the $H$-space multiplication $\CP^\infty \times \CP^\infty \to \CP^\infty$.
\end{corollary}
\begin{proof}
By the splitting principle, it is enough to check this on sums of line bundles.  A sum of line bundles corresponds to a totally decomposed divisor, and on a pair of such divisors $\bigsqcup_{i=1}^n [a_i]$ and $\bigsqcup_{j=1}^m [b_j]$, the map acts by \[\left(\bigsqcup_{i=1}^n [a_i] \right) \left( \bigsqcup_{j=1}^m [b_j] \right) = \bigsqcup_{i, j} [\mu_{\CP^\infty_E}(a_i, b_j)]. \qedhere\]
\end{proof}

Finally, we can connect our analysis of the divisors coming from topological vector bundles with the line bundles studied at the start of the section.
\begin{lemma}
Let $\zeta: X_E \to X_E \times \CP^\infty_E$ denote the pointing of the formal curve $\CP^\infty_E$, and let $\sheaf I(\P(\xi)_E)$ denote the ideal sheaf on $X_E \times \CP^\infty_E$ associated to the divisor subscheme $\P(\xi)_E$.  There is a natural isomorphism of sheaves over $X_E$: \[\zeta^* \sheaf I(\P(\xi)_E) \cong \ThomSheaf{\xi}. \qed\]
\end{lemma}

\begin{remark}
In terms of a complex-oriented $E$ and \Cref{CohomologyOfProjectivization}, the effect of pulling back along the zero section is to set $t = 0$, which collapses the Chern polynomial to just the top class $c_n(\xi)$.  This element, called \textit{the Euler class of $\xi$}, provides the $E^* X$--module generator of $E^* T(\xi)$ --- or, equivalently, the trivializing section of $\ThomSheaf{\xi}$.
\end{remark}

\begin{theorem}\label{ComplexOrientationsInTermsOfTrivs}
A trivialization $t\co \ThomSheaf{\L - 1} \cong \sheaf O_{\CP^\infty_E}$ of the Thom sheaf associated to the canonical bundle induces a ring map $MU \to E$.
\end{theorem}
\begin{proof}
Suppose that $\xi$ is a rank $n$ vector bundle over $X$, and let $f\co Y \to X$ be the space guaranteed by the splitting principle to provide an isomorphism $f^* \xi \cong \bigoplus_{j=1}^n \L_j$.  The chosen trivialization $t$ then pulls back to give a trivialization of $\sheaf I(\P(f^* \xi)_E)$, and by finite flatness this descends to also give a trivialization of $\sheaf I(\P(\xi)_E)$.  Pulling back along the zero section gives a trivialization of $\ThomSheaf{\xi}$.  Then note that the system of trivializations produced this way is multiplicative, as a consequence of $\P(\xi \oplus \zeta)_E \cong \P(\xi)_E \sqcup \P(\zeta)_E$.  In the universal examples, this gives a sequence of compatible maps $MU(n) \to E$ which assemble on the colimit $n \to \infty$ to give the desired map of ring spectra.
\end{proof}



\todo{I think another definition of the Thom space is as the cofiber of $\P(V) \to \P(V \oplus \C)$.  This might come in handy.}

\todo{Say that the top Chern class is the Euler class / the Thom class.}









\section{Operations and a model for cobordism}

Our eventual goal, like in \Cref{UnorientedBordismChapter}, is to give an algebro-geometric description of $MU_*(*)$ and of the cooperations $MU_* MU$.  There is such a description that passes through the Adams spectral sequence, also like last time, but $MU_*(*)$ is an integral algebra and so we cannot make do with working out the mod--$2$ Adams spectral sequence alone.  We would have to at least work out the mod--$p$ Adams spectral sequence for every $p$, but there is the following unfortunate theorem:
\begin{theorem}
There is an isomorphism
\[H\F_pP_0 H\F_pP \cong \F_p[\xi_0^\pm, \xi_1, \xi_2, \ldots] \otimes \Lambda[\tau_0, \tau_1, \ldots]\]
with $|\xi_j| = 2p^j-2$ and $|\tau_j| = 2p^j - 1$. \qed
\end{theorem}
\noindent There are odd--dimension classes in this algebra, and because we are no longer working in characteristic $2$ we see that the dual mod--$p$ Steenrod algebra is \emph{graded-commutative}.  This is the first time we have encountered Hindrance \#\ref{SkewCommutativeDeficiency} from \Cref{TheSteenrodAlgebraSection} in the wild, and for now we will simply avoid these methods and find another approach.

There is such an alternative proof, due to Quillen, that bypasses the Adams spectral sequence.  This approach has some deficiencies of its own: it requires studying the algebra of operations $MU^* MU$, which we do not expect to be at all commutative, and it requires studying \textit{power operations}, which are in general very technical creatures.  However, we will eventually want to talk about power operations anyway, and because this is the road less traveled we will elect to take it.  Our job today is to define these two kinds of cohomology operations, as well as revisit the model of complex cobordism Quillen uses.

The description of the first class of operations follows immediately from our discussion of complex cobordism up to this point, so we will begin there.  We learned in \Cref{IdentificationOfBUnWithDivn} that for any complex-oriented cohomology theory $E$ we have the calculation \[E^* BU \cong E^*\llbracket \sigma_1, \sigma_2, \ldots, \sigma_j, \ldots\rrbracket,\] and we gave a rich interpretation of this in terms of divisor schemes: \[BU_E \cong \Div_0 \CP^\infty_E.\]  Two lectures ago, we learned that the stable divisor scheme has a universal property: it is the free formal group on the formal curve $\CP^\infty_E$.  Another avatar of this same fact is a description of the \emph{homology ring}, using the maps \[E_* BU(n) \otimes E_* BU(m) \to E_* BU(n+m)\] to induce a multiplicative structure on $E_* BU$:
\begin{corollary}
Let $E$ be a complex-orientable cohomology theory. Then: \[E_* BU \cong \Sym_{E_*} \widetilde E_* \CP^\infty.\]  A specific complex orientation of $E$ begets \[E_* \CP^\infty \cong E_*\{\beta_0, \beta_1, \ldots, \beta_n, \ldots\}\] and hence \[E_* BU \cong \Sym_{E_*} E_*\{\beta_1, \beta_2, \ldots\} = E_*[b_1, b_2, \ldots]. \qed\]
\end{corollary} \oweproof{Free formal schemes agree with symmetric Hopf algebras on comodules}

Thomifying these ``$\oplus$'' maps gives maps \[E_* MU(n) \otimes E_* MU(m) \to E_* MU(n+m),\] and the naturality of the $E$--Thom isomorphism produces an additional corollary:
\begin{corollary}
The Thom isomorphism $E_* BU \cong E_* MU$ respects both the $E_*$--module structure and the ring structure.  Hence, \[E_* MU \cong E_*[c_1, c_2, \ldots, c_n, \ldots],\] where $c_j$ is the image of $b_j$ under the Thom map. \qed
\end{corollary}

\todo{A corollary of the splitting principle is supposed to be that a Thom isomorphism for $\CP^\infty$ begets Thom isomorphisms for everything, and hence a ring spectrum map $MU \to E$.  We should produce that corollary now.  This is Lemma II.4.6 in Adams's blue book.}

\noindent This compact description of $E_* MU$ as an algebra will be useful to us later, but right now we are interested in $E^* MU$ and especially in $MU^* MU$.  The former is \emph{not} a ring\todo{Does this totally prohibit us from giving a formal group re-exposition of Quillen's proof?  I wonder...}, and although the latter is a ring its multiplication is exceedingly complicated.  Instead, we will content ourselves with an $E_*$--module basis:
\begin{definition}
Let $\alpha = (\alpha_1, \ldots, \alpha_n, \ldots)$ denote a multi-index where every entry is nonnegative and almost every entry is zero, and let $c_\alpha$ denote the corresponding monomial \[c_\alpha = \prod_{j=1}^\infty c_j^{\alpha_j}.\]  Additionally, we let $s_\alpha \in E^* MU$ denote the image of $c_\alpha$ under the duality isomorphism \[E^* MU = \CatOf{Modules}_{E_*}(E_* MU, E_*).\]  It is called the \textit{$\alpha${\th} Landweber--Novikov operation} (from $MU$ to $E$).
\end{definition}

\begin{remark}\citeme{I.5.1 in Adams's blue book}
Let $E = MU$.  The Landweber--Novikov operations are the \emph{stable} operations acting on $MU$--cohomology, analogous to the Steenrod operations we started the semester talking about.  They satisfy the following properties:
\begin{itemize}
\item $s_0$ is the identity.
\item $s_\alpha$ is natural: $s_\alpha(f^* x) = f^*(s_\alpha x)$.
\item $s_\alpha$ is stable: $s_\alpha(\sigma x) = \sigma(s_\alpha x)$.
\item $s_\alpha$ is additive: $s_\alpha(x + y) = s_\alpha(x) + s_\alpha(y)$.
\item $s_\alpha$ satisfies a Cartan formula.  Define \[s_{\t}(x) = \sum_\alpha s_\alpha(x) \cdot t_1^{\alpha_1} t_2^{\alpha_2} \cdots t_n^{\alpha_n} \cdots =: \sum_{\alpha} s_\alpha(x) \t^\alpha\] for an infinite sequence of indeterminates $t_1$, $t_2$, \ldots.  Then: \[s_{\t}(x y) = s_{\t}(x) \cdot s_{\t}(y).\]
\item Let $\xi\co X \to BU(n)$ classify a vector bundle and let $\phi$ denote the Thom isomorphism \[\phi\co MU^* X \to MU^* T(\xi).\]  Then the Chern classes of $\xi$ are related to the Landweber--Novikov operations on the Thom spectrum by the formula \[\sum_\alpha \phi c_\alpha(\xi) \t^\alpha = \sum_\alpha s_\alpha \phi(1) \t^\alpha.\]
\end{itemize}
\end{remark}

\todo[inline]{Jeremy Hahn, following Rudyak, produced a proof of the incidence relation which doesn't rely on this (particular) geometric model of complex bordism.  His write-up of the $p = 2$ case is elsewhere in the repository.  The end of this lecture and all of the next one should be reworked to use this other perspective!  Manifolds are gross.}

We now turn to the construction of the other cohomology operations we will be interested in: the power operations.  Power operations get their name from their \emph{multiplicative} properties, and correspondingly we do not (\textit{a priori}) expect them to be additive operations, so they are quite distinct from the Landweber--Novikov operations.  Power operations arise from ``$E_\infty$'' structures on ring spectra\footnote{Or, by some accounts, ``$H_\infty$'' structures.}, but most such structures arise in nature from geometric models of cohomology theories.  To produce them for complex cobordism, we will use a particular model, alluded to in \Cref{IntroductionSection}.

\todo[inline]{It is very annoying that you tend to switch $f$, $i$, and $j$; $X$, $Y$, and $Z$; what is attached to what; and what is drawn in what direction.  You'd do well to standardize this.}

\begin{definition}
Let $f: Y \to X$ be a map of manifolds.  A \textit{complex-orientation on the map $f$} is the data of a factorization
\begin{center}
\begin{tikzcd}
 & E \arrow{d} \\
Y \arrow{r}{f} \arrow[densely dotted]{ru}{i} & X
\end{tikzcd}
\end{center}
through a complex vector bundle $E$ on $X$ such that $i$ is an embedding and its normal bundle $\nu_i$ has a complex structure.\todo{Account for the odd-dimensional case and the dimension-jumping case.}  Two such factorizations are \textit{equivalent} when they appear as subbundles of a larger bundle and the embeddings are isotopic, compatibly with the structures on their normal bundles.
\end{definition}
\begin{lemma}
For $\dim E \gg 0$, this equivalence class is unique, if it exists. \qed
\end{lemma}
\begin{definition}
Two complex-oriented maps $f_0\co Y_0 \to X$ and $f_1\co Y_1 \to X$ are called \textit{cobordant} when there is a complex-oriented map $W \to X \times \R$ and elements $b_0, b_1 \in \R$ such that
\begin{center}
\begin{tikzcd}
Y_0 \arrow{r} \arrow{d} & X \times \{b_0\} \arrow{d} & Y_1 \arrow{r} \arrow{d} & X \times \{b_1\} \arrow{d} \\
W \arrow{r} & X \times \R & W \arrow{r} & X \times \R
\end{tikzcd}
\end{center}
become pull-back squares of complex-oriented maps of manifolds.
\end{definition}

\begin{theorem}[Thom]\citeme{This is cited as [Tho51] in Matt's thesis}
For a manifold $X$, $MU^{-q}(X)$ is canonically isomorphic to the cobordism classes of complex-oriented maps of dimension $q$. \qed \todo{Remark that this, as expected, puts the cobordism ring into negative degrees.}
\end{theorem}

\begin{remark}
This model has a variety of nice features.  For instance, its two variances are visible from the construction.  For a map $g\co X' \to X$, there is an induced map $g^*\co MU^* X \to MU^* X'$ given by selecting a class $f\co Y \to X$, perturbing $g$ so that it is transversal to $f$, and taking the pullback
\begin{center}
\begin{tikzcd}
Y \times_X X' \arrow{r} \arrow{d} & Y \arrow{d} \\
X' \arrow{r} & X.
\end{tikzcd}
\end{center}
But, also, if $g$ is additionally proper and complex-orientable, then it induces a map \[g_*\co MU^q X' \to MU^{q-d} X,\] where $d$ is the dimension of $g$.  This is simply by postcomposition: a representative $f'\co Y' \to X'$ begets a new representative $g_* f' = g \circ f'$.  This construction goes by various names: the \textit{Gysin map}, the \textit{complex-oriented pushforward}, the \textit{shriek map}, \ldots.
\end{remark}

Additionally, these push and pull maps are related:
\begin{lemma}\label{PushPullFormulaForMU}
Consider a Cartesian square of manifolds
\begin{center}
\begin{tikzcd}
Y \times_X Z \arrow{r}{g'} \arrow{d}{f'} & Z \arrow{d}{f} \\
Y \arrow{r}{g} & X,
\end{tikzcd}
\end{center}
where $g$ is transversal to $f$, $f$ is proper and complex-oriented, and $f'$ is endowed with the pull-back of the complex orientation of $f$. Then \[g^* f_* = f'_* (g')^*\co h(Z) \to h(Y). \qed\]
\end{lemma}


We are now in a position to describe the power operations.
\begin{definition}\label{DefnPowerOperationForMU}
Consider a class in $MU^{-2q}(X)$ represented by a proper complex-oriented map $f \co Y \to X$.  Its $n$--fold Cartesian product determines a class $f^{\times n} \co Y^{\times n} \to X^{\times n}$, and taking the homotopy quotient by a group $G$ acting on $\{1, \ldots, n\}$ gives a class \[Y^{\times n} \to X^{\times n} \to EG \times_G X^{\times n}\] and hence an \textit{external power operation} \[P^{\mathrm{ext}} \co MU^{-2q}(X) \to MU^{-2qn}(EG \times_G X^{\times n}).\]  Pulling back along the diagonal $\Delta: X \to X^{\times n}$ gives the the \textit{internal power operation} \[P \co MU^{-2q}(X) \to MU^{-2qn}(BG \times X).\]  Its action on the class represented by a proper complex-oriented even-dimensional map $f\co Z \to X$ can also be written as \[P(f_* 1) = \Delta^* f^{\times n}_{hG}{}_* 1.\]
\end{definition}

\begin{remark}
It's apparent that we've really needed this geometric model to accomplish this construction: we needed to understand how to take Cartesian powers of maps in a way that inherited a $G$--action.  This is not data that a ring spectrum is naturally equipped with, and if we were to tease out exactly what extra information we need to encode this operation, we would eventually arrive at the notion of an $E_\infty$--ring spectrum.
\end{remark}

\begin{remark}
A picky reader will (rightly) point out that $BG$ is not a manifold, and so we shouldn't be mixing it with out geometric model for $MU$.  This is a fair point, but since $BG$ can be approximated through any cellular dimension by a manifold, we won't worry about it.
\end{remark}

\begin{remark}
The chain model for ordinary homology is actually rigid enough to define power operations there, too.  Curiously, they are all generated by the quadratic power operations (i.e., the ``squares''), and all the quadratic power operations turn out to be \emph{additive} --- that is, you just get the Steenrod squares again!  This appears to be a lucky degeneracy, but tomorrow we will exploit something very similar with a particular power operation in complex cobordism.
\end{remark}

\todo{Can we name some of the formal properties of power operations? Multiplicativity, say?}








\section{An incidence relation among operations}

\todo{Danny pointed out that this is a little confused about fixed points versus orbits and homotopy vs genuine.  Make sure this is straightened out.}
\todo{It would be nice if all the Cartesian diagrams in this section were typeset with the little pullback corners.}

Our goal today is to apply a version of \Cref{PushPullFormulaForMU} to the push-pull definition of the power operation for $MU$ given in \Cref{DefnPowerOperationForMU}.  The relevant Cartesian square in that case has the form
\begin{center}
\begin{tikzcd}
W \arrow{r} \arrow{d}{g} & EG \times_G Y^{\times k} \arrow{d}{f^{\times k}_{hG}} \\
BG \times X \arrow{r}{\Delta} & EG \times_G X^{\times k}.
\end{tikzcd}
\end{center}
However, since we have so little control over vertical map $f^{\times k}_{hG}$, we can't rely on the other hypotheses of \Cref{PushPullFormulaForMU} to be satisfied.  So, we investigate the following slightly more general situation.

\begin{definition}
Let $X$ be a manifold.  Two closed submanifolds $Y$ and $Z$ are said to \textit{intersect cleanly} when $W = Y \cap Z$ is a submanifold and for each $w \in W$, the tangent space of $W$ at $w$ is given by $T_w W = T_w Y \cap T_w X$.  In this case, we draw a Cartesian square
\begin{center}
\begin{tikzcd}
W \arrow{r}{j'} \arrow{d}{i'} & Z \arrow{d}{i} \\
Y \arrow{r}{j} & X.
\end{tikzcd}
\end{center}
The \textit{excess bundle} of the intersection, $F$, is defined by the exact sequence\todo{We had to stare at this in class to decide that it was reasonable.}
\begin{center}
\begin{tikzcd}
& & (i')^* TY \arrow{rd} \\
0 \arrow{r} & \nu_{i'} \arrow{rr} \arrow{ru} & & (j')^* \nu_i \arrow{r} & F \arrow{r} & 0.
\end{tikzcd}
\end{center}
\end{definition}

\begin{remark}
The submanifolds $Y$ and $Z$ intersect transversally exactly when $F = 0$.
\end{remark}

The proof of the following Lemma is fairly easy, but geometric, so we omit it.

\begin{lemma}[{\cite[Proposition 3.3]{Quillen}}]\label{CleanIntersectionFormula}
Suppose that $\nu_{i'}$, $\nu_i$, and $F$ are endowed with complex structures compatible with this exact sequence. For $z \in MU^*(Z)$, \[j^* i_* z = i'_*(e(F) \cdot (j')^* z)\] in $MU^{*+a}(Y, Y \setminus W)$, where $a = \dim \nu_i$. \todo{Mention what $e(F)$ is (the Euler class of $F$, right?)} \qed
\end{lemma}

Now let $G$ be a finite group and let $i\co Z \to X$ be an embedding of $G$--manifolds. Then the $G$--fixed submanifold $X^G$ and $Z$ intersect cleanly in the diagram
\begin{center}
\begin{tikzcd}
Z^G \arrow{r}{r_Z} \arrow{d}{i^G} & Z \arrow{d}{i} \\
X^G \arrow{r}{r_X} & X.
\end{tikzcd}
\end{center}
Since $r_Z^*(\nu_i)$ is a $G$--bundle over a trivial $G$ space, there is a decomposition $r_Z^*(\nu_i) = \nu_{i^G} \oplus \mu_i$, where $\nu_{i^G}$ has no $G$--action and $\mu_i = F$, the excess bundle, carries all of the nontrivial $G$--action.  Applying $EG \times_G (-)$ to the diagram and picking $z \in MU^*(EG \times_G Z)$, \Cref{CleanIntersectionFormula} then gives \[r_X^* i_* z = i^G_*(e(\mu_i) \cdot r_Z^* z) \in MU^*(BG \times X^G, (BG \times X^G) \setminus (BG \times Z^G)).\]  Replacing the embedding condition with orientability, this gives the following:

\begin{lemma}[{\cite[Proposition 3.8]{Quillen}}]\label{ProperCOIntersectionFormula}
Let $f\co Z \to X$ be a proper complex-oriented $G$-map, represented by a factorization \[Z \xrightarrow i E \xrightarrow p X.\]  Let $\mu(E)$ be excess summand of $r_X^* E$ corresponding to the part of $E$ on which $G$ acts nontrivially, where, as before, $r_X$ is the inclusion of the fixpoint submanifold $X^G \subseteq X$.  Then, for $z \in MU^*(EG \times_G Z)$, we have: \[e(\mu(E)) \cdot r_X^* f_* z = f^G_*(e(\mu_i) \cdot r_Z^* z) \in MU^*(BG \times X^G). \qed\]
\end{lemma}

We are now in a position to apply \Cref{ProperCOIntersectionFormula} to our power operation square.

\begin{lemma}[{\cite[Proposition 3.12]{Quillen}}]\label{PowerOpAndEulerClasses}
Suppose $G$ acts transitively on $\{1, \ldots, k\}$ and let $\rho$ denote the induced reduced regular $G$-representation. Suppose that $f: Z \to X$ is a proper complex-oriented map of dimension $2q$ and that $m$ is an integer larger\todo{There's no reason to use $m$, then $r$, then change $r$'s name to $n$ in Lecture 2.6.  Straighten out this terrible naming scheme.} than the dimension of $Z$, so that $m \eps + \nu_f$ is a vector bundle over $Z$, well-defined up to isomorphism\todo{Make it clearer what you mean here. You want the witness to the complex--orientability of $f$ to be homotopically independent of choice.}, where $\eps$ is the trivial complex line bundle. Then \[e(\rho)^m P(f_* 1) = f_* e(\rho \otimes (m \eps + \nu_f)) \in MU^{2m(k-1)-2qk}(BG \times X).\]
\end{lemma}
\begin{proof}
We can take $m$ large enough that the complex-orientation on $f$ can be represented by a factorization \[Z \xrightarrow i m \eps \xrightarrow p X,\] and consider its $k${\th} power \[Z^{\times k} \xrightarrow{i^{\times k}} (m \eps)^{\times k} \xrightarrow{p^{\times k}} X^{\times k}.\]  We calculate the excess bundles to be
\begin{align*}
\mu_{i^{\times k}} & = \rho \otimes \nu_i, &
\mu((m \eps)^{\times k}) & = \rho \otimes m \eps.
\end{align*}
Since $G$ acts transitively, $\Delta: X \to X^{\times k}$ represents the inclusion of the $G$--fixed points.  Packaging all this into \Cref{ProperCOIntersectionFormula} gives \[e(\rho \otimes m\epsilon) \cdot \Delta^* f^{\times k}_{hG}{}_*(1) = f_*(e(\rho \otimes \nu_i) \cdot r_{W \to Z^{\times k}}^*(1)).\]
We then investigate each part separately:\todo{The claim about $\nu_i = m \eps + \nu_f$ is a little mysterious. We had to stare at it too before it became believable.}
\begin{align*}
e(\rho \otimes m\eps) & = e(\rho^{\oplus m}) = e(\rho)^m, &
\Delta^* f^{\times k}_{hG} {}_*(1) & = P(f_* 1), \\
e(\rho \otimes \nu_i) & = e(\rho \otimes (m \eps + \nu_f)), &
r^*_{W \to Z^{\times k}}(1) & = 1
\end{align*}
from which the claim follows.
\end{proof}

The utility of this theorem comes from our ability to compute just a little bit about the Euler classes involved in its statement.
\begin{corollary}[{\cite[Proposition 3.17]{Quillen}}]\label{QuillensKeyRelation}
\todo{Should we use $a_\alpha(v)$ as the notation?} \todo{In previous sections, you've been using $\xi$ to denote arbitrary vector bundles, not $E$.}Specialize to $G = C_k$, and let $\eta$ denote the line bundle on $BG$ owing to the inclusion $C_k \subseteq U(1)$.  Set $e(\eta) = v$ and $e(\rho) = w$.  Then, the Steenrod operation and Landweber operations are related by the formula \[w^{r+q} P x = \sum_{|\alpha| \le r} w^{r - |\alpha|} a(v)^\alpha s_\alpha(x)\] for $x \in MU^{-2q}(X)$ and $r$ is any integer sufficiently large with respect to $\dim X$ and $q$, where $a_j(T)$ are power series with coefficients in the subring $C$ generated by the tautological formal group law on $MU^*(*)$.
\end{corollary}
\begin{proof}
The bundle $\rho$ splits as $\bigoplus_{i=1}^{k-1} \eta^{\otimes i}$.  Then, if $\L$ is any other line bundle with a trivial $G$--action,
\begin{align*}
e(\rho \otimes \L) & = e \left( \bigoplus_{i=1}^{k-1} \eta^i \otimes \L \right) = \prod_{i=1}^{k-1} e(\eta^i \otimes \L) \\
& = \prod_{i=1}^{k-1} F([i]_F(v), e(\L)) = w + \sum_{j=1}^\infty a_j(v) e(\L)^j,
\end{align*}
where \todo{You could justify this part. The point is to look at the product of all the factor summands which don't involve $e(\L)$ at all.}\[w = e(\rho) = (k-1)! v^{k-1} + \sum_{j \ge k} b_j v^j\] for $b_j \in C$.  In general, the splitting principle shows that $e(\rho \otimes E)$ has \[e(\rho \otimes E) = \sum_{|\alpha| \le r} w^{r-|\alpha|} a(v)^\alpha c_\alpha(E).\]  Setting $E = m \eps + \nu_f$, we calculate $r = \dim(m \eps + \nu_f) = m - q$.  Inserting this into \Cref{PowerOpAndEulerClasses} then gives
\begin{align*}
w^m P(f_* 1) & = f_* \left( \sum_{|\alpha| \le r} w^{r - |\alpha|} a(v)^\alpha c_\alpha(m \eps + \nu_f) \right) \\
w^{r+q} P(f_* 1) & = \sum_{|\alpha| \le r} w^{r - |\alpha|} a(v)^\alpha f_* c_\alpha(m \eps + \nu_f) \\
& = \sum_{|\alpha| \le r} w^{r - |\alpha|} a(v)^\alpha s_\alpha(f_* 1). \qedhere
\end{align*}
\todo{Check that this last line is right. Can you pull Gysin maps past Euler classes?  What happened to $m \eps$? --- are you using the definition of Landweber--Novikov operations for $\nu_i$ instead of $\nu_f$?  Why?}
\end{proof}

This formula is quite remarkable --- it says that a certain power operation defined for $MU$ is, in fact, additive and stable (after multiplying by $w$ some)!  This is certainly not the case in general, and I'm not aware of an \textit{a priori} reason to expect this to have happened all along.  Tomorrow, we will use it to power an induction to say something about the coefficient ring $MU_*$.






\section{Quillen's theorem}

With \Cref{QuillensKeyRelation} in hand, we deduce Quillen's major structural theorem about $MU_*$.  We will continue to use the following notations:
\begin{itemize}
\item $C$ is the subring of $MU_*$ generated by the coefficients of the formal group law associated to the identity complex--orientation.
\item $G = C_k$ acts by cyclic permutation on $\{1, \ldots, k\}$.  In particular, the action is transitive.
\item $\rho$ is the associated regular representation, and $w = e(\rho)$ its Euler class.
\item $\eta\co BC_k \to BU(1)$ is the associated line bundle, and $v = e(\eta)$ its Euler class.
\end{itemize}

\begin{theorem}[{\cite[Theorem 5.1]{Quillen}}]
If $X$ has the homotopy type of a finite complex, then
\begin{align*}
MU^*(X) & = C \cdot \sum_{q \ge 0} MU^q(X), \\
\widetilde{MU}^*(X) & = C \cdot \sum_{q > 0} MU^q(X).
\end{align*}
\end{theorem}
\begin{proof}
We can focus on the claim \[\widetilde{MU}^{2*}(X) \stackrel{?}{=} C \cdot \sum_{q > 0} MU^{2q}(X) =: R^{2*},\] since $MU^{2*+1}(*) = 0$ and $\widetilde{MU}^{2*+1}(X)$ can be handled by suspending $X$ once, and then the unreduced case follows directly.  We will show this by working $p$--locally and inducting on the value of ``$*$''.\todo{Remark on the base case: in all the negative dimensions, the claim is trivial.}  Suppose that \[R^{-2j}_{(p)} = \widetilde{MU}^{-2j}(X)_{(p)}\] for $j < q$ and consider $x \in \widetilde{MU}^{-2q}(X)$.  Then, for $n \gg 0$, we have \[w^{n+q} P x = \sum_{|\alpha| \le n} w^{n - |\alpha|} a(v)^\alpha s_\alpha x.\]  Recall that $w$ is a power series in $v$ with coefficients in $C$ and leading term $(p-1)! v^{p-1}$, so that $v^{p-1} = w \cdot \theta(v)$ for some invertible series $\theta$ with coefficients in $C$.  Since $s_\alpha$ lowers degree, we have $s_\alpha x \in R$ by the inductive hypothesis, so we may write \[v^m(w^qPx - x) = \psi_x(v)\] with $\psi_x(T) \in R_{(p)}\llbracket T \rrbracket$.

Suppose $m \ge 1$ is the least integer for which we can write such an equation --- we will show $m = 1$ in a moment.  Applying the inclusion $i\co X \to X \times B\Z/p$ to this equation sets $v = 0$ and yields $\psi_x(0) = 0$, hence $\psi_x(T) = T \phi_x(T)$ and \[v (v^{m-1}(w^qPx - x) - \phi_x(v)) = 0.\]  Since $v$ annihilates this equation, we can use the Gysin sequence associated to the spherical bundle \[S^1 \to S(\eta) \to BC_p\] to produce a class $y \in \widetilde{MU}^{2(m-1)-2q}(X)$ with\todo{It's not clear (from this presentation) why $\<p\>(v)$ is involved in this sequence or where the shift by $-1$ in the dimension went.  I'm a little confused about Quillen's presentation of the total space as ``$S^\infty \times_{C_p} S^1$'', too.} \[v^{m-1}(w^q P x - x) = \phi_x(x) + y \<p\>(v).\]  If $m > 1$, then $y \in R$ for degree reasons and hence the right-hand side gives an equation contradicting our minimality hypothesis.  So, $m = 1$, and the outer factor of $v^{m-1}$ is not present in the last expression.  Restricting along $i$ again, we obtain the equation \[\left. \begin{array}{rr} -x & \text{if $q > 0$} \\ x^p - x & \text{if $q = 0$} \end{array} \right\} = \phi_x(0) + py.\]

In the first case, where $q > 0$, it follows that $MU^{-2q}(X) \subseteq R^{-2q} + pMU^{-2q}(X)$, and since $MU^{-2q}(X)$ has finite order torsion, it follows that $MU^{-2q}(X) = R^{-2q}$.  In the other case, $x$ can be rewritten as a sum of things in $R^{0}$, things in $p MU^{0}(X)$, and things in $(MU^0)^p$.  Since the ideal $\widetilde{MU}^0(X)$ is nilpotent, it follows that $\widetilde{MU}^0(X) = R^0$, and induction proves the theorem.
\end{proof}

\begin{corollary}\label{QuillenSurjective}
The coefficients of the formal group law span $MU_*$. \qed
\end{corollary}

\begin{remark}
This proof actually also goes through for $MO$ as well.  In that case, it's even easier, since the equation $2 = 0$ in $\pi_0 MO$ causes much of the algebra to collapse.  One can try to further perturb this proof in two ways:
\begin{enumerate}
\item One can try to replace the identity complex--orientation $MU \xrightarrow{\id} MU$ with a nontrivial complex--orientation $MU \xrightarrow{\phi} E$ which is suitably compatible with power operations.  It would be nice to understand why this doesn't give more information about $E$ than what's visible in the Hurewicz image of $\phi$.  Or, conversely, it would be nice to understand a proof of Mahowald's theorem that the free $E_2$--algebra with $p = 0$ is $H\F_p$, which this proof portends to give information about.\todo{Straighten this out.}
\item One can also try to replace $MO$ and $MU$ with $MSp$ or $MSO$.  These, too, have presentations in terms of bordism theories and hence similar power operations to the ones we used above.  On the other hand, the Euler classes in $MSp$--theory, while simple, are not so well-behaved, because they are not controlled by a formal group law.  Characteristic classes in $MSO$--theory are not even simple.\todo{This isn't well-stated either.}
\end{enumerate}
\end{remark}

We now have a foothold on $\pi_* MU$, and this alone is enough to move us to study $\moduli{fgl}$, the moduli scheme of formal group laws.  However, while we're here, it's possible for us to prove the rest of Quillen's theorem, if we get just slightly ahead of ourselves and assume one algebraic fact about $\sheaf O_{\moduli{fgl}}$.  The place to start is with the following topological observation about mixing complex--orientations:

\begin{lemma}[{\cite[Lemma 6.3 and Corollary 6.5]{AdamsBlueBook}}]\label{OrientationsOnEAndMU}
Let $\phi\co MU \to E$ be complex--oriented and consider the two orientations
\begin{align*}
\S \sm MU & \xrightarrow{\eta_E \sm 1} E \sm MU, &
MU \sm \S & \xrightarrow{\phi \sm \eta_{MU}} E \sm MU.
\end{align*}
The two induced coordinates $x^E$ and $x^{MU}$ on $\CP^\infty_{E \sm MU}$ are related by the formulas
\begin{align*}
x^{MU} & = \sum_{j=0}^\infty b_j^E (x^E)^{j+1} = g(x^E), \\
g^{-1}(x^{MU} +_{MU} y^{MU}) & = g^{-1}(x^E) +_E g^{-1}(y^E).
\end{align*}
where $E_* MU \cong E_*[b_1, b_2, \ldots]$.
\end{lemma}
\begin{proof}
The second formula is a direct consequence of the first.  The first formula comes from taking the module generators $\beta_{j+1} \in E_{2(j+1)} \CP^\infty = E_{2j} MU(1)$ and pushing them forward to get the algebra generators $b_j \in E_{2j} MU$.  Then, the triangle
\begin{center}
\begin{tikzcd}
{[\CP^\infty, MU]} \arrow{rr} \arrow{rd} & & {[\CP^\infty, E \sm MU]} \arrow{ld}{\cong} \\
& \CatOf{Modules}_{E_*}(E_* \CP^\infty, E_* MU)
\end{tikzcd}
\end{center}
allows us to pair $x^{MU}$ with $(x^E)^{j+1}$ to determine the coefficients of the series.
\end{proof}

\begin{corollary}[{\cite[Corollary 6.6]{AdamsBlueBook}}]
In particular, for the orientation $MU \to H\Z$ we have \[x_1 +_{MU} x_2 = \exp^H(\log^H(x_1) + \log^H(x_2)),\] where $\exp^H(x) = \sum_{j=0}^\infty b_j x^{j+1}$. \qed
\end{corollary}

However, one also notes that $H\Z_* MU = \Z[b_1, b_2, \ldots]$ carries the universal example of a formal group law with a logarithm --- this observation is independent of any knowledge about $MU_*$.  It turns out that this brings us one step away from understanding $MU_*$:

\begin{theorem}[{To be proven as \Cref{LazardsTheorem}}]\label{DummyLazardsThm}
There is a ring $\sheaf O_{\moduli{fgl}}$ carrying the universal formal group law, and it is free: it is a polynomial ring over $\Z$ in countably many generators. \qed
\end{theorem}

\begin{corollary}\label{QuillensTheorem}
The map $\sheaf O_{\moduli{fgl}} \to MU_*$ is an isomorphism.
\end{corollary}
\begin{proof}
We proved in \Cref{QuillenSurjective} that this map is surjective.  We also proved in \Cref{RationalFGLsHaveLogarithms} that every rational formal group law has a logarithm, i.e., the long composite \[\sheaf O_{\moduli{fgl}} \otimes \Q \to MU_* \otimes \Q \xrightarrow{\cong} (H\Z_* MU) \otimes \Q\] is an isomorphism.  Using \Cref{DummyLazardsThm}, it follows that the map is also injective, hence an isomorphism.
\end{proof}

\begin{corollary}
The ring $\pi_*(MU \sm MU)$ carries the universal example of two strictly isomorphic formal group laws.  Additionally, the ring $\pi_0 (MUP \sm MUP)$ carries the universal example of two isomorphic formal group laws.
\end{corollary}
\begin{proof}
Combine \Cref{OrientationsOnEAndMU} and \Cref{QuillensTheorem}.
\end{proof}




\todo{Make a point about the difference between the two ``moduli problems'' here (or in the context lecture, Lecture 3.1): the natural map $\CatOf{RingSpectra}(MU \mmod MU \sm MU, E) \to \moduli{fg}(E_*)$ given by passing to homotopy groups hits \emph{at most one} connected component.  See also the beginning of the next Case Study for a relevant todo.}



\todo[inline]{There's buzz about a ``Frobenius map'' for structured rings going around these days. I guess the point is that an $E_2$--algebra structure is enough to get a multiplicative map $E^0 X \to E^0 X \otimes E^0 BC_p$. This isn't additive, so it can't come from an infinite loop map, but it becomes additive when passing to the Tate construction: $E^X \to (E^X)^{tC_p}$, using the fact that the genuine $C_p$ fixed points of $E^{X^{\times p}}$ is $E^X$, and the square relating genuine, homotopy, and geometric fixed points.  Mike has been claiming that these results of Quillen's can be interpreted in this way, but I'm not sure what the interpretation is.  He says it has something to do with inverting the Euler class and the part of Quillen's argument that involves walking down the multiples of Euler classes on both sides of the equation.}



















% \subsection*{Run off}








% % divisors and line bundles on $1$--dimensional objects?

% poincare duality for manifolds with oriented tangent bundle

% wrong-way maps: $\zeta^* \zeta_* 1$ gives the Euler class of the bundle

% Explicit Thom isomorphism map for universal cohomology: $\xi: X \to BU(n)$ Thomifies to $T(\xi) \to MU(n) \to MU$, representing a class $g \in MU^* T(\xi)$, and this gives a map
% \begin{align*}
% MU^*(X) & \to MU^* T(\xi) & \cong MU^*(E, E_0)\\
% x & \mapsto & g \smile p^*(x),
% \end{align*}
% where $p: E \to X$ is the projection and $E_0$ is the image of the zero section.

% The wrong--way maps come from conjugating by Poincar\'e duality: \[E^* X \cong E_{d_X-*} X \to E_{d_X-*} Y \cong E^*{*-d_X+d_Y} Y.\]  Poincar\'e duality comes from asserting that the stable normal bundle is oriented for the theory, and then Atiyah duality says \[D(X_+) \simeq \Susp^{-n} T(\nu) [\simeq T(\nu - n\eps) \simeq T(-\tau)]. \]

% ------

% Q: Once you know that $MU$ has the universal formal group law on it, does the description of $MU_* MU$ follow immediately from evenness?  Probably?\todo{No, it follows from \emph{freeness}, and this should go into the operations/model section.}

% -----

% Need to talk about Gysin pushforwards in complex bordism and in ordinary cohomology.  Compare these with the theory of Thom isomorphism in general.  They're equivalent, right?  A complex orientation makes proper maps induce shriek maps, and shriek maps can be used to deduce what Chern classes are by push-pull: if $\zeta: X \to E$ is the zero section of a complex bundle $\xi$, then $e(\xi) = i^* i_*(1)$ I think.


