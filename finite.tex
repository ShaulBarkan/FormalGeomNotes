% -*- root: main.tex -*-

\chapter{Finite spectra}\label{ChapterFiniteSpectra}


\todo{Write an introduction for me.}

\todo{still need to talk about closed and open subschemes, their basic properties}

\todo{Do you need to come to grips with ind-coherent sheaves?}




\section{The context of a spectrum}\label{StableContextLecture}
\citeme{Pridham's article \textit{Presenting higher stacks as simplicial schemes} seems like a good reference?  Maybe some Toen things are appropriate?  I don't really know where this simplicial scheme stuff is written down.}

\todo[inline]{I don't think I mention Hopf algebroids during this lecture! This is a miserable oversight that \emph{must} be corrected.  Also, I should mention the cotensor product for Hopf algebroids.}

Today we will make good on our promise, made during our investigation of the unoriented bordism ring, to explain where the Adams spectral sequence comes from.  This story neatly divides into two parts, and the first half is just an investigation of how rich of an algebraic category $\CatOf{C}$ we can find that supports a factorization
\begin{center}
\begin{tikzcd}
\CatOf{Spectra} \arrow{rr}{E_*} \arrow{rd} & & \CatOf{Modules}_{E_*} \\
& \CatOf{C} \arrow{ru}.
\end{tikzcd}
\end{center}
Our answer to this question will come out of considering Grothendieck's framework of descent.  Classically, descent concerns itself with a map $f\co R \to S$ of a rings and an $S$--module $N$, and it asks questions like:
\begin{itemize}
\item When is there an $R$--module $M$ such that $N \cong M \otimes_R S = f^* M$?
\item What extra data can be placed on $N$, called \textit{descent data}, so that the category of descent data for $N$ is equivalent to the category of preimages of $N$ along $f^*$?
\item What conditions can be placed on $f$ so that the category of descent data for any given module is always contractible, called \textit{effectivity}?
\end{itemize}

The essential structure of these answers is easy to guess if we proceed by example, using the few tools available to us.  Suppose that we begin instead with an $R$--module $M$ and we set $N = M \otimes_R S$.  By tensoring up, we have two $R$--algebra maps $S \to S \otimes_R S$, given by including along either factor, and we can further tensor $N$ up to $N \otimes_R S$ or $S \otimes_R N$.  Since $N$ came from the $R$--module $M$, these are canonically isomorphic: \[\phi\co ((f \otimes 1) \circ f)^* M \cong ((1 \otimes f) \circ f)^* M.\]  Repeating this process produces more isomorphisms which compose according to the triangle
\begin{center}
\begin{tikzcd}
N \otimes_R S \otimes_R S \arrow["\phi_{13}", "\simeq"']{rr} \arrow["\phi_{12}", "\simeq"']{rd} & & S \otimes_R S \otimes_R N \\
& S \otimes_R N \otimes_R S \arrow["\phi_{23}"', "\simeq"]{ru},
\end{tikzcd}
\end{center}
where $\phi_{ij}$ denotes applying $\phi$ to the $i${\th} and $j${\th} coordinates.\citeme{Allen said he knew a good reference for this descent picture}

\begin{definition}
Let $f\co R \to S$ be a map of rings as above.  An $S$--module $N$ equipped with an isomorphism $S \otimes_R N \cong N \otimes_R S$ of $S \otimes_R S$--modules which causes the above triangle to commute is called a \textit{descent datum} for the map $f$.
\end{definition}

\begin{remark}\label{CanonicalCoring}
There are other ways to view this data.  For example, later on we will revisit it from the categorical perspective of \textit{comonads}.  However, there is another perspective which we have already encountered earlier on: that of the \textit{canonical coalgebra} or \textit{Amitsur complex}.  Associated to the map $f\co R \to S$, we can form the ring $S \otimes_R S$, which supports a map \[S \otimes_R S \simeq S \otimes_R R \otimes_R S \to S \otimes_R S \otimes_R S \simeq (S \otimes_R S) \otimes_S (S \otimes_R S).\]  One can check that descent data on a $S$--module is the same as the data of a coaction against $S \otimes_R S$.  As a first step, notice the similarity of function signatures: \[N \xrightarrow{\psi} N \otimes_S (S \otimes_R S) \simeq N \otimes_R S.\]
\end{remark}

The following theorem is the usual culmination of an initial investigation into descent:
\begin{theorem}[Grothendieck]
If $f\co R \to S$ is faithfully flat, then there is an equivalence of $R$--modules and $S$--modules equipped with descent data.
\end{theorem}
\begin{proof}[Jumping off point]
The basic observation is that $0 \to R \to S \to S \otimes_R S$ is an exact sequence of $A$--modules.  This makes much of the homological algebra involved work out.
\end{proof}

In our situation, this hypothesis will essentially never be satisfied, so we will pursue a less dramatic statement of the properties of descent.  To see what kind of theorem one might expect, consider the example of $f\co \Z \to \F_p$, which is neither faithful nor flat.  Then, consider the following list of problems (and their partial solutions):
\begin{itemize}
\item The tensor functor $f^*$ cannot distinguish even between the $\Z$--modules $\Z$ and $\Z/p$.  However, if we use $Lf^*$ and resolve $\Z/p$ as $\Z \xrightarrow{p} \Z$, the complexes $Lf^*(\Z)$ and $Lf^*(\Z/p)$ do look distinct.
\item Once we pass to the derived category, then we are no longer in a situation where we can expect the single cocycle condition from the descent data above to suffice.  Instead, we can form a \text{simplicial scheme}, called the \textit{descent object}, by the formula
\[\sheaf D_{\Z \to \F_p} := \left\{
\begin{tikzcd}[ampersand replacement=\&]
\Spec \F_p \arrow{r} \arrow[leftarrow,shift left=\baselineskip]{r} \arrow[leftarrow,shift right=\baselineskip]{r} \&
\begin{array}{c} \Spec \F_p \\ \times_{\Spec \Z} \\ \Spec \F_p \end{array} \arrow[leftarrow, shift left=(2*\baselineskip)]{r} \arrow[shift left=\baselineskip]{r} \arrow[leftarrow]{r} \arrow[shift right=\baselineskip]{r} \arrow[leftarrow, shift right=(2*\baselineskip)]{r} \&
\begin{array}{c} \Spec \F_p \\ \times_{\Spec \Z} \\ \Spec \F_p \\ \times_{\Spec \Z} \\ \Spec \F_p \end{array} \arrow[leftarrow, shift left=(3*\baselineskip)]{r} \arrow[shift left=(2*\baselineskip)]{r} \arrow[leftarrow, shift left=\baselineskip]{r} \arrow{r} \arrow[leftarrow, shift right=\baselineskip]{r} \arrow[shift right=(2*\baselineskip)]{r} \arrow[leftarrow, shift right=(3*\baselineskip)]{r} \&
\cdots
\end{tikzcd}
\right\}.\]
This is meant to look like the \v{C}ech nerve for the ``cover'' $\Spec \F_p \to \Spec \Z$.
\item Accordingly, we need to update our notion of quasicoherent sheaf to live over a simplicial scheme~\cite[Tag 09VK]{stacks-project}.  Given a simplicial scheme $X$, a sheaf $\sheaf F$ on $X$ will be a sequence of sheaves $\sheaf F[n]$ on $X[n]$ as well as, for each map $\phi: [m] \to [n]$ in the simplicial indexing category inducing a map $X(\phi): X[n] \to X[m]$, a choice of map of sheaves \[\sheaf F(\phi)_* \co \sheaf F[m] \to X(\phi)_* \sheaf F[n].\]  Such a sheaf will be called \textit{quasicoherent} when it is levelwise quasicoherent.
\item Finally, we can characterize the structure a quasicoherent sheaf over $\sheaf D_{\Z \to \F_p}$ receives when it is tensored down from $\Z$.  Such a sheaf enjoys that the adjoint map \[\sheaf F(\phi)^* \co X(\phi)^* \sheaf F[m] \to \sheaf F[n]\] is an isomorphism, and in this case we say that $\sheaf F$ is \textit{Cartesian}.
\end{itemize}

\begin{lemma}\citeme{Hovey's \textit{Morita theory for Hopf algebroids and presheaves of groupoids}.}
Without passing to the derived category, there is an equivalence of categories between Cartesian quasicoherent sheaves on the descent object and quasicoherent sheaves equipped with descent data. \qed
\end{lemma}

The real utility of this framework is that it pulls apart the question of descent into two distinct pieces, summarized in the following theorem:
\begin{theorem}\label{AlgebraicCompletionSituation}
Let $i\co A \to X$ be a closed subscheme, and consider the formal completion
\begin{center}
\begin{tikzcd}
A \arrow{rr}{i} \arrow{rd}{j} & & X \\
& X^\wedge_A. \arrow{ru}{k}
\end{tikzcd}
\end{center}
If $X$ is Noetherian, then $k^*$ is flat as a functor of sheaves, $j^*$ is conservative as a functor in the derived category of sheaves, and there is an equivalence of derived categories of sheaves over $X^\wedge_A$ and sheaves over the descent object $\sheaf D_{A \to X}$.\todo{Surely you're supposed to be saying ``bounded'' sometimes when you talk about the derived category.} \qed
\end{theorem}

\begin{remark}\todo{Is this true? Ha, well, I hope so.}
The usual theorem about faithfully flat descent then follows by using the hypotheses on $i$ to deduce that, e.g., if $i^*$ and $j^*$ are both conservative, then so must $k^*$ be.
\end{remark}

We now transfer what we've learned to the situation of homotopical algebra.  Recalling that spectra are equivalent to $\S$--modules, $\S$ the usual sphere spectrum, then any other ring spectrum comes equipped with a unit map $\eta: \S \to E$ and hence push and pull functors
\begin{align*}
\eta_*\co M & \mapsto M, &
\eta^*\co X & \mapsto E \sm X.
\end{align*}
Correspondingly, to any spectrum $X$ we can define the following cosimplicial spectrum:
\begin{definition}
Let $\sheaf D_E(X)$ be the cosimplicial spectrum determined by the formula
\[\mathcal{D}_E(X) := \left\{
\begin{tikzcd}
\begin{array}{c} E \\ \sm \\ X \end{array} \arrow[leftarrow, shift left=\baselineskip]{r}{\mu} \arrow[shift left=(2*\baselineskip)]{r}{\eta_L} \arrow{r}{\eta_R} &
\begin{array}{c} E \\ \sm \\ E \\ \sm \\ X \end{array} \arrow[shift left=(3*\baselineskip)]{r} \arrow[leftarrow, shift left=(2*\baselineskip)]{r} \arrow[shift left=\baselineskip]{r}{\Delta} \arrow[leftarrow]{r} \arrow[shift right=\baselineskip]{r} &
\begin{array}{c} E \\ \sm \\ E \\ \sm \\ E \\ \sm \\ X \end{array} \arrow[shift left=(4*\baselineskip)]{r} \arrow[leftarrow, shift left=(3*\baselineskip)]{r} \arrow[shift left=(2*\baselineskip)]{r} \arrow[leftarrow, shift left=\baselineskip]{r} \arrow{r} \arrow[leftarrow, shift right=\baselineskip]{r} \arrow[shift right=(2*\baselineskip)]{r} &
\cdots
\end{tikzcd}
\right\}.\]
It is called \textit{the descent object for $X$ from $E$ to $\S$}.
\end{definition}
It is not always the case that $X^\wedge_E$ can be lifted from a cosimplicial object in the homotopy category to a sufficiently structured cosimplicial object that we could take its totalization or homotopy colimit.
\begin{lemma}
When $E$ is an $A_\infty$--ring spectrum, the descent object $\sheaf D_E(X)$ can be naturally considered as a cosimplicial object in the $\infty$--category of spectra. \qed \todo{I don't intend to prove this, but maybe we could say some mealy words about why it's true. At worst, we could give reference to the relevant part of Higher Algebra.}
\end{lemma}
\begin{definition}
Let $E$ be an $A_\infty$--ring spectrum.  Then $X^\wedge_E := \Tot \sheaf D_E(X)$ is called the \textit{$E$--nilpotent completion of $X$}.  The spectral sequence resulting from the coskeletal filtration is called the \textit{$E$--Adams spectral sequence} (for $X$).
\end{definition}

In general, it's quite rare that the $E$--nilpotent completion of a spectrum $X$ recovers $X$, but in the nice cases we typically work in, it has been known to happen.  In particular, there is the following theorem:
\begin{lemma}\citeme{Ravenel's \textit{Localizations w/r/t ...} paper}
Let $E$ be a connective $A_\infty$ ring spectrum and let $X$ be any connective spectrum.  Then $X^\wedge_E$ is equivalent to the ``$\pi_0 E$--localization'' of $X$, i.e., for a prime $p$ the spectrum $X^\wedge_E$ is $p$--local if $\pi_0 E$ is $p$--local, it is $p$--complete if $\pi_0 E$ is $p$--torsion, and otherwise it is just $X$. \qed
\end{lemma}
\begin{proof}[Proof sketch]
\todo{You really can just look at the Adams tower...}
\end{proof}

Finally, we can compare the topological situation with the algebraic situation.  To have any hope of applying algebra and algebraic geometry, we must impose some nicety properties.  Here is the first:

\begin{definition}
$E$ satisfies \CH, the \textbf Commutativity \textbf Hypothesis, when $\pi_* E^{\sm j}$ is commutative for all $j \ge 1$.
\end{definition}

\begin{definition}
Suppose that $E$ is a ring spectrum satisfying {\CH}.  We define a simplicial scheme associated to $E$, called its \textit{context}, to be
\begin{align*}
\mathcal{M}_E & := \Spec \pi_* \sheaf D_E(\S) \\
& = \left\{
\begin{tikzcd}[ampersand replacement=\&]
\Spec \pi_* E \arrow{r} \arrow[leftarrow,shift left=\baselineskip]{r} \arrow[leftarrow,shift right=\baselineskip]{r} \&
\Spec \pi_* \left( \begin{array}{c} E \\ \sm \\ E \end{array} \right) \arrow[leftarrow, shift left=(2*\baselineskip)]{r} \arrow[shift left=\baselineskip]{r} \arrow[leftarrow]{r} \arrow[shift right=\baselineskip]{r} \arrow[leftarrow, shift right=(2*\baselineskip)]{r} \&
\Spec \pi_* \left( \begin{array}{c} E \\ \sm \\ E \\ \sm \\ E \end{array} \right) \arrow[leftarrow, shift left=(3*\baselineskip)]{r} \arrow[shift left=(2*\baselineskip)]{r} \arrow[leftarrow, shift left=\baselineskip]{r} \arrow{r} \arrow[leftarrow, shift right=\baselineskip]{r} \arrow[shift right=(2*\baselineskip)]{r} \arrow[leftarrow, shift right=(3*\baselineskip)]{r} \&
\cdots
\end{tikzcd}
\right\}.
\end{align*}
\end{definition}

The context is the wellspring of the algebraic category $\CatOf C$ dreamed of in the introduction to this lecture.

\begin{definition}\label{DefnHomologyFunctorsValuedInSheaves}
For a ring spectrum $E$ satisfying {\CH} and input spectrum $X$, we define the following diagram of abelian groups:
\begin{align*}
\Gamma \context{E}(X) & := \left\{
\begin{tikzcd}[ampersand replacement=\&]
\pi_* \left( \begin{array}{c} E \\ \sm \\ X \end{array} \right) \arrow[leftarrow, shift left=\baselineskip]{r} \arrow[shift left=(2*\baselineskip)]{r} \arrow{r} \&
\pi_* \left( \begin{array}{c} E \\ \sm \\ E \\ \sm \\ X \end{array} \right) \arrow[shift left=(3*\baselineskip)]{r} \arrow[leftarrow, shift left=(2*\baselineskip)]{r} \arrow[shift left=\baselineskip]{r} \arrow[leftarrow]{r} \arrow[shift right=\baselineskip]{r} \&
\pi_* \left( \begin{array}{c} E \\ \sm \\ E \\ \sm \\ E \\ \sm \\ X \end{array} \right) \arrow[shift left=(4*\baselineskip)]{r} \arrow[leftarrow, shift left=(3*\baselineskip)]{r} \arrow[shift left=(2*\baselineskip)]{r} \arrow[leftarrow, shift left=\baselineskip]{r} \arrow{r} \arrow[leftarrow, shift right=\baselineskip]{r} \arrow[shift right=(2*\baselineskip)]{r} \&
\cdots
\end{tikzcd}
\right\},
\end{align*}
The $j$\th object is a module for $\sheaf{O}(\context{E}[j])$, and hence determines a quasicoherent sheaf over the scheme $\context{E}[j]$.  Suitably interpreted, the maps of abelian groups determine maps of pushforwards so that $\context{E}(X)$ is a quasicoherent sheaf over the simplicial scheme $\context{E}$.
\end{definition}

There is also a common hypothesis on $E$ that brings us back into the world of coalgebra, down from simplicial schemes.
\begin{definition}
Take $E_* E$ to be an $E_*$--module using the left-unit map.  We will say that $E$ satisfies \FH, the \textbf Flatness \textbf Hypothesis, when the right-unit map $E_* \to E_* E$ is a flat map of $E_*$--modules.\footnote{The essential point of this is that it causes $E_* E \otimes_{E_*} E_* X$ to become a homology theory and $E_* E \otimes_{E_*} E_* X \to (E \sm E)_* X$ to become an isomorphism on a point.  Alternatively, this can be viewed as a degeneration condition on the K\"unneth spectral sequence for $E_* (E \sm E)$.}
\end{definition}

\begin{remark}
The main utility of this is that it obviates us from working through the homological algebra of sheaves over simplicial schemes.  Instead, since {\FH} causes $\context{E}$ to become $1$--truncated, we can refer to \Cref{CanonicalCoring} and simply refer back to the homological algebra of comodules.  In light of the discussion in \Cref{HopfAlgebrasFromFiniteGroups,HF2HomologyIsValuedInAutGaEquivarModules}, we also see an interpretation of these groupoid--valued simplicial schemes: they are valued in sets equipped with an action by $\Spec E_* E$, which acts also on the base $\Spec E_*$.  To denote this ``homotopical quotient'' or ``action groupoid'', we will write \[\Spec E_* \mmod \Spec E_* E.\]  Such affine groupoid--valued schemes are themselves quite tangible: their rings of functions form \textit{Hopf algebroids}, and Cartesian quasicoherent sheaves on the groupoid scheme correspond to comodules for the Hopf algebroid.\citeme{A lot of this could use citation.  Most of it is in Ravenel's appendix or Hovey's paper.}
\end{remark}

\begin{remark}
This homotopical perspective is quite useful --- for instance, a map of groupoid--schemes which induces on points a natural weak equivalence of groupoids also induces an equivalence of comodule categories.  In fact, the \emph{derived} comodule category depends only upon the stack associated to the groupoid--scheme, which allows still more contexts to be identified.  We won't need this observation in what's to come, though, and it introduces substantial technical distractions.\todo{Is there an instance where stacky equivalence (rather than natural groupoid equivalence) is relevant for algebraic topologists?  Or, really, anyone?}\todo{Could also explain the difference: levelwise sheaves of $0$--types vs sheaves of $\infty$--types.}
\end{remark}

\todo{I think there is a notion of quasicoherent sheaf directly over $\sheaf D_E$ and an interpretation of Cartesian sheaves in that setting.  I think that a different view on {\FH} is that it causes the functor $\pi_*$ to preserve Cartesianness.}

\begin{example}
Most of the homology theories we will discuss have this property.  For an easy example, $H\F_2P$ certainly has this property: there is only one possible algebraic map $\F_2 \to \mathcal A_*$, so {\FH} is necessarily satisfied.  This grants us access to a description of the context for $H\F_2$: \[\context{H\F_2P} = \Spec F_2 \mmod \InternalAut{\G_a}.\]
\end{example}

\begin{example}\label{ContextOfMUPExample}
The context for $MUP$ is considerably more complicated, but Quillen's theorem can be equivalently stated as giving a description of it.  It is isomorphic to the moduli of formal groups: \[\context{MUP} \simeq \moduli{fg} := \moduli{fgl} \mmod \moduli{ps}^{\gpd},\] where $\moduli{ps} = \InternalEnd(\A^1)$ is the moduli of self-maps of the affine line (i.e., of power series) and $\moduli{ps}^{\gpd}$ is the multiplicative subgroup of invertible such maps.
\end{example}

\todo{Say what open, closed, flat maps of simplicial schemes are?}
\todo{Jon thinks that this picture can be instructively recast in terms of the cotangent complex.  I'm not sure how, but it's something to keep in mind for later.}






\section{The structure of $\moduli{fgl}$}\label{StructureOfMfgl}

Our first goal for today is to outline a program for the rest of this Case Study.  Yesterday, we developed a rich target for $E$--homology: sheaves over an algebro-geometric object $\context{E}$.  Furthermore, we have explored in \Cref{ContextOfMUPExample} an identification $\context{MUP} \simeq \moduli{fg}$, where $\moduli{fg}$ is the ``moduli of formal groups''.  Our initial goal for today is to outline a program by which we can leverage this to study $MUP$.  Abstractly, one can hope to study any sheaf, including $\context{MUP}(X)$, by analyzing its stalks.  The main utility of Quillen's theorem is that it gives us access to a concrete model of $\context{MUP}$, so that we can determine where to even look for those stalks.

With this in mind, given a map $f$ in the diagram
\begin{center}
\begin{tikzcd}
\Spec R \arrow{r}{f} & \moduli{fgl} \arrow[-,double]{r} \arrow{d} & \context{MU}[0] \arrow{d} \arrow[-,double]{r} & \Spec MUP_0 \\
& \moduli{fg} \arrow[-,double]{r} \arrow[leftarrow]{lu} & \context{MU},
\end{tikzcd}
\end{center}
life would be easiest if the $R$--module determined by $f^* \context{MUP}(X)$ were itself the value of a homology theory $R_0(X) = MUP_0 X \otimes_{MUP_0} R$.  After all, the pullback of some arbitrary sheaf along some arbitrary map has no special behavior, but homology functors do have familiar special behaviors which we could hope to exploit.  Generally, this is unreasonable to expect: homology theories are functors which convert cofiber sequences of spectra to long exact sequences of groups, but base--change from $\moduli{fg}$ to $\Spec R$ preserves exact sequences exactly when $f$ is \textit{flat}.  In that case, this gives the following theorem:

\begin{theorem}[Landweber]
Given such a diagram where the diagonal arrow is flat, the functor \[R_0(X) := MUP_0(X) \otimes_{MUP_0} R\] is a homology theory. 
\end{theorem}

\noindent In the course of proving this theorem, Landweber devised a method to recognize flat maps.  Recall that a map $f$ is flat exactly when for any closed substack $i\co A \to \moduli{fg}$ with ideal sheaf $\mathcal I$ there is an exact sequence \[0 \to f^* \mathcal I \to f^* \mathcal O_{\moduli{fg}} \to f^* i_* \mathcal O_A \to 0.\]  Landweber classified the closed substacks of $\moduli{fg}$, thereby giving a method to check maps for flatness.

This appears to be a moot point, however, as it is unreasonable to expect this idea to apply to computing stalks: the inclusion of a closed substack (and so, in particular, a closed point $\Gamma$) is flat only in highly degenerate cases.  We saw in \Cref{AlgebraicCompletionSituation} that this can be repaired: the inclusion of the formal completion of a closed substack of a Noetherian\footnote{$\moduli{fg}$ is not Noetherian, but we will find that each closed point except $\G_a$ lives in an open substack that happens to be Noetherian.} stack is flat, and so we naturally become interested in the infinitesimal deformation spaces of the closed points $\Gamma$ on $\moduli{fg}$.  If we can analyze those, then Landweber's theorem will produce homology theories called $E_\Gamma$.  Moreover, if we find that these deformation spaces are \emph{smooth}, it will follow that their deformation rings support regular sequences.  In this excellent case, by taking the regular quotient we will be able to recover a \emph{homology theory} $K_\Gamma$ which plays the role of computing the stalk of $\context{MUP}(X)$ at $\Gamma$.\footnote{Incidentally, this program has no content when applied to $\context{H\F_2}$, as $\Spec \F_2$ is simply too small.}

We have thus assembled a task list:
\begin{itemize}
\item Describe the open and closed substacks of $\moduli{fg}$.
\item Describe the geometric points of $\moduli{fg}$.
\item Analyze their infinitesimal deformation spaces.
\end{itemize}
This will occupy us for the next few lectures.  Today, we will embark on this analysis by studying the scheme $\moduli{fgl}$ which naturally covers the stack $\moduli{fg}$.

\begin{definition}
There is an affine scheme $\moduli{fgl}$ classifying formal group laws.  Begin with the scheme classifying all bivariate power series:
\begin{align*}
\Spec \Z[a_{ij} \mid i, j \ge 0] & \leftrightarrow \left\{ \text{bivariate power series} \right\}, \\
f \in \Spec\Z[a_{ij} \mid i, j \ge 0](R) & \leftrightarrow \sum_{i, j \ge 0} f(a_{ij}) x^i y^j.
\end{align*}
Then, set $\moduli{fgl}$ to be the closed subscheme selected by the formal group law axioms in \Cref{FGLDefinition}.
\end{definition}

This presentation of $\moduli{fgl}$ as a subscheme appears to be extremely complicated in that its ideal is generated by many hard-to-describe elements, but $\moduli{fgl}$ itself is actually not complicated at all.  We will prove the following theorem:
\begin{theorem}[{\cite[Th\'eor\`eme II]{Lazard}}]\label{LazardsTheorem}
There is a noncanonical isomorphism \[\sheaf{O}_{\moduli{fgl}} \cong \Z[t_n \mid 1 \le n < \infty]. \qed\]
\end{theorem}

The most important consequence of this is \emph{smoothness}:
\begin{corollary}\label{MfglIsSmooth}
Given a formal group law $\phi$ over a ring $R$ and a surjective ring map $f\co S \to R$, there exists a formal group law $\widetilde \phi$ over $S$ with \[\phi = f^* \widetilde \phi. \qed\]
\end{corollary}

\begin{remark}
One might hope that the filtration above has an immediate geometric realization.  After all, one can consider the $n${\th} order formal neighborhood $\A^{1, (n)}$ of \Cref{FiniteOrderAffineSpaceDefn}.  The appropriate analogue of \Cref{MapsOfFVarsArePowerSeries} shows that a map \[\A^{1, (n)} \times \A^{1, (n)} \to \A^{1, (n)}\] is represented by a bivariate power series, \emph{modulo the ideal $(x^{n+1}, y^{n+1})$}.  This ideal is distinct from $(x, y)^{n+1}$, and so the source scheme of a formal $n$--bud is not the square of $\A^{1, (n)}$, and a formal $n$--bud does \emph{not} determine a group object on some finite scheme.  This is actually a good thing: there are structure theorems preventing many of these intermediate group structures on finite schemes from existing.\citeme{Akhil is who reminded me of this, back in Berkeley.}
\end{remark}

\todo{There's some hidden text here about $n$--buds, but I don't think we ever care about it.}
% In addition to the Corollary above, we can deduce the following:

% \begin{corollary}[{cf. \Cref{LazardsTheorem}}]\citeme{Lazard??}
% There are elements $t_n \in L$, $0 < n < \infty$, such that
% \begin{align*}
% L_r & \cong \Z[t_1, t_2, \ldots, t_r], &
% L_\infty & \cong \Z[t_1, t_2, \ldots, t_r, \ldots],
% \end{align*}
% and such that $L_r \to L_{r+1}$ is the evident inclusion. \qed
% \end{corollary}

% \begin{corollary}[{\cite[Th\'eor\`eme III]{Lazard}}]
% Given an $r$--bud over $R$, there exists an $(r+1)$--bud extending it. \qed
% \end{corollary}

\begin{proof}[{Proof of \Cref{LazardsTheorem}}]
Let $U = \Z[b_0, b_1, b_2, \ldots] / (b_0 - 1)$ be the universal ring supporting a ``strict'' exponential \[\exp(x) := \sum_{j=0}^\infty b_j x^{j+1}\] with compositional inverse \[\log(x) := \sum_{j=0}^\infty m_j x^{j+1}.\]  They induce a formal group law on $U$ by the formula \[x +_U y = \exp(\log(x) + \log(y)),\] classified by a map $u\co L_\infty \to U$.  Modulo decomposables, this map can be computed as \[u(a_{i(n-i)}) = \binom{n}{i} b_{n-1} \pmod{\text{decomposables}}.\]  Writing $d_n = \gcd\left( \binom{n}{i} \middle| 0 < k < n \right)$, the map $Qu$ on degree $2n$ has image the subgroup generated by $d_{n+1} b_n$.  We write $T_{2n}$ for this subgroup.  Using the splitting of $Qu$ from \Cref{Symmetric2CocycleLemma}.4 below, we use the freeness of $U$ to \emph{choose} an algebra splitting \[U \xrightarrow{v} L_\infty \xrightarrow{u} U.\]  The map $v$ is an isomorphism because $uv$ is injective and because we have checked that $v$ is surjective on indecomposables.\todo{Do the intermediate rings matter here?}
\end{proof}

\begin{definition}
In order to prove the missing \Cref{Symmetric2CocycleLemma}, it will be useful to study the series $+_\phi$ ``up to degree $n$'', i.e., modulo $(x, y)^{n+1}$.  Such a truncated series satisfying the analogues of the formal group law axioms is called a \textit{formal $n$--bud}.  Additionally, a \textit{symmetric $2$--cocycle} is a symmetric polynomial $f(x, y)$ satisfying the equation \[f(x, y) - f(t + x, y) + f(t, x + y) - f(t, x) = 0.\]
\end{definition}

\begin{lemma}[Symmetric $2$--cocycle lemma]\label{Symmetric2CocycleLemma}
The following are equivalent and true:
\begin{enumerate}
\item Symmetric $2$--cocycles that are homogeneous polynomials of degree $n$ are spanned by \[c_n = \frac{1}{d_n} \cdot ((x + y)^n - x^n - y^n).\]
\item For $F$ is an $r$--bud, the set of $(r+1)$--buds extending $F$ form a torsor under addition for $R_{2n-2} \otimes c_r$.
\item Any homomorphism $(QL)_{2n} \to A$ factors through the map $(QL)_{2n} \to T_{2n}$.
\item There is a canonical splitting $T_{2n} \to (QL)_{2n}$.
\end{enumerate}
\end{lemma}
\begin{proof}[Equivalences]\renewcommand{\qedsymbol}{}
Verifying that Claims 1 and 2 are equivalent is a matter of writing out the purported $(r+1)$--buds and taking their difference.  To see that Claim 2 is equivalent to Claim 3, follow the chain \[\CatOf{Groups}((QL)_{2n}, A) \cong \CatOf{Rings}(\Z \oplus (QL)_{2n}, \Z \oplus \Susp^{2n} A) \cong \CatOf{Rings}(L, \Z \oplus \Susp^{2n} A).\]  This shows that such a homomorphism of groups determines an extension of the $n$--bud $\G_a$ to an $(n+1)$--bud, which takes the form of a $2$--cocycle with coefficients in $A$, and hence factors through $T_{2n}$.  Finally, Claim 4 is the universal case of Claim 3.
\end{proof}

\begin{proof}[Proof of Claim 1]\citeme{This follows Chapter 3 of COCTALOS.}
It suffices to show the Lemma over a finitely generated ring.  In fact, the Lemma is true for $A \oplus B$ if and only if it's true for $A$ and for $B$, so the structure theorem for finitely generated abelian groups reduces to the cases of $\Z$ and $\Z/p^r$.  If $A \subseteq B$ and the Lemma is true for $B$, it's true for $A$, so we can further reduce the $\Z$ case to $\Q$.\todo{Rephrase this in terms of localizations.}  We can also reduce from $\Z/p^r$ to $\Z/p$ using an inductive, Bockstein-style argument.  Hence, we can now freely assume that our ground object is a prime field.

For a formal group scheme $\G$, we can form a simplicial scheme $B\G$ in the usual way:
\[B\G := \left\{
\begin{tikzcd}[ampersand replacement=\&]
\begin{array}{c} * \\ \times \\ * \end{array} \arrow{r} \arrow[leftarrow,shift left=\baselineskip]{r} \arrow[leftarrow,shift right=\baselineskip]{r} \&
\begin{array}{c} * \\ \times \\ \G \\ \times \\ * \end{array} \arrow[leftarrow, shift left=(2*\baselineskip)]{r} \arrow[shift left=\baselineskip]{r} \arrow[leftarrow]{r} \arrow[shift right=\baselineskip]{r} \arrow[leftarrow, shift right=(2*\baselineskip)]{r} \&
\begin{array}{c} * \\ \times \\ \G \\ \times \\ \G \\ \times \\ *\end{array} \arrow[leftarrow, shift left=(3*\baselineskip)]{r} \arrow[shift left=(2*\baselineskip)]{r} \arrow[leftarrow, shift left=\baselineskip]{r} \arrow{r} \arrow[leftarrow, shift right=\baselineskip]{r} \arrow[shift right=(2*\baselineskip)]{r} \arrow[leftarrow, shift right=(3*\baselineskip)]{r} \&
\cdots
\end{tikzcd}
\right\}.\]
By applying the functor $\InternalHom{FormalGroups}(-, \G_a)(k)$, we get a cosimplicial abelian group, hence a cochain complex, of which we can take the cohomology.  In the case $\G = \G_a$, the $2$--cocycles in this cochain complex are \emph{precisely} the things we've been calling $2$--cocycles\footnote{They aren't obligated to be symmetric, though.}, so we are interested in computing $H^2$.  The first observation in this direction is that $d^1(x^k) = d_k c_k$.  Secondly, one may check that this complex also computes \[\Cotor_{\sheaf O_{\G}^*}(k, k) \cong \Ext_{\sheaf O_{\G}^*}(k, k),\] which we're now going to compute using a more efficient complex.

\begin{itemize}
\item[$\Q$:] There is a resolution \[0 \to \Q[t] \xrightarrow{\cdot t} \Q[t] \to \Q \to 0,\] from which this follows: \[H^* \InternalHom{FormalGroups}(B\G_a, \G_a)(\Q) = \begin{cases} \Q & \text{when $* = 0$}, \\ \Q & \text{when $* = 1$}, \\ 0 & \text{otherwise}. \end{cases}\] This means that every $2$--cocycle is a coboundary, symmetric or not.
\item[$\F_p$:] Again, we switch to working with $\Ext$ over a free divided power algebra.  Such an algebra splits as a tensor of truncated polynomial algebras, and again computing a minimal free resolution results in the calculation \[H^* \InternalHom{FormalGroups}(B\G_a, \G_a)(\F_p) = \Lambda[\alpha_k \mid k \ge 0] \otimes \F_p[\beta_k \mid k \ge 0],\] with $\alpha_k \in \Ext^1$ and $\beta_k \in \Ext^2$.  In fact, $\alpha_k$ is represented by $x^{p^k}$ and $\beta_k$ is represented by $c_{p^k}(x, y)$, and in the case $p = 2$ the exceptional class $\alpha_{k-1}^2$ is represented by $C_{2^k}(x, y)$.\todo{I'm not sure how to do any of these calculations! Ha.}  Since we have representatives for the surviving homology classes and we know where the bounding class lives, it follows that $c_n(x, y)$ and $x^{p^a} y^{p^b}$ give a basis for \emph{all} of the $2$--cocycles.  It's easy to select the symmetric ones, and it agrees with the prediction of the statement of the Lemma.
\end{itemize}
This finally concludes the proof of \Cref{LazardsTheorem}.
\end{proof}

\todo[inline]{Also, people seem to say things about the Mischenko logarithm rather than the invariant differential, but I think we should phrase things in those terms.}
\todo[inline]{Section 12 of Neil's FG notes talk about the infinite height subscheme of $\moduli{fgl}$. He compares it to $H_* MO$ and to the Hurewicz image $\pi_* MU \to H_*(MU; \F_p)$.}







\section{The structure of $\moduli{fg}$ I: Large scales}

Having described the structure of $\moduli{fgl}$, we turn to understanding the quotient stack $\moduli{fg}$.  Earlier, we proved the following theorem:

\begin{theorem}\todo{Be careful about $\star$-isomorphisms versus isomorphisms.}
Let $k$ be any field of characteristic $0$.  Then there is a unique map \[\Spec k \to \moduli{fg}. \qed\]
\end{theorem}
\begin{proof}
This is a rephrasing of \Cref{RationalFGLsHaveLogarithms} in the language of stacks.
\end{proof}

We would like to have a similar classification of the closed points in positive characteristic.  We proved the theorem above by solving a certain differential equation, which necessitated integrating a power series.  Integration is what we expect to fail in positive characteristic.  The following definition tracks \emph{where} it fails:
\begin{definition}
Let $+_\phi$ be a formal group law.  Let $n$ be the largest degree such that there exists a formal power series $\ell$ with \[\ell(x +_\phi y) = \ell(x) + \ell(y) \pmod{(x, y)^{n}},\] i.e., $\ell$ is a logarithm for the $n$--bud determined by $+_\phi$.  The \textit{$p$--height of $+_\phi$} is defined to be $\log_p(n)$.
\end{definition}

In fact, we will show that this definition is well-behaved, in the following sense:
\begin{lemma}\label{FGLHeightIsAnInteger}
Over a field of positive characteristic, the $p$--height of a formal group law is always an integer.  (That is, $n = p^d$ for some natural number $d$.)
\end{lemma}
\noindent We will have to develop some machinery to get there.  First, notice that this definition of height really depends on the formal group rather than the formal group law.

\begin{lemma}
The height of a formal group law is an isomorphism invariant, i.e., it descends to a function on $\moduli{fg}$.
\end{lemma}
\begin{proof}
The series $\ell$ is a partial logarithm for the formal group law $\phi$, i.e., an isomorphism between the formal group defined by $\phi$ and the additive group.  Since isomorphisms compose, this statement follows.
\end{proof}

With this in mind, we look for a more standard form for formal group laws.  In light of our goal Lemma, the most obvious standard form is as follows:
\begin{definition}
Suppose that a formal group law $+_\phi$ does have a logarithm.  We say that $+_\phi$ has a \textit{$p$--typical logarithm} in the case that its logarithm has the form \[\log_\phi(x) = \sum_{j=0}^\infty \ell_j x^{p^j}.\]
\end{definition}

\begin{lemma}\label{EveryLogHaspTypification}\citeme{This is due to Hazewinkel}
Every formal group law $+_\phi$ with a logarithm $\log_\phi$ is naturally isomorphic to one whose logarithm is $p$--typical, called the $p$--typification of $+_\phi$.
\end{lemma}
\begin{proof}
\todo[inline]{The slickest proof is to extend the base ring by a bunch of unramified roots of unity, use those to write down an isomorphism that obviously gets rid of the non-power-of-$p$ terms in the logarithm, then note that the formula for the isomorphism is Galois invariant and so descends to the original ring.}
\todo{Build Quillen's idempotent. Maybe lift it to $L_p MU$?}
\end{proof}

Of course, not every formal group law supports a logarithm --- after all, this is the point of ``height''.  There are two ways around this: one is to pick a surjection $S \to R$ from a torsion-free ring $S$, \emph{choose} a lift of the formal group law to $S$, then pass to $S \otimes \Q$ and study how much of the resulting logarithm descends to $R$.  However, it is not clear that this procedure is independent of choice.  We therefore pursue an alternative approach: an intermediate definition that applies to all formal group laws and which specializes to the one above in the presence of a logarithm.  To do this, we consider what computations are made easier with this sort of formula for a logarithm, and we arrive at the following:

\begin{definition}
The \textit{$p$--series} of a formal group law $+_\phi$ is given by the formula \[[p]_\phi(x) := \overset{\text{$p$ times}}{\overbrace{x +_\phi \cdots +_\phi x}}.\]
\end{definition}

\begin{lemma}
If $+_\phi$ is a formal group law with $p$--typical logarithm, then there are elements $v_n$ with \[[p]_\phi(x) = px +_\phi v_1 x^p +_\phi v_2 x^{p^2} +_\phi \cdots +_\phi v_n x^{p^n} +_\phi \cdots.\]
\end{lemma}
\begin{proof}[Proof sketch]
This comes from comparing the two series
\begin{align*}
\log_\phi(px) & = px + \cdots, \\
\log_\phi([p]_\phi(x)) & = p \log_\phi(x) = px + \cdots.
\end{align*}
The difference is concentrated degrees of the form $p^d$, beginning in degree $p$, so one can find an element $v_1$ so that \[p \log_\phi(x) - (\log_\phi(px) + \log_\phi(v_1 x^p))\] starts in degree $p^2$, and so on.  In all, this gives the equation
\begin{align*}
p \log_\phi(x) & = \log_\phi (px) + \log_\phi(v_1 x^p) + \log_\phi(v_2 x^{p^2}) + \cdots \\
\intertext{at which point we can use formal properties of the logarithm to deduce}
\log_\phi [p]_\phi(x) & = \log_\phi \left(px +_\phi v_1 x^p +_\phi v_2 x^{p^2} +_\phi \cdots +_\phi v_n x^{p^n} +_\phi \cdots\right) \\
[p]_\phi(x) & = px +_\phi v_1 x^p +_\phi v_2 x^{p^2} +_\phi \cdots +_\phi v_n x^{p^n} +_\phi \cdots \qedhere
\end{align*}
\end{proof}

\begin{remark}
In fact, the rational logarithm coefficients can be recursively recovered from the coefficients $v_d$, using a similar manipulation:
\begin{align*}
p \log_\phi(x) & = \log_\phi\left([p]_\phi(x)\right) \\
p \sum_{n=0}^\infty m_n x^{p^n} & = \log_\phi \left(\sum_{d=0}^\infty{}_\phi v_d x^{p^d} \right) = \sum_{d=0}^\infty \log_\phi\left(v_d x^{p^d}\right) \\
\sum_{n=0}^\infty p m_n x^{p^n} & = \sum_{d=0}^\infty \sum_{j=0}^\infty m_j v_d^{p^j} x^{p^{d+j}} = \sum_{n=0}^\infty \left( \sum_{k=0}^n m_k v_{n-k}^{p^k} \right) x^{p^n},
\end{align*}
implicitly taking $m_0 = 1$ and $v_0 = p$.
\end{remark}

\todo{Did you make it clear what the $v_n$ have to do with height?}

\begin{definition}
A formal group law is itself said to be \textit{$p$--typical} when its $p$--series has the above form.  (In particular, the logarithm of a $p$--typical formal group law is a $p$--typical logarithm.)
\end{definition}

\begin{corollary}[{\Cref{EveryLogHaspTypification}}]\label{EveryFGLIsPTypical}
Every formal group law is naturally isomorphic to a $p$--typical one.
\end{corollary}
\begin{proof}
The procedure applied to the formal group law $+_\phi$ in the proof of \Cref{EveryLogHaspTypification} applies equally well to an arbitrary formal group law, even without a logarithm --- it just wasn't clear what was being gained.  Now, it is clear: we are gaining the conclusion of this Corollary.
\end{proof}

\begin{remark}
There is an inclusion of groupoid--valued sheaves from $p$--typical formal group laws with isomorphisms to all formal group laws with isomorphisms.  \Cref{EveryFGLIsPTypical} can be viewed as presenting this inclusion as a deformation retraction, and in particular the inclusion is a natural \emph{equivalence} of groupoids.  It follows\needproof{Equivalence of groupoids is a reasonable notion in stacks} that these both present the same stack: $\moduli{fg}$.\todo{This isn't stated very well.}
\end{remark}

\begin{proof}[{Proof of \Cref{FGLHeightIsAnInteger}}]
Replace the formal group law by its $p$--typification.  \todo{Finish this.}
\end{proof}

We're now in a position to 
\begin{theorem}\citeme{Lazard Theorem IV on p.\ 270}
Let $k$ be an algebraically closed field of positive characteristic.  There is a bijection between maps \[\Gamma: \Spec k \to \moduli{fg}\] and numbers $1 \le d \le \infty$ given by $\Gamma \mapsto \height(\Gamma)$.
\end{theorem}
\begin{proof}
The easy part of the proof is surjectivity: take the $p$--typical formal group law over $\F_p$ determined by the $p$--series $[p]_{\phi_d}(x) = x^{p^d}$, sometimes called the \textit{Honda formal group law}.

To show injectivity, we must show that every formal group law over $\bar k$ is isomorphic to the appropriate Honda group law. \todo{Produce such an isomorphism.}
\end{proof}

\todo[inline]{Here's a remark, 11.2 in Neil's FG class notes: the functor to iso classes of formal groups (i.e., the course moduli) is definitely not a scheme, since it's infinite on $\F_p$, so it's infinite on $\Z$, yet it's a singleton on $\Q$. If it were representable by a scheme, the value on $\Z$ would inject to the value on $\Q$.}


\textbf{Now analyze the open and closed substacks.}

\begin{lemma}\citeme{Wilson's primer, Theorem 4.6}
\[\sum_{\substack{i \ge 0 \\ j > 0}}{}^F t_i \eta_R(v_j)^{p^i} \equiv \sum_{\substack{i > 0 \\ j \ge 0}}{}^F v_i t_j^{p^i} \pmod p.\]
\end{lemma}

\begin{corollary}\citeme{Lemmas 4.7-8 of Wilson's primer}
The ideal $I_n = (p, v_1, \ldots, v_{n-1}) \subseteq BP_*$ is invariant.  Also, $\eta_R(v_n) \equiv v_n \pmod{I_n}$.
\end{corollary}

\begin{theorem}\citeme{Theorem 4.9 of Wilson's primer}
If $I$ is an invariant prime ideal, then $I = I_n$ for some $n$.
\end{theorem}
\begin{proof}[Proof sketch]
Inductively assume that $I_n \subseteq I$.  If this is not an equality, we want to show that $I_{n+1} \subseteq I$ is forced.  Take $y \in I \setminus I_n$; if we could show \[\eta_R(y) = a v_n^j t^K + \cdots,\] we could proceed by primality to show that $v_n \in I$ and hence $I_{n+1} \subseteq I$.  This is possible, but requires serious bookkeeping.
\end{proof}

\textbf{Here's an omnibus theorem:}

\begin{theorem}[Landweber, part 2: classification of closed substacks]
Let $BP_*$ be the ring classifying formal group laws with $p$--typical logarithms.  It has the form $BP_* \cong \Z_{(p)}[v_1, v_2, \ldots, v_d, \ldots]$, where $v_d \equiv p \ell_d \pmod{\text{decomposables}}$.  The unique closed substack of $\moduli{fg} \times \Spec \Z_{(p)}$ of codimension $d$ is selected by $BP_* / (p, v_1, \ldots, v_{d-1})$, and its complementary open substack of dimension $d$ is selected by either of $v_d^{-1} BP_*$ or $v_d^{-1} \Z_{(p)}[v_1, \ldots, v_d]$. \qed
\end{theorem}

\begin{remark}
It's worth pointing out how strange this is. In Euclidean geometry, open subspaces are always top-dimensional, and closed subspaces can drop dimension.  Here, proper open substacks of every dimension appear, and every nonempty closed substack is $\infty$--dimensional (albeit of positive codimension).
\end{remark}








\section{The structure of $\moduli{fg}$ II: Small scales}


\textbf{Now analyze the deformation spaces}

We now turn to the deformation theory of formal groups, which is about the appearance of formal groups in families.

\begin{definition}
Given a formal group $\Gamma$ classified by a map $\Spec k \to \moduli{fg}$, then a \textit{deformation of $\Gamma$ to a scheme $X$} is a factorization \[\Spec k \to X \to \moduli{fg}.\]  If $X$ is a nilpotent thickening of $\Spec k$ (or an ind-system of such), then the deformation is said to be \textit{infinitesimal}.
\end{definition}

The study of all possible infinitesimal deformations of a particular map $\Spec k \to \moduli{fg}$ has a geometric interpretation, embodied by the following Lemma:

\begin{lemma}\todo{Actually, maybe this came up at the beginning of the previous day?}
Let $\Spec k \to Y$ be any map, and let $\Spec k \to X \to Y$ be a factorization through a nilpotent thickening $X$ of $\Spec k$.  Then there is a natural further factorization \[\Spec k \to X \dashrightarrow Y^\wedge_X \to Y. \qed\]
\end{lemma}

\noindent The spirit of the Lemma, then, is that the study of infinitesimal deformations of $\Gamma: \Spec k \to \moduli{fg}$ is equivalent to the study of $(\moduli{fg})^\wedge_\Gamma$ itself.  So, this fits into our program of analyzing the structure of $\moduli{fg}$.

\begin{example}
Consider the case of an infinitesimal parameter space $X = \A^1$.  A map $\A^1 \to \moduli{fg}$ then is presented by a map $\A^1 \to \moduli{fgl}$, which corresponds to a ``family'' of formal group laws $+_{\phi_h}$ of the form \[x +_{\phi_h} y = (x +_\phi y) + h(x +_{\phi(1)} y) + h^2(x +_{\phi(2)} y) + \cdots\] for some series $+_{\phi(n)}$.  In particular, $+_{\phi(0)}$ is a formal group law over $k$.
\end{example}

The analysis of $(\moduli{fg})^\wedge_\Gamma$ is due to Lubin and Tate, but we follow a more structured approach written down by Lazarev.\citeme{Cite both of these}
\begin{definition}\todo{Phrase this geometrically?  Can you?}
Let $+_\phi$ be a formal group law. The deformation complex $\widehat C^*(\phi)$ is defined by \[R \to R\ps{x_1} \to R\ps{x_1, x_2} \to R\ps{x_1, x_2, x_3} \to \cdots\] with differential
\begin{align*}
(df)(x_1, \ldots, x_{n+1}) & = \phi_1\left(\sum_{i=1}^n {}_\phi x_i, x_{n+1} \right) \cdot f(x_1, \ldots, x_n) \\
& \quad + \sum (-1)^i f(x_1, \ldots, x_i +_\phi x_{i+1}, \ldots, x_{n+1}) \\
& \quad (-1)^{n+1} \left( \phi_2\left(x_1, \sum_{i=2}^{n+1} {}_\phi x_i \right) \cdot f(x_2, \ldots, x_{n+1}) \right),
\end{align*}
where we have written
\begin{align*}
\phi_1(x, y) & = \frac{\partial x +_\phi y}{\partial x}, &
\phi_2(x, y) & = \frac{\partial x +_\phi y}{\partial y}.
\end{align*}
\end{definition}

This complex tracks the data of infinitesimal deformations.  For instance, consider a deformed automorphism $f$ of $+_\phi$, expressed as \[f(x) = f_0(x) + h f_1(x) + h^2 f_2(x) + \cdots,\] and satisfying \[f(x +_\phi y) = f(x) +_\phi f(y).\]  Applying $\left.\frac{\partial}{\partial h}\right|_{h=0}$ to this equality gives \[f_1(x +_\phi y) = \phi_1(x, y)f_1(x) + \phi_2(x, y)f_1(y)\] and thus $f_1$ is a $1$--cocycle in the deformation complex.  A similar sequence of observations culminates in the following theorem:
\begin{theorem}\citeme{Definitely blame this on Lazarev}
Let $+_\phi$ be a formal group law over a ring $R$ and let $S \to R$ be a square--zero extension with kernel $M$.
\begin{enumerate}
\item Automorphisms of $+_\phi$ over $S$ give elements in $\widehat Z^1(\phi)$.
\item Two automorphisms are equivalent if...?
\item Extensions of $+_\phi$ to $S$ give elements in $\widehat Z^2(\phi)$.
\item Two such extensions are isomorphic over $S$ if their cocycles are bounded by an element in $\widehat B^2(\phi)$. \qed
\end{enumerate}
\todo{Summarize all the facts about the deformation complex here.}
\end{theorem}

So, this complex contains all the information we're interested in.  Miraculously, we actually already studied the main input to computing this complex yesterday:

\begin{lemma}
This is quasi-isomorphic to the usual bar complex.
\end{lemma}
\begin{proof}\todo{Actually write this.}
The usual bar complex receives a map defined by \[f \mapsto \check{f}(x_1, \ldots, x_n) = F_1\left( 0, \sum_{i=1}^n {}F x_i \right)^{-1} f(x_1, \ldots, x_n),\] and this is the quasi-isomorphism.

You'll probably want to know the formulas
\begin{align*}
F_1(x, y) & = F_1(0, x)^{-1} F_1(0, F(x, y)), \\
F_2(x, y) & = F_2(y, 0)^{-1} F_2(F(x, y), 0) = F_1(0, y)^{-1} F_1(0, F(x, y)).
\end{align*}
\end{proof}

\begin{lemma}
Let $\G$ be a formal group of finite height $\height(\G) = d$ over a field $k$.  Then, $H^2(\G; M \otimes \G_a)$ is a free $k$--vector space of dimension $(d - 1)$.
\end{lemma}
\begin{proof}
Filter $\G$ by degree and consider the resulting spectral sequence.  Its input is $H^*(\G_a; M \otimes \G_a)$, which we computed yesterday.  To compute the differentials, one computes \[(x +_\phi y)^{p^r} - (x^{p^r} + y^{p^r}) = (\text{unit}) \cdot c_{p^{d + r}}(x, y) + \cdots,\] where we used $c_{p^d}^{p^r} = c_{p^{r+d}}$.  So, we see that there are at least $d - 1$ things at the bottom of the spectral sequence which are not coboundaries, and we need to check that they are indeed cocycles.\todo{This follows from Corollary 7.4.4 in the crystals notes.  Finish this.}
\end{proof}

\textbf{Now build the universal object and check its universal property.  The construction is done in 7.4.3 and the universal property is done in 7.5 of the crystals notes.}



Show that the action of the stabilizer group lifts to an action on Lubin--Tate space

Lubin--Tate deformation theory



\citeme{Neil's FG notes in the first half of section 18 talk about additive extensions and their relation to infinitesimal deformations.  In the second half, he (more or less) talks about the de Rham crystal and shows that $\Ext_{\mathrm{rigid}}(G, \G_a) \cong \operatorname{Prim}(H^1_{dR}(G/X))$ in 18.37.}






\todo[inline]{I still have some confusion about the formal similarity between deforming formal group laws over square-zero extensions of the base and deforming formal $n$--buds over the finite order nilpotent neighborhoods of a point.  This would be a good place to sort that out.}

----

The nice picture I drew for Catherine

-----

We have now arrived at the conclusion of our program from \Cref{StructureOfMfgl} for manufacturing interesting homology theories from Quillen's theorem: we have an ample supply of open substacks of $\moduli{fg}$, and we have analyzed its closed points as well as their deformation neighborhoods.  We make the following definitions:
\begin{itemize}
\item Recall that the moduli of $p$--typical group laws is affine, presented by the scheme $\Spec BP_*$, $BP_* := \Z_{(p)}[v_1, v_2, \ldots, v_d, \ldots]$.  Since the inclusion of $p$--typical group laws into all group laws induces an equivalence of stacks, it follows that this formula determines a homology theory, called \textit{Brown--Peterson homology}: \[BP_*(X) := MU_*(X) \otimes)_{MU_*} BP_*.\]
\item A chart for the open substack $\moduli{fg}^{\le d}$ in terms of $\moduli{fg}{}_{,(p)} \cong \Spec \Z_{(p)}[v_1, v_2, \ldots, v_d, \ldots]$ is $\Spec \Z_{(p)}[v_1, v_2, \ldots, v_d^\pm]$.  It follows that there is a homology theory $E(d)_*(X)$, called \textit{the $d${\th} Johnson--Wilson homology}, defined by \[E(d)_*(X) := BP_*(X) \otimes_{BP_*} E(d)_*.\]
\item Similarly, for a formal group $\Gamma$ of height $d < \infty$, there is a chart $\Spf \Z_p\ps{u_1, \ldots, u_{d-1}}$ for its deformation neighborhood.  Correspondingly, there is a homology theory $E_\Gamma{}_*$, called \textit{the Morava $E$--theory for $\Gamma$}, determined by \textbf{Actually, this one is complicated: you want to tensor down to every formal neighborhood, but there may be infinite divisibility data to keep track of and shuffle around.  I should think this through more carefully.}
\item Finally, since $(p, u_1, \ldots, u_{d-1})$ forms a regular sequence on $E_\Gamma{}_*$, we can form the regular quotient $K_\Gamma$ in the homotopy category.  This determines a spectrum, and hence determines a homology theory called \textit{the Morava $K$--theory for $\Gamma$}.
\end{itemize}

Landweber's flatness criterion

A $BP_*$--module $M$ gives a flat sheaf on $\moduli{fg}$ exactly when $(p, v_1, v_2, \ldots, v_{d-1}, \ldots)$ is a regular sequence $M$ too.  (In particular, $BP_*$ is itself such a module, and so gives rise to a homology theory $BP$ with $\context{BP} \simeq \moduli{fg} \times \Spec \Z_{(p)}$.)


Point out that it's most important to understand $K(d)$ for each $d$, not $K_\Gamma$ for every $\Gamma$.

\begin{remark}
Morava $K$--theory at the even prime is not commutative.

\todo[inline]{cf.\ Section 5.2 of KLW.  Or, at $p = 2$ there is a derivation $Q: K \to \Susp K$ with $ab - ba = u Q(a) Q(b)$, so $K^0 X$ is commutative whenever $K^1 X = 0$ (cf.\ Theorem 2.13 of Strickland's \textit{Products on $MU$--modules}).}
\end{remark}







\section{Nilpotence and periodicity}

\todo{I think it makes sense to break the following lecture into two days.}

Having constructed these ``stalk'' homology theories, I want to show that you can actually perform stalkwise analyses of the sheaves coming from bordism theory.  Our example case is a famous theorem: the solution of Ravenel's nilpotence conjectures by Devinatz, Hopkins, and Smith.  Their theorem concerns spectra which ``detect nilpotence'' in the following sense:

\begin{definition}
A ring spectrum $E$ \textit{detects nilpotence} if, for any ring spectrum $R$, the kernel of the Hurewicz homomorphism $E_*: \pi_* R \to E_* R$ consists of nilpotent elements.
\end{definition}

First, a word about why one would care about such a condition.  The following theorem is classical:
\begin{theorem}[Nishida]
Every homotopy class $\alpha \in \pi_{\ge 1} \S$ is nilpotent. \qed
\end{theorem}

\noindent However, people studying $K$--theory in the '$70$s discovered the following phenomenon:

\begin{theorem}[Adams]
Let $M_{2n}(p)$ denote the mod--$p$ Moore spectrum with bottom cell in degree $2n$.  Then there is an index $n$ and a map $v: M_{2n}(p) \to M_0(p)$ such that $KU_* v$ acts by multiplication by the $n$\textsuperscript{th}\, power of the Bott class.  The minimal such $n$ is given by the formula \[n = \begin{cases} p-1 & \text{when $p \ge 3$}, \\ 4 & \text{when $p = 2$}. \qed \end{cases}\]
\end{theorem}

\noindent In particular, this means that $v$ cannot be nilpotent, since a null-homotopic map induces the zero map in any homology theory.  Just as we took the non-nilpotent endomorphism $p$ in $\pi_0 \End \S$ and coned it off, we can take the endomorphism $v$ in $\pi_{2p-2} \End M_0(p)$ and cone it off to form a new spectrum called $V(1)$.

\begin{remark}
The spectrum $V(1)$ is actually defined to be a finite spectrum with $BP_* V(1) \cong BP_* / (p, v_1)$. At $p = 2$ this spectrum doesn't exist and this is a misnomer.  More generally, at odd primes $p$ Nave shows that $V((p+1)/2)$ doesn't exist.\citeme{Cite Lee Nave}
\end{remark}

One can ask, then, whether the pattern continues: does $V(1)$ have a non-nilpotent self-map, and can we cone it off to form a new such spectrum with a new such map?  Can we then do that again, indefinitely?  In order to study this question, we are motivated to find spectra $E$ as above, since an $E$ that detects nilpotence cannot send such a nontrivial self-map to zero.  In fact, we found one such $E$ already:

\begin{theorem}[Devinatz--Hopkins--Smith]\label{DevinatzHopkinsSmith}\citeme{Devinatz, Hopkins, Smith}
Complex cobordism $MU$ detects nilpotence. \qed
\end{theorem}

They also show that the $MU$ is the universal object which detects nilpotence, in the sense that any other ring spectrum can have this property checked stalkwise on $\context{MU}$:

\begin{corollary}[Hopkins--Smith]\citeme{Hopkins--Smith}
A ring spectrum $E$ detects nilpotence if and only if $K(d)_* E \ne 0$ for all $0 \le d \le \infty$ and for all primes $p$.
\end{corollary}
\begin{proof}
If $K(d)_* E = 0$ for some $d$, then the non-nilpotent map $\S \to K(d)$ lies in the kernel of the Hurewicz homomorphism for $E$, so $E$ fails to detect nilpotence.

Hence, for any $d$ we must have $K(d)_* E \ne 0$.  Because $K(d)_*$ is a field, it follows by picking a basis of $K(d)_* E$ that $K(d) \sm E$ is a nonempty wedge of suspensions of $K(d)$.  So, for $\alpha \in \pi_* R$, if $E_* \alpha = 0$ then $(K(d) \sm E)_* \alpha = 0$ and hence $K(d)_* \alpha = 0$.  So, we need to show that if $K(d)_* \alpha = 0$ for all $n$ and all $p$ then $\alpha$ is nilpotent.  Taking \Cref{DevinatzHopkinsSmith} as given, it would suffice to show merely that $MU_* \alpha$ is nilpotent.  This is equivalent to showing that the ring spectrum $MU \sm R[\alpha^{-1}]$ is contractible or that the unit map is null: \[\S \to MU \sm R[\alpha^{-1}].\]

Pick a prime $p$ and recall the regular sequence of Landweber's theorem.  We define a spectrum $P(d+1)$ to be the regular quotient of $BP$ by $(p, v_1, \ldots, v_d)$.  A nontrivial result of Johnson and Wilson shows that if $MU_* X = 0$ for any $X$, then for any $d$ we have $K([0, d])_* X = 0$ and $P(d+1)_* X = 0$.\footnote{It is immediate that $MU_* X = 0$ forces $P(d+1)_* X = 0$ and $v_{d'}^{-1} P(d')_*(X) = 0$ for all $d' < d$.  What's nontrivial is showing that $v_{d'}^{-1} P(d')_*(X) = 0$ if and only if $K(d')_*(X) = 0$.}  Taking $X = R[\alpha^{-1}]$, have assumed all of these are zero except for $P(d+1)$.  But $\colim_d P(d+1) \simeq H\F_p \simeq K(\infty)$, and $\S \to K(\infty) \sm R[\alpha^{-1}]$ is assumed to be null as well.  By compactness of $\S$, that null-homotopy factors through some finite stage $P(d+1) \sm R[\alpha]$ with $d \gg 0$.
\end{proof}

As another example of the primacy of these methods, we can show the following interesting result.  Say that $R$ is a field spectrum when every $R$--module (in the homotopy category) splits as a wedge of suspensions of $R$.  It is easy to check (as mentioned in the proof above) that $K(d)$ is an example of such a spectrum.

\begin{corollary}
Every field spectrum $R$ splits as a wedge of Morava's $K(d)$ theories.
\end{corollary}
\begin{proof}
Set $E = \bigvee_{\text{primes $p$}} \bigvee_{d \in [0, \infty]} K(d)$, so that $E$ detects nilpotence.  The class $1$ in the field spectrum $R$ is non-nilpotent, so it survives when paired with some $K$--theory $K(d)$, and hence $R \sm K(d)$ is not contractible.  Because both $R$ and $K(d)$ are field spectra, the smash product of the two simultaneously decomposes into a wedge of $K(d)$s and a wedge of $R$s.  So, $R$ is a retract of a wedge of $K(d)$s, and picking a basis for its image on homotopy shows that it is a sub-wedge of $K(d)$s.
\end{proof}

\begin{remark}
This is interesting in its own right, because field spectra are exactly those spectra which have K\"unneth isomorphisms.  So, even if you weren't neck-deep in algebraic geometry, you might still have struck across these homology theories just if you like to compute things, since K\"unneth formulas make things computable.
\end{remark}

We're now well-situated to address Ravenel's question about finite spectra and periodic self-maps.  The solution to this problem passes through some now-standard machinery for triangulated $\otimes$--categories.

\begin{definition}
A subcategory of the category of a triangulated category (e.g., $p$--local finite spectra) is \textit{thick} if it is closed under weak equivalences, it is closed under retracts, and it has a $2$-out-of-$3$ property for cofiber sequences.
\end{definition}

\noindent Examples of thick subcategories include:
\begin{itemize}
\item The category $\CatOf{C}_d$ of $p$--local finite spectra which are $K(d-1)$--acyclic.  (For instance, if $d = 1$, the condition of $K(0)$--acyclicity is that the spectrum have purely torsion homotopy groups.)  These are called ``finite spectra of type at least $d$''.
\item The category $\CatOf{D}_d$ of $p$--local finite spectra $F$ which have a self-map $v: \Susp^N F \to F$, $N \gg 0$, inducing multiplication by a unit in $K(d)$--homology.  These are called ``$v_d$--self--maps''.
\end{itemize}
Hopkins and Smith show the following classification theorem:

\begin{theorem}[Hopkins--Smith]\citeme{Hopkins--Smith}
Any thick subcategory $\CatOf C$ of $p$--local finite spectra must be $\CatOf C_d$ for some $d$.
\end{theorem}
\begin{proof}
\todo{I'm not totally convinced of this reduction.}It is sufficient to show that any object $X \in \CatOf C$ with $X \in \CatOf C_d$ induces an inclusion $\CatOf C_d \subseteq \CatOf C$.
Let $Y \in \CatOf C_d$ be any other spectrum of type at least $d$.  Consider the endomorphism ring spectrum $R = F(X, X)$ and the fiber $f: F \to \S$ of its unit map.\todo{Make it clear what $f$ is. Draw the fiber sequence or something.}  The action of $f$ under $K(n)$--homology is an isomorphism exactly when $X$ is $K(n)$--acyclic, and because the $K(n)$--acyclicity of $X$ implies the $K(n)$--acyclicity of $Y$, it follows that $1 \sm f: Y \sm F \to Y \sm \S$ is always null on $K(n)$--homology for all $n$.  By a small variant of the local nilpotence detection theorem, it follows that \[Y \sm F^{\sm j} \xrightarrow{1 \sm f^{\sm j}} Y \sm \S^{\sm j}\] is null for $j \gg 0$, and hence that \[\operatorname{cofib}\left( Y \sm F^{\sm j} \xrightarrow{1 \sm f^{\sm j}} Y \sm \S^{\sm j} \right) \simeq Y \sm \operatorname{cofib} f^{\sm j} \simeq Y \vee (Y \sm \Susp F^{\sm j}),\] so that $Y$ is a retract.  However, consider the following smash version of the octahedral axiom: the factorization \[F \sm F^{\sm (j-1)} \xrightarrow{f \sm 1} \S \sm F^{\sm (j-1)} \xrightarrow{1 \sm f^{\sm (j-1)}} \S \sm \S^{\sm (j-1)}\] begets a cofiber sequence \[F \sm \operatorname{cofib} f^{\sm (j-1)} \to \operatorname{cofib} f^{\sm j} \to \operatorname{cofib} f \sm \S^{\sm (j-1)}.\]
Using $\operatorname{cofib}(f) = X \sm DX \in \CatOf C$, one can inductively show that $\operatorname{cofib}(f^{\sm j})$, hence $Y \sm \operatorname{cofib}(f^{\sm j})$, and hence $Y$ all belong to $\CatOf C$ as well.
\end{proof}

They also show the \emph{considerably} harder theorem:

\begin{theorem}[Hopkins--Smith]\citeme{Hopkins--Smith}
A $p$--local finite spectrum is $K(d-1)$--acyclic exactly when it admits a $v_d$--self--map.
\end{theorem}
\begin{proof}[Executive summary of proof]
Given the classification of thick subcategories, if a property is closed under thickness then one need only exhibit a single spectrum with the property to know that all the spectra in the thick subcategory it generates also all have that property.  Inductively, they manually construct finite spectra $M_0(p^{i_0}, v_1^{i_1}, \ldots, v_{d-1}^{i_{d-1}})$ for sufficiently large\footnote{Compare this asymptotic condition with the assertion yesterday that there is no root of $v: M_8(2) \to M_0(2)$.} indices $i_*$ which admit a self-map $v$ governed by a commuting square
\begin{center}
\begin{tikzcd}
BP_* M_{|v_d| i_d}(p^{i_0}, v_1^{i_1}, \ldots, v_{d-1}^{i_{d-1}}) \arrow{r}{v} \arrow[-,double]{d} & BP_* M_0(p^{i_0}, v_1^{i_1}, \ldots, v_{d-1}^{i_{d-1}}) \arrow[-,double]{d} \\
\Susp^{|v_d| i_d} BP_* / (p^{i_0}, v_1^{i_1}, \ldots, v_{d-1}^{i_{d-1}}) \arrow{r}{- \cdot v_d^{i_d}} & BP_* / (p^{i_0}, v_1^{i_1}, \ldots, v_{d-1}^{i_{d-1}}).
\end{tikzcd}
\end{center}
These maps are guaranteed by very careful study of Adams spectral sequences.
\end{proof}







\section{The spectrum of the stable category}

As part of a broad attempt to analyze a geometric object through its modules, Paul Balmer has demonstrated the following theorem:

\begin{definition}
Given a triangulated $\otimes$--category $\CatOf C$, define a thick subcategory $\CatOf C' \subseteq \CatOf C$ to be a \textit{$\otimes$--ideal} when it has the additional property that $x \in \CatOf C'$ forces $x \otimes y \in \CatOf C'$ for any $y \in \CatOf C$.  Moreover, $\CatOf C'$ is said to be \textit{prime} when $x \otimes y \in \CatOf C'$ forces at least one of $x \in \CatOf C'$ or $y \in \CatOf C'$.  Define the \textit{spectrum} of $\CatOf C$ to be its collection of prime $\otimes$--ideals, topologized so that $U(x) = \{\CatOf C' \mid x \in \CatOf C'\}$ form a basis of opens.
\end{definition}

\begin{theorem}[Balmer]
The spectrum of $D^{\perf}(\CatOf{Mod}_R)$ is naturally homeomorphic to the Zariski spectrum of $R$. \qed
\end{theorem}

Balmer's construction applies much more generally.  The category $\CatOf{Spectra}$ can be identified with $\CatOf{Modules}_{\S}$, and so one can attempt to compute the Balmer spectrum of $\CatOf{Modules}_{\S}^{\perf} = \CatOf{Spectra}^{\mathrm{fin}}$.  In fact, we just finished this.
\begin{theorem}
The Balmer spectrum of $\CatOf{Spectra}_{(p)}^{\mathrm{fin}}$ consists of the thick subcategories $\CatOf C_d$, and $\{\CatOf C_n\}_{n=0}^d$ are its open sets.
\end{theorem}
\begin{proof}
Using the characterization of $\CatOf C_d$ as the kernel of $K(d-1)_*$, we see that it is a prime $\otimes$--ideal: \[K(d-1)_*(X \sm Y) \cong K(d-1)_* X \otimes_{K(d-1)_*} K(d-1)_* Y\] is zero exactly when at least one of $X$ and $Y$ is $K(d-1)$--acyclic.
\end{proof}

In fact, our favorite functor\footnote{However, this functor is \emph{not} a map of triangulated categories, so this has to be interpreted lightly.}\todo{Make this a remark.} $MU_*: \CatOf{Spectra} \to \CatOf{QCoh}(\context{MU})$ induces a homeomorphism of the Balmer spectrum of $\CatOf{Spectra}^{\mathrm{fin}}$ to that of $\moduli{fg}$.

--------

Balmer's construction is remarkably successful at describing the most salient features of the stable category, but it falls a ways short of the rich ``spectrum'' object we've come to know from algebraic geometry.  In particular, we have only a topological space, and not anything like a locally ringed space (or a space otherwise equipped locally with algebraic data).  It's also totally unclear why $MU$ plays such an important mediating role between geometry (i.e., the stable category) and algebra (i.e., the moduli of formal groups).
%\footnote{We will address this in our situation, but in general this is an open question: given a ring spectrum $R$, how can one recognize these local categories of spectra in terms of $R$, without reference to auxiliary spectra like $MU$?  Or, just as importantly: what makes $MU$ a special $\S$--algebra?}
Nonetheless, taking that as granted, we can use Bousfield's theory of homological localization to access ``local'' categories of spectra of the sort that a sheaf of local rings would supply us with.

\begin{theorem}[Bousfield]
Let $R_*$ denote the homology theory associated to a flat map $j: \Spec R \to \moduli{fg}$ by Landweber's theorem.  There is then a diagram\footnote{The meat of this theorem is in overcoming set-theoretic difficulties in the construction of $\CatOf{Spectra}_R$.  Bousfield accomplished this by describing a model structure on $\CatOf{Spectra}$ for which $R$--equivalences create the weak--equivalences.}
\begin{center}
\begin{tikzcd}[column sep=2.2cm,row sep=2cm]
\CatOf{Spectra}_R \arrow[red]{r}{R_* \quad \mathrm{conservative}} \arrow[leftarrow, shift left=0.20cm, red]{d}{L_R} & \CatOf{QCoh}(\Spec R) \arrow[shift left=0.20cm, red, leftarrow]{d}{j^*} \\
\CatOf{Spectra} \arrow[leftarrow,shift left=0.20cm, "\dashv"']{u}{i} \arrow[red]{ru}{R_*} \arrow[red]{r}{MU_*} & \CatOf{QCoh}(\context{MU}), \arrow[leftarrow, shift left=0.20cm, "\dashv"']{u}{j_*}
\end{tikzcd}
\end{center}
such that $i$ is left-adjoint to $L_R$, $j^*$ is left-adjoint to $j_*$, $i$ and $j_*$ are inclusions of full subcategories, the red composites are all equal, and $R_*$ is conservative on $\CatOf{Spectra}_R$. \qed
\end{theorem}

In the case when $R$ models the inclusion of the deformation space around the point $\Gamma_d$, we will denote the localizer by \[\CatOf{Spectra} \xrightarrow{\widehat L_d} \CatOf{Spectra}_{\Gamma_d}.\]  In the case when $R$ models the inclusion of the open complement of the unique closed substack of codimension $d$, we will denote the localizer by \[\CatOf{Spectra} \xrightarrow{L_d} \CatOf{Spectra}_d = \CatOf{Spectra}_{\moduli{fg}^{\le d}}.\]  We have set up our situation so that the following properties of these localizations either have easy proofs or are intuitive from the algebraic analogue of $j^* \vdash j_*$:
\begin{enumerate}
\item There is an equivalence \[L_d X \simeq (L_d \S) \sm X,\] analogous to $j^* M \simeq R \otimes M$ in the algebraic setting.  Because $L_{K(d)}$ is associated to the inclusion of a formal scheme (i.e., an ind-finite scheme), it has the formula \[\widehat L_d X \simeq \lim_I \left( M_0(v^I) \sm L_d X \right)\] analogous to $j^* M \simeq \lim_j (R/I^j \otimes M)$ in the complete algebraic setting.\citeme{Ravenel (and Hopkins)}
\item Because the open substack of dimension $d$ properly contains both the open substack of dimension $(d-1)$ and the infinitesimal deformation neighborhood of the closed point of height $d$, there are natural factorizations
\begin{align*}
\operatorname{id} \to L_d \to L_{d-1}, & & \operatorname{id} \to L_d \to \widehat L_d.
\end{align*}
In particular, $L_d X = 0$ implies both $L_{d-1} X = 0$ and $\widehat L_d X = 0$.
\item The inclusion of the open substack of dimension $d-1$ into the one of dimension $d$ has relatively closed complement the point of height $d$.  Algebraically, this gives a gluing square (or Mayer-Vietoris square), and this is reflected in homotopy theory by a homotopy pullback square (the chromatic fracture square):\todo{Write in a proof for this.}
\begin{center}
\begin{tikzcd}
L_d \arrow{r} \arrow{d} \arrow[dr, phantom, "\lrcorner", very near start] & \widehat L_d \arrow{d} \\
L_{d-1} \arrow{r} & L_{d-1} \widehat L_d.
\end{tikzcd}
\end{center}
\end{enumerate}

There are also considerably more complicated facts known about these functors:
\begin{theorem}[Hopkins--Ravenel]
The homotopy limit of the tower \[\cdots \to L_d F \to L_{d-1} F \to \cdots \to L_1 F \to L_0 F\] recovers the $p$--local homotopy type of any finite spectrum $F$.\footnote{Spectra satisfying this limit property are said to be \textit{chromatically complete}, which is closely related to being \textit{harmonic}, i.e., being local with respect to $\bigvee_{d=0}^\infty K(d)$.  (I believe this a joke about ``music of the spheres''.)  It is known that nice Thom spectra (and in particular every suspension and finite spectrum) is harmonic, that every finite spectrum is chromatically complete, and that there exist some harmonic spectra which are not chromatically complete.}
\end{theorem}

\noindent This suggests a productive method for analyzing the homotopy groups of spheres: study the homotopy groups of each $L_d \S$ and perform the reassembly process encoded by this inverse limit.  Using the fracture square, one sees that it is also profitable to consider the homotopy groups of $\widehat L_d \S$.  In fact, the spectral version of $\context{E}(F)$ considered on the first day furnishes us with a tool by which we can approach this:\todo{You can rework this now, since you've already talked about the Adams spectral sequence.}

\begin{theorem}[Bousfield, et al.]
The coskeletal filtration of $\mathcal D_E(F)$ gives a spectral sequence converging to the homotopy of its totalization, $F^\wedge_E$.  When $F$ is finite and $E$ models either of the cases above, this spectral sequence converges to $\pi_* L_E F$.  Furthermore, there is a line bundle $\omega$ on $\context{E}$ such that\footnote{The identification of the $E_2$--page as computing stack cohomology is the first place where we really mean to employ the full technology of stacks in this talk.  Everywhere else, we have been essentially content to speak of simplicial presheaves.} \[E_2^{*, *} = H^*_{\mathrm{stack}}(\context{E}; \context{E}(F) \otimes \omega^{\otimes *}) \Rightarrow \pi_* L_E F.\]
\end{theorem}

The utility of this theorem is in the identification with stack cohomology.  In the case $E = E_{\Gamma_d}$, recall that $\context{E_{\Gamma_d}}[0]$ is a smooth infinitesimal thickening of the spectrum of a field, so that \[\context{E_{\Gamma_d}} = \left( \moduli{fg} \right)^\wedge_{\Gamma_d} \simeq \widehat{\mathbb A}^{d-1}_{\mathbb W(k)} \mmod \underline{\operatorname{Aut}}(\Gamma_d)\] as in the first example of $E = H\F_2$ on the first day.  But, in this specific case, there is an identification of stack cohomology with group cohomology: \[H^*_{\mathrm{stack}}(* \mmod \underline{G}; \mathcal M) = H^*_{\mathrm{group}}(G; M).\]  Another theorem from the arithmetic geometry literature gives \[\operatorname{Aut}(\Gamma_d) \cong \left( \mathbb{W}(k)\langle S \rangle \middle/ \left( \begin{array}{c} Sw = w^\phi S, \\ S^d = p \end{array} \right) \right)^\times,\] and so we have reduced the computation of all of the stable homotopy groups of spheres to a very difficult problem in profinite group cohomology --- but one which is arithmetically founded, so that arithmetic geometry might continue to lend a hand. \oweproof{Form of the stabilizer group}

\begin{example}[Adams]
In the case $d = 1$, $\operatorname{Aut}(\Gamma_1) = \Z_p^\times$ and it acts on $\pi_* E_1 = \Z_p[u^\pm]$ by $\gamma \cdot u^n \mapsto \gamma^n u^n$.  At odd primes $p$ (so that $p$ is coprime to the torsion part of $\Z_p^\times$), one computes \[H^s(\operatorname{Aut}(\Gamma_1); \pi_* E_1) = \begin{cases}\Z_p & \text{when $s = 0$}, \\ \bigoplus_{j = 2(p-1)k} \Z_p\{u^j\} / (pk u^j) & \text{when $s = 1$}, \\ 0 & \text{otherwise}. \end{cases}\]  This, in turn, gives the calculation \[\pi_t \widehat L_1 \S^0 = \begin{cases} \Z_p & \text{when $t = 0$}, \\ \Z_p / (pk) & \text{when $t = t|v_1| - 1$}, \\ 0 & \text{otherwise}. \end{cases}\]  These groups are familiar to homotopy theorists: the $J$--homomorphism $J: BU \to BF$ described on the first day selects exactly these elements (for nonnegative $t$).
\end{example}





\oldsection*{Stuff that might belong in this chapter}

Mention that complex orientations cause $\context{MU}(E)$ to degenerate, so that stackiness of $\context{MU}(E)$ is a measure of the failure of orientability.  Mention also Wood's cofiber sequence?  Talk about Hopkins's stacky pullback theorem from his Talbot lecture?  I think we can only do this \emph{after} Quillen's theorem has been proven, though it only really fits conceptually into the lecture on contexts.

Achim's nice proof of Hopf invariant 1?  The $E(1)$ version?  Ravenel's very odd proof of the Kervaire problem?
